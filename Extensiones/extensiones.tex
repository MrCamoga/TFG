% !TeX root=../tfg2.tex

\chapter{Extensiones de grupos}

% introduccion explicar qué se va a hacer, mencionar que se darán interpretaciones de los grupos H^1 y H^2. comentar si eso que el H^3
El objetivo de este capítulo es estudiar todas las formas de construir un grupo $E$ a partir de dos grupos $Q$ y $N$, de tal forma que $E$ tenga un subgrupo normal isomorfo a $N$ y cuyo cociente sea isomorfo $Q$. Para ello, definiremos una relación de equivalencia sobre el conjunto de extensiones y daremos una clasificación usando los grupos de cohomología $H^1$ y $H^2$ (Teoremas \ref{h1} y \ref{h2}).

\section{Equivalencia de extensiones}

% comentar que ya se ha explicado en el cpaitulo anterior las sucesiones exactas etc

\begin{definicion}
	Una extensión de un grupo $Q$ por un grupo $N$ es una sucesión exacta corta
	\begin{equation}\label{eq:ext}
		\extension i \pi N E Q 
	\end{equation}
	Al grupo $N$ lo llamaremos kernel de la extensión, al grupo $Q$ el cociente de la extensión y a $E$ el grupo central de la extensión.
	
	Decimos que otra extensión $\extension {i'} {\pi'} N {E'} Q $ es equivalente si existe un homomorfismo $\homo f E {E'}$ tal que el siguiente diagrama conmuta

	%	\begin{figure}[h!] % quitar figure?
	%		\centering	
			% https://q.uiver.app/?q=WzAsMTAsWzEsMCwiTiJdLFsyLDAsIkUiXSxbMywwLCJRIl0sWzQsMCwiMSJdLFswLDAsIjEiXSxbMCwxLCIxIl0sWzEsMSwiTiJdLFszLDEsIlEiXSxbNCwxLCIxIl0sWzIsMSwiRSciXSxbNCwwXSxbMCwxLCJpIl0sWzEsMiwiXFxwaSJdLFsyLDNdLFs1LDZdLFs2LDksImknIl0sWzksNywiXFxwaSciXSxbNyw4XSxbMiw3LCIiLDEseyJvZmZzZXQiOi0xLCJzdHlsZSI6eyJoZWFkIjp7Im5hbWUiOiJub25lIn19fV0sWzcsMiwiIiwxLHsib2Zmc2V0IjotMSwic3R5bGUiOnsiaGVhZCI6eyJuYW1lIjoibm9uZSJ9fX1dLFsxLDksImYiXSxbMCw2LCIiLDEseyJvZmZzZXQiOi0xLCJzdHlsZSI6eyJoZWFkIjp7Im5hbWUiOiJub25lIn19fV0sWzYsMCwiIiwxLHsib2Zmc2V0IjotMSwic3R5bGUiOnsiaGVhZCI6eyJuYW1lIjoibm9uZSJ9fX1dXQ==
	\begin{equation}\begin{tikzcd}
		1 & N & E & Q & 1 \\
		1 & N & {E'} & Q & 1
		\arrow[from=1-1, to=1-2]
		\arrow["i", from=1-2, to=1-3]
		\arrow["\pi", from=1-3, to=1-4]
		\arrow[from=1-4, to=1-5]
		\arrow[from=2-1, to=2-2]
		\arrow["{i'}", from=2-2, to=2-3]
		\arrow["{\pi'}", from=2-3, to=2-4]
		\arrow[from=2-4, to=2-5]
		\arrow[shift left=1, no head, from=1-4, to=2-4]
		\arrow[shift left=1, no head, from=2-4, to=1-4]
		\arrow["f", from=1-3, to=2-3]
		\arrow[shift left=1, no head, from=1-2, to=2-2]
		\arrow[shift left=1, no head, from=2-2, to=1-2]
	\end{tikzcd}
	\end{equation}
	%\end{figure}
\end{definicion}

El interés está en clasificar todas las extensiones de $Q$ por $N$. La equivalencia de extensiones captura la idea de que los grupos centrales se construyen a partir de $N$ y $Q$ esencialmente de la misma forma, ya que $N$ y $Q$ se incluyen y proyectan de la misma manera. % repetitivo: de la misma forma/manera

A continuación se dan unos ejemplos conocidos de extensiones de grupos.

\begin{ejemplo}
	El grupo diédrico $D_{2n} = \langle a,b\ |\ a^n = b^2 = 1, \ a^b = a^{-1}\rangle$ es una extensión de $C_2=\langle b\rangle$ por $C_n=\langle a\rangle$.
\end{ejemplo}

\begin{ejemplo}
	El grupo alternado $A_n$ se define como el kernel del homomorfimo del signo $\homo {\sgn} {S_n} {\{1,-1\}^\times}$ que manda una permutación a $1$ si es producto de un número par de transposiciones, y a $-1$ si es un número impar. Por tanto, $S_n$ es una extensión de $C_2$ por $A_n$.
	\begin{equation*}
		\extension i {sgn} {A_n} {S_n} {C_2}
	\end{equation*}
\end{ejemplo}

\begin{ejemplo}
	Siguiendo el ejemplo anterior, dado un cuerpo $\mathbb K$ y $n\in \Z^+$, el grupo especial lineal $\SL_n(\mathbb K)$ es el grupo de las matrices $n\times n$ sobre $\mathbb K$ con determinante $1$. $\SL_n(\mathbb K)$ es precisamente el kernel del homomorfismo determinante $\homo \det {\GL_n(\mathbb K)} {\mathbb K^\times}$. Así, $\GL_n(\mathbb K)$ es una extensión de $\mathbb K^\times$ por $\SL_n(\mathbb K)$.
	\begin{equation*}
		\extension i \det {\SL_n(\mathbb K)} {\GL_n(\mathbb K)} {\mathbb K^\times}
	\end{equation*}
\end{ejemplo}

\begin{ejemplo}
	Dados dos grupos $G$ y $H$, se construye la extensión trivial $\extension i \pi  G {G\times H} H$ con $G\times H$ el producto directo e $i$ y $\pi$ la inclusión y proyección canónicas.
\end{ejemplo}

\begin{ejemplo}
	Más generalmente, dados dos grupos $G$ y $N$ y una acción de grupos $\homo \varphi G {Aut(N)}$, el producto semidirecto $N\rtimes_\varphi G$ es una extensión de $G$ por $N$.
\end{ejemplo}

% reeescribir demostracion
\begin{observacion}\label{prop:eqiso}
	Por la Proposición \ref{prop:sflem}, dos extensiones equivalentes $\extension {i_{j}} {\pi_j} N {E_j} Q$ para $j=1,2$ dan lugar a grupos centrales isomorfos.
\end{observacion}

El siguiente ejemplo muestra que el recíproco no es cierto, y por tanto la equivalencia de extensiones es más fuerte que tener grupos centrales isomorfos. 
% ejemplo, poner dem en apendice?
\begin{ejemplo}
	Las extensiones $\extension {\times 3} {\times n} {\Z_3} {\Z_9} {\Z_3}$ para $n=1,2$ no son equivalentes.
	% https://q.uiver.app/?q=WzAsMTAsWzAsMCwiMSJdLFsxLDAsIlxcbWF0aGJie1p9XzMiXSxbMiwwLCJcXG1hdGhiYntafV85Il0sWzMsMCwiXFxtYXRoYmJ7Wn1fMyJdLFs0LDAsIjEiXSxbMCwxLCIxIl0sWzEsMSwiXFxtYXRoYmJ7Wn1fMyJdLFsyLDEsIlxcbWF0aGJie1p9XzkiXSxbMywxLCJcXG1hdGhiYntafV8zIl0sWzQsMSwiMSJdLFswLDFdLFsxLDIsIlxcdGltZXMgMyJdLFsyLDMsIlxcdGltZXMgMSJdLFszLDRdLFs1LDZdLFs2LDcsIlxcdGltZXMgMyJdLFs3LDgsIlxcdGltZXMgMiJdLFs4LDldLFsxLDYsIiIsMSx7Im9mZnNldCI6LTEsInN0eWxlIjp7ImhlYWQiOnsibmFtZSI6Im5vbmUifX19XSxbNiwxLCIiLDEseyJvZmZzZXQiOi0xLCJzdHlsZSI6eyJoZWFkIjp7Im5hbWUiOiJub25lIn19fV0sWzIsNywiZiIsMl0sWzMsOCwiIiwxLHsib2Zmc2V0IjotMSwic3R5bGUiOnsiaGVhZCI6eyJuYW1lIjoibm9uZSJ9fX1dLFs4LDMsIiIsMSx7Im9mZnNldCI6LTEsInN0eWxlIjp7ImhlYWQiOnsibmFtZSI6Im5vbmUifX19XV0=
%\[\begin{tikzcd}
%	1 & {\Z_3} & {\Z_9} & {\Z_3} & 1 \\
%	1 & {\Z_3} & {\Z_9} & {\Z_3} & 1
%	\arrow[from=1-1, to=1-2]
%	\arrow["{\times 3}", from=1-2, to=1-3]
%	\arrow["{\times 1}", from=1-3, to=1-4]
%	\arrow[from=1-4, to=1-5]
%	\arrow[from=2-1, to=2-2]
%	\arrow["{\times 3}", from=2-2, to=2-3]
%	\arrow["{\times 2}", from=2-3, to=2-4]
%	\arrow[from=2-4, to=2-5]
%	%\arrow[shift left=1, no head, from=1-2, to=2-2]
%	%\arrow[shift left=1, no head, from=2-2, to=1-2]
%	%\arrow["f"', from=1-3, to=2-3]
%	%\arrow[shift left=1, no head, from=1-4, to=2-4]
%	%\arrow[shift left=1, no head, from=2-4, to=1-4]
%\end{tikzcd}\]
	\begin{demostracion} %$\Z_9$ tiene un único subgrupo de orden $3$ y por tanto la inclusión es única. Para la proyección, se debe mandar el $\bar1$ de $\Z_9$ al $\bar1$ o $\bar2$ de $\Z_3$, lo que da lugar a las dos extensiones mostradas.
		Lo probamos por reducción al absurdo. Supongamos que existe $\homo f {\Z_9} {\Z_9}$ que haga a las dos extensiones equivalentes. Un automorfismo $f$ de $\Z_9$ viene dado por $f(x) = kx$ con $x\in \Z_9$ y $k\in \Z_9^{\times} = \{1,2,4,5,7,8\}$		 
		
		Para que $\pi_2\circ f = \pi_1$, $(\times2\circ f)(x) = 2kx = x \mod 3$, $k \equiv 2 \mod 3$, por lo que $k = 2,5,8$.
		
		Por otro lado, para que $f\circ i_1 = i_2$, $(f\circ \times3)(x) = 3kx = 3x \mod 9$, por lo que $k = 1,4,7$.
		
		Por tanto, no existe un isomorfismo $f$ que haga al diagrama conmutativo y las extensiones no son equivalentes.
	\end{demostracion}
\end{ejemplo}

\begin{proposicion}
	La equivalencia de extensiones es una relación de equivalencia.
	\begin{demostracion}
		\begin{enumerate}
			\item Reflexiva: $E$ es equivalente a $E$ tomando $f=1_E$.
			\item Simétrica: Si $\homo f {E_1} {E_2}$ es una equivalencia, por la Observación \eqref{prop:eqiso}, $\homo {f^{-1}} {E_2} {E_1}$ es una equivalencia. 
			\item Transitiva: Si $\homo f {E_1} {E_2}$ y $\homo g {E_2} {E_3}$ son equivalencias, $g\circ f\circ i_1 = g\circ i_2 = i_3$ y $\pi_1 \circ g \circ f = \pi_2 \circ f = \pi_3$, entonces $\homo {g\circ f} {E_1} {E_3}$ es una equivalencia.
		\end{enumerate}
	\end{demostracion}
\end{proposicion}

El estudio de las extensiones lo haremos haciendo uso de secciones, que se definen a continuación. La idea será tomar una sección y tratar de construir extensiones equivalentes definiendo una operación de grupo sobre el producto $N\times Q$.

\begin{definicion}
	Sea $\homo \pi A B$ un homomorfismo de grupos, una sección $s$ de $\pi$ es una inversa a la derecha de $\pi$, esto es, $\homo s B A$ tal que $\pi \circ s = 1_B$.
\end{definicion}

\begin{observacion}\label{extact}
	Una extensión $\extension i \pi N E Q$ determina, por conjugación por elementos de $E$, un homomorfismo $\homo \alpha E {\Aut(N)}$ definido por
	\begin{equation*} % tecnicamente es i(n) e i(N) pero bueno es inyectiva noseque da igual imagino no creo q se enfade nadie por hacer un pequeño abuso de notacion
		\alpha(g)(n) = \,^gn = gng^{-1}
	\end{equation*}
	
	Entonces, $\alpha(N) = \Inn(N)$ y $\alpha$ induce un homomorfismo  % explicar que esta bien definida
	\begin{align*}
		\tilde\alpha\colon E/N &\to \Out(N) \\
		gN &\mapsto \overline{\alpha(g)} 
	\end{align*}
	
	El homomorfismo $\tilde\alpha$ se conoce como el kernel abstracto de la extensión. % no se usa
	Fijando una sección $s$ de $\pi$, para todo $q\in Q$, la conjugación por $s(q)$ determina un automorfismo $\varphi(s(q))$ de $N$ definido por $\varphi(s(q))(n) = \alpha(s(q))(n)$. 
	
	Notese que la función $\homo \varphi Q {\Aut(N)}$ no es necesariamente un homomorfismo de grupos, pero sí lo es salvo automorfismos internos. En particular, si la sección $s$ es un homomorfismo o el grupo de automorfismos internos de $N$ es trivial, como se estudia en las secciones \ref{sec:split} y \ref{sec:ab}, entonces $\varphi$ sí es un homomorfismo y podremos hablar de grupos de la acción de la extensión. %%?????????????????
\end{observacion}

\begin{observacion}
	Dos extensiones $E$ y $E'$ equivalentes dan lugar a un mismo kernel abstracto.
	\begin{demostracion}
	 	$f\left(i(n)^{s(q)}\right) = i'(n)^{s'(q)}$
	 \end{demostracion}
\end{observacion}

Por ello, para estudiar las extensiones salvo equivalencia procederemos de la siguiente forma.
\begin{enumerate}
	\item Determinar todas las acciones externas de $Q$ en $N$ que dan lugar a una extensión de grupos de $Q$ por $N$.
	\item Para cada acción, construir todas las extensiones de $Q$ por $N$ que dan lugar a esa acción.
	\item Ver cuáles de ellas son equivalentes.
\end{enumerate}
% podemos fijar una acción $\homo \varphi Q {\Aut(N)}$ y estudiar las extensiones que dan lugar a esa acción.

\begin{definicion}
	Sean $Q$ y $N$ grupos y $\varphi$ una acción de $Q$ en $N$. Denotamos por $\Ext_\varphi(Q,N)$ al conjunto de clases de extensiones equivalentes de $Q$ por $N$ que dan lugar a la acción $\varphi$.
\end{definicion}


