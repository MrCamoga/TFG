\documentclass[11pt,a4paper,twoside,spanish]{book}

\usepackage[utf8]{inputenc}
\usepackage{pkgs/TFG}

\usepackage{mathtools}
\usepackage{hyperref}
\usepackage{tikz,pgfplots,pkgs/quiver}
\usepackage{adjustbox}
\usetikzlibrary{positioning, babel}
\pgfplotsset{compat=1.18}

\usepackage{verbatim}


%\newcommand{\Z}{\mathbb{Z}}
%\newcommand{\N}{\mathbb{N}}


% dark theme
\usepackage{xcolor}
%\pagecolor[rgb]{0,0,0}
%\color[rgb]{1,1,1}



\DeclareMathOperator{\SylowSubgroup}{Syl}
\DeclareMathOperator{\HallSubgroup}{Hall}
\DeclareMathOperator{\OSubgroup}{O}
\DeclareMathOperator{\Normalizer}{N}
\DeclareMathOperator{\Center}{Z}
\DeclareMathOperator{\mcd}{mcd}
\DeclareMathOperator{\mcm}{mcm}
%\DeclareMathOperator{\Ker}{Ker}
\DeclareMathOperator{\Image}{Im}
%\DeclareMathOperator{\Hom}{Hom}
\DeclareMathOperator{\Aut}{Aut}
\DeclareMathOperator{\Inn}{Inn}
\DeclareMathOperator{\Out}{Out}
\DeclareMathOperator{\Ext}{Ext}
\DeclareMathOperator{\SL}{SL}
\DeclareMathOperator{\GL}{GL}
\DeclareMathOperator{\PGL}{PGL}
\DeclareMathOperator{\PSL}{PSL}
\DeclareMathOperator{\car}{car}
\DeclareMathOperator{\sgn}{sgn}
%\DeclareMathOperator{\det}{det}
%\DeclareMathOperator{\HomFunctor}{Hom}
\DeclareMathOperator{\ExtFunctor}{Ext}
\DeclareMathOperator{\TorFunctor}{Tor}
\DeclareMathOperator{\AbFunctor}{Ab}
\DeclareMathOperator{\Pletra}{P}

\newcommand{\FHom}[3]{\Hom_{#1}(#2,#3)}
\newcommand{\FTor}[3]{\TorFunctor_{#1}(#2,#3)}
\newcommand{\FExt}[3]{\Ext_{#1}(#2,#3)}
\newcommand{\Syl}[2]{\SylowSubgroup_{#1}(#2)}
\newcommand{\Hall}[2]{\HallSubgroup_{#1}(#2)}
\newcommand{\Core}[2]{\OSubgroup_{#1}(#2)}
\newcommand{\Ab}[1]{#1^\text{ab}}
\newcommand{\gensub}[1]{\left\langle#1\right\rangle}
\newcommand{\Norm}[2]{\Normalizer_{#1}(#2)}
\newcommand{\norm}{\trianglelefteq}
\newcommand{\ord}[1]{\left|#1\right|}%\vert
\newcommand{\homo}[3]{#1\colon #2\to #3}
\newcommand{\extension}[5]{1\xrightarrow{} #3 \xrightarrow{#1} #4\xrightarrow{#2} #5 \xrightarrow{} 1}
\newcommand{\transversal}[2]{\{#1_1,\ldots,#1_#2\}}
\newcommand{\transfer}[2]{\tau_{#1/ #2}}
\newcommand{\pretransfer}[2]{P_{#1/ #2}}
\newcommand{\ptransfer}[2]{\tilde\tau_{#1/ #2}}


\begin{document}
\renewcommand{\tablename}{Tabla}
\frontmatter

\titulo{Extensiones de grupos y teoremas de Hall}
\nombre{Carlos Moya García}
\director{Jon González Sánchez}
\otrodirector{}
\fecha{22 de junio de 2022}

\maketitle
\thispagestyle{empty}
\pagestyle{plain}
\tableofcontents
\clearpage{\pagestyle{empty}\cleardoublepage}
\pagestyle{fancy}
%\fancyhf{}
\renewcommand{\chaptermark}[1]{\markboth{#1}{}}
\fancyhead[LO]{\slshape\nouppercase{\leftmark}}
\fancyhead[RO]{\thepage}
\fancyhead[LE]{\thepage}
\fancyhead[RE]{\slshape\nouppercase{\rightmark}}

\incluir{Introduccion}

\mainmatter
\renewcommand{\chaptermark}[1]{\markboth{\chaptername\ \thechapter. #1}{}}

\incluir{Capitulos/cohomology.tex}
\incluir{Capitulos/extensiones.tex}
\incluir{Capitulos/hall.tex}
\incluir{Capitulos/transfer.tex}
%\incluir{Extensiones/split.tex}
%\incluir{Extensiones/abelian.tex}
%\incluir{Extensiones/central.tex}

\apendices
\incluir{Apendices/ejercicios.tex}
%% !TeX root=../tfg2.tex
\chapter{Normalización de un cociclo}\label{apen:norm}

Tomando una sección $s$ de la extensión $\extension i \pi A E Q$ y usando la Proposición \ref{prop:normcoc} si llamamos a $x = c(1,1)$, podemos normalizar el cociclo sumandole el $2$-coborde $b(q_1,q_2) = -q_1\cdot x$.

\begin{equation}
	\tilde c(q_1,q_2) = c(q_1,q_2) - q_1\cdot c(1,1) = s(q_1)s(q_2)s(q_1q_2)^{-1} - q_1\cdot s(1)
\end{equation}

\begin{align*}
	i(a_1)s(q_1)i(a_2)s(q_2) &= i(a_1)i(a_2)^{q_1}s(q_1)s(q_2) \\
	&= i(a_1 + q_1\cdot a_2 + c(q_1,q_2) - q_1\cdot x)
\end{align*}

% operacion

Operación definida en $A\times Q$ a partir de la operación de $E$
\begin{align*}
	(a_1,q_1)(a_2,q_2) = (a_1+q_1\cdot a_2 + c(q_1,q_2) - q_1\cdot x,q_1q_2)
\end{align*}
	
% identidad
Identidad
\begin{align*}
	&(0,1)(a,q) = (0 + 1\cdot a + c(1,q) - 1\cdot x,q) = (a,q) \\
	&(a,q)(0,1) = (a+q\cdot 0 + c(q,1) - q\cdot x,q) = (a,q)
\end{align*}
	
% inverso
Inverso
\begin{align*}
	(a,q)(x-q^{-1}\cdot c(q,q^{-1}) - q^{-1}\cdot a,q^{-1}) &= (a + q\cdot x - c(q,q^{-1}) -a + c(q,q^{-1}) - q\cdot x, 1) = (0,1) \\
	(x-q^{-1}\cdot c(q,q^{-1}) - q^{-1}\cdot a,q^{-1})(a,q) &= (x-q^{-1}\cdot c(q,q^{-1}) - q^{-1}\cdot a + q^{-1}\cdot a + c(q^{-1},q) - q^{-1}\cdot x, 1) =
\end{align*}

\begin{align*}
	&x -q^{-1}\cdot c(q,q^{-1}) + c(q^{-1},q) - q^{-1}\cdot x = \\
	&x - c(1,q^{-1}) + c(q^{-1},1) - c(q^{-1},q) + c(q^{-1},q) - q^{-1}\cdot x = \\
	&x - c(1,q^{-1}) + c(q^{-1},1) - q^{-1}\cdot x = \tilde c(q^{-1},1)-\tilde c(1,q^{-1}) = 0
\end{align*}
	
% asociatividad
Asociatividad
\begin{align*}
	&(a_1,q_1)(a_2,q_2)(a_3,q_3) =\\
	 &(a_1 + q_1\cdot a_2 + c(q_1,q_2) - q_1\cdot x,q_1q_2)(a_3,q_3) =\\
	  &(a_1 + q_1\cdot a_2 + c(q_1,q_2) - q_1\cdot x + q_1q_2\cdot a_3 + c(q_1q_2,q_3)-q_1q_2\cdot x,q_1q_2q_3)
\end{align*}

\begin{align*}
	&(a_1,q_1)(a_2,q_2)(a_3,q_3) = \\ 
	&(a_1,q_1)(a_2+q_2\cdot a_3 + c(q_2,q_3) - q_2\cdot x,q_2q_3) =\\
	&(a_1+q_1\cdot a_2 + q_1q_2\cdot a_3 + q_1\cdot c(q_2,q_3) - q_1q_2\cdot x + c(q_1,q_2q_3) - q_1\cdot x,q_1q_2q_3)
\end{align*}
\incluir{Apendices/baer.tex}



\incluir{Bibliografia}

\end{document}