% !TeX root=../tfg2.tex

\chapter{Transfer}

Sea $G$ un grupo y $H$ un subgrupo de $G$ de índice finito $n$. Dado un homomorfismo $\homo \theta H A$ sobre un grupo abeliano, el transfer permite definir un homomorfismo de $G$ sobre $A$. %El estudio de homomorfismos sobre grupos abelianos es de gran interés 

Para ello tomamos $\transversal t n$ un conjunto de representantes de las coclases a derecha de $H$ en $G$. La multiplicación a derecha de las coclases $Ht_i$ por elementos de $G$ nos define una acción
\begin{align}\label{minitransfer}
	G/H\times G &\to {G/H} \nonumber\\
	(Ht_i,g) &\mapsto Ht_ig = Ht_{(i)g},
\end{align}
donde $i\mapsto (i)g$ es una permutación de $S_n$. Se tiene entonces que $t_igt_{(i)g}^{-1}\in H$ para todo $g\in G$ y esto nos da una forma natural de definir una función sobre $G$ haciendo el producto siguiente que llamaremos pre-transfer de $G$ a $H$.
\begin{equation}
	{\pretransfer G H}(g)= \prod_{i=1}^n t_igt_{(i)g}^{-1}.
\end{equation}
El pre-transfer no es en general independiente del transversal ni un homomorfismo de grupos. Estos dos problemas se pueden solucionar componiendo el pre-transfer con un homomorfismo sobre un grupo abeliano.

\section{Homomorfismo del transfer}

\begin{definicion}
	Sea $G$ un grupo, $H$ un subgrupo de $G$ de índice finito $n$, $\{t_1,\ldots, t_n\}$ un transversal a derecha de $H$ a $G$ y $\homo \theta H A$ un homomorfismo sobre un grupo abeliano $A$. El transfer de $\theta$ se define como la composición de $\theta$ con el pre-transfer $\pretransfer G H$
	\begin{align}
		{\transfer G H} \colon 	G &\to A \nonumber \\
							g &\mapsto \prod_{i=1}^n \theta\left(t_igt_{(i)g}^{-1}\right).
	\end{align}
	
	%Dado que $A$ es abeliano, el orden de los factores no importa y $\transfer G H$ está bien definido.  % sacar fuera, poner arriba? nose
\end{definicion}

\begin{proposicion}
	El transfer $\homo {\transfer G H} G A$ es un homomorfismo que no depende de la elección del transversal.
	\begin{demostracion}
		Sean  $\transversal t n$ y  $\transversal s n$ dos transversales tales que $s_i = h_it_i$ con $h_i\in H$. Entonces, para todo $g\in G$
		\begin{align*}
			\prod_{i=1}^n \theta\left(s_igs_{(i)g}^{-1}\right) 
			&= \prod_{i=1}^n \theta\left(h_it_igt_{(i)g}^{-1}h_{(i)g}^{-1}\right) \\
			&= \prod_{i=1}^n \theta\left(t_igt_{(i)g}^{-1}\right)\theta\left(h_ih_{(i)g}^{-1}\right)\\
			&=\prod_{i=1}^n \theta\left(t_igt_{(i)g}^{-1}\right).
		\end{align*}
		De donde deducimos que el transfer no depende de la elección del transversal.
		% observar que las h se cancelan porque (i)g es una permutacion de los indices

		Para ver que es un homomorfismo, cogemos $x,y\in G$
		\begin{align*}
			\theta^*(xy) &= \prod_{i=1}^n \theta\left(t_ixyt_{(i)xy}^{-1}\right) \\
			&=  \prod_{i=1}^n \theta\left((t_ixt_{(i)x}^{-1})(t_{(i)x}yt_{(i)xy}^{-1})\right) \\
			&= \prod_{i=1}^n \theta\left(t_ixt_{(i)x}^{-1}\right)\theta\left(t_{(i)x}yt_{(i)xy}^{-1}\right)\\
			&= \theta^*(x)\theta^*(y).
		\end{align*}
		Por lo tanto el transfer es un homomorfismo, como queríamos demostrar.
	\end{demostracion}
\end{proposicion}

\begin{observacion}
	Sea $G$ un grupo y $\homo \tau G A$ un homomorfismo sobre un grupo abeliano $A$. Entonces, $G' \leq \Ker(\tau)$ y por tanto $\tau$ factoriza de forma única a través de la abelianización de $G$. Es decir, existe un único homomorfismo $\homo {\tilde\tau} {\Ab G} A $ que hace al siguiente diagrama conmutativo.
	% https://q.uiver.app/?q=WzAsMyxbMCwwLCJHIl0sWzIsMCwiQSJdLFswLDIsIkdee1xcdGV4dHthYn19Il0sWzAsMSwiXFx0aGV0YSJdLFswLDIsIlxccGkiLDJdLFsyLDEsIlxcZXhpc3RzIVxcdGlsZGVcXHRoZXRhIiwyXV0=
\[\begin{tikzcd}
	G && A \\
	\\
	{\Ab G}
	\arrow["\tau", from=1-1, to=1-3]
	\arrow["\AbFunctor"', from=1-1, to=3-1]
	\arrow["{\exists!\tilde\tau}"', from=3-1, to=1-3]
\end{tikzcd}\]
\end{observacion}

\section{Cálculo del transfer}

En ocasiones, dado un $x\in G$ se puede calcular el transfer de manera eficiente con una buena elección de los representantes de las coclases. El elemento $x$ actúa sobre las coclases $\transversal {Ht} n$ por multiplicación a derecha permutándolas. Las órbitas bajo esta acción tienen la siguiente forma

\begin{equation}
	\left\{Hs_i,\ldots,Hs_i^{l_i-1}\right\}
\end{equation}
con $l_i$ el menor entero positivo tal que $Hs_ix^{l_i} = Hs_i$.

Se tiene entonces que los elementos $s_ix^j$ con $i=1,\ldots,k$, $j=0,\ldots,l_i-1$ y $k$ el número de órbitas forman un transversal de $H$ a $G$. Además, por el teorema de la Órbita-Estabilizador, se tiene que $l_i \bigm| \ord{G}$.

% por favor quitar este trabalenguas
\begin{lema}
	Si $\transversal s k$ son representantes de las coclases de las órbitas de la acción de $x$, entonces
	\begin{equation}
		 \tau(x) = \prod_{i=1}^k \theta(\,^{s_i}x^{l_i}).
	\end{equation}
	\begin{demostracion}
		Miramos la contribución de la órbita de $s_i$ al transfer de $x$.
		\begin{align*}
			\prod_{j=0}^{l_i-1}\theta(s_ix^j x (s_ix^j)^{-1}) = \prod_{j=0}^{l_i-1}\theta(s_ixs_i^{-1}) = \theta(s_ixs_i^{-1})^{l_i} = \theta(\,^{s_i}x^{l_i}).
		\end{align*}
		
		Multiplicando las contribuciones de todas las órbitas se llega al resultado.
	\end{demostracion}
\end{lema}

\section{Transfer a un subgrupo}

Nos centramos a continuación en el caso en que $\theta$ es la abelianización del subgrupo $H$, es decir, $\homo \theta H {\Ab H} = H/H'$. En este caso, denotaremos al transfer por $\homo {\transfer G H} G {\Ab H}$ y diremos que $\transfer G H$ es el transfer de $G$ a $H$.


\begin{ejemplo}
	Si $G$ es abeliano y $H\leq G$ con $\ord{G:H} = n < \infty$, el transfer de $G$ a $H$ es $g \mapsto g^n$.
	\begin{demostracion}
		Sea $\transversal t n$ un transversal de $H$. Entonces para todo $g\in G$
		\begin{equation*}
			\prod_{i=1}^n t_ig(t_{(i)g})^{-1} = g^n \prod_{i=1}^n t_i(t_{(i)g})^{-1} = g^n.
		\end{equation*}
	\end{demostracion}
\end{ejemplo}

\begin{proposicion}
	Sea $G$ un grupo y $H\leq K \leq G$ tal que $\ord{G:H}$ es finito. Entonces, el transfer de $G$ a $H$ es $\transfer G H = \ptransfer K H \circ \transfer G K = \ptransfer K H \circ \ptransfer G K \circ \AbFunctor$.
	
	% POR FAVOR LAS CURSIVAS ESTAN RARAS
	
	% https://q.uiver.app/?q=WzAsNixbMCwwLCJHIl0sWzIsMCwiSyJdLFs0LDAsIkgiXSxbMCwyLCJHXntcXHRleHR7YWJ9fSJdLFsyLDIsIktee1xcdGV4dHthYn19Il0sWzQsMiwiSF57XFx0ZXh0e2FifX0iXSxbMCwxLCJQX3tHL0t9Il0sWzEsMiwiUF97Sy9IfSJdLFswLDMsIkFiIl0sWzEsNCwiQWIiXSxbMiw1LCJBYiJdLFszLDQsIlxcdGlsZGVcXHRhdV97Ry9LfSJdLFs0LDUsIlxcdGlsZGVcXHRhdV97Sy9IfSJdLFswLDQsIlxcdGF1X3tHL0t9Il0sWzEsNSwiXFx0YXVfe0svSH0iXV0=
\[\begin{tikzcd}
	G && K && H \\
	\\
	{\Ab G} && {\Ab K} && {\Ab H}
	\arrow["{\pretransfer G K}", from=1-1, to=1-3]
	\arrow["{\pretransfer K H}", from=1-3, to=1-5]
	\arrow["\AbFunctor", from=1-1, to=3-1]
	\arrow["\AbFunctor", from=1-3, to=3-3]
	\arrow["\AbFunctor", from=1-5, to=3-5]
	\arrow["{\ptransfer G K}", from=3-1, to=3-3]
	\arrow["{\ptransfer K H}", from=3-3, to=3-5]
	\arrow["{\transfer G K}", from=1-1, to=3-3]
	\arrow["{\transfer K H}", from=1-3, to=3-5]
\end{tikzcd}\]
	\begin{demostracion}
		Sean $\transversal g n$ y $\transversal k m$ transversales a derecha de $K$ a $G$ y de $H$ a $K$ respectivamente. Entonces $$ G = \bigsqcup_{i=1}^{n} K g_i = \bigsqcup_{i=1}^n\bigsqcup_{j=1}^m Hk_jg_i $$ y $\{k_jg_i \ : \ i=1,\ldots,n, \ j=1,\ldots m\}$ es un transversal de $H$ a $G$.
		
		Miramos como actúa un elemento $g$ en la coclase $Hk_jg_i$. 
		\begin{equation}\label{eq:transact}
			Hk_jg_ig = Hk_j g_ig(g_{(i)g}^{-1}g_{(i)g}) = Hk_{j} (g_igg_{(i)g}^{-1})g_{(i)g} = Hk_{(j)g_igg_{(i)g}^{-1}}g_{(i)g}. % problema: el g_ig no se puede aplicar a Hk_j salvo que esté en K
		\end{equation}
		
		Evaluando $\ptransfer K H \circ \transfer G K$ en $g$ y utilizando \eqref{eq:transact} obtenemos que	
		\begin{align*}
		{\ptransfer K H} \circ {\transfer G K (g)} &= {\ptransfer K H}\left(\prod_{i=1}^n K' g_i g g_{(i)g}^{-1}\right) \\
			&= \prod_{j=1}^m\prod_{i=1}^n H'k_j g_i g g_{(i)g}^{-1} k_{(j)g_i g g_{(i)g}^{-1}}^{-1}  \\
			&= \prod_{j=1}^m\prod_{i=1}^n H'k_j g_i g  (k_{(j)g_i g g_{(i)g}^{-1}} g_{(i)g})^{-1} \\
			%&= \prod_{j=1}^m\prod_{i=1}^n H'k_j g_i g  (k_{(j)g_{(i)g}})^{-1} \\
			%&= \prod_{j=1}^m\prod_{i=1}^n H'k_j g_i g  (k_{j}g_{(i)g})^{-1} \\
			%&= \prod_{j=1}^m\prod_{i=1}^n H'k_j g_i g  (k_{j}g_{i})_g^{-1} \\
			& = {\transfer G H}(g).
		\end{align*}
	\end{demostracion}
\end{proposicion}

\begin{proposicion}
	Sea $G$ un grupo y $H\leq G$ de índice finito $n$. Sea $\transfer G H$ el transfer de $G$ a $H$. Entonces $(i\circ {\ptransfer G H})(gG') = g^nG'$ donde $i$ es la inclusión de $\Ab H$ en $\Ab G$ definida por $i(hH') = hG'$.

% https://q.uiver.app/?q=WzAsMyxbMCwwLCJHIl0sWzAsMiwiXFxBYiBHIl0sWzIsMiwiXFxBYiBIIl0sWzEsMiwiXFxwdHJhbnNmZXIgRyBIIiwyLHsib2Zmc2V0IjoxfV0sWzIsMSwiaSIsMix7Im9mZnNldCI6MX1dLFswLDIsIlxcdHJhbnNmZXIgRyBIIl0sWzAsMSwiXFxBYkZ1bmN0b3IiLDJdXQ==
\[\begin{tikzcd}
	G \\
	\\
	{\Ab G} && {\Ab H}
	\arrow["{\ptransfer G H}"', shift right=1, from=3-1, to=3-3]
	\arrow["i"', shift right=1, from=3-3, to=3-1]
	\arrow["{\transfer G H}", from=1-1, to=3-3]
	\arrow["\AbFunctor"', from=1-1, to=3-1]
\end{tikzcd}\]
	
	\begin{demostracion}
		Sea $\transversal t n$ un transversal de $H$ y $\transfer G H$ el transfer de $G$ a $H$. %Dado $h\in H$, $t_iht_{(i)h}^{-1} \in H$ %(g) = \prod_{i=1}^n  t_igt_{(i)g}^{-1}H'
		
		\begin{align*}
		(i\circ {\ptransfer G H})(gG') 
		&= i(\prod_{i=1}^n  t_igt_{(i)g}^{-1}H') \\
		&= \prod_{i=1}^n  i(t_igt_{(i)g}^{-1}H') \\
		&= \prod_{i=1}^n  t_igt_{(i)g}^{-1}G' \\
		&= g^n\prod_{i=1}^n  t_it_{(i)g}^{-1}G'=g^nG'.
		\end{align*}
	\end{demostracion} 
\end{proposicion}

%\begin{proposicion}
%	Sea $G$ un grupo y $H$ un subgrupo abeliano de $G$ de índice finito $n$. Entonces, el transfer de $G$ a $H$ es $\theta(g)=g^n$
%	\begin{demostracion}
%		% https://q.uiver.app/?q=WzAsMyxbMCwwLCJHIl0sWzIsMCwiSCJdLFswLDIsIlxcQWJ7R30iXSxbMCwxLCJcXHRoZXRhIl0sWzAsMiwiXFxwaSIsMl0sWzIsMSwiXFx0aWxkZVxcdGhldGEiLDAseyJvZmZzZXQiOi0xfV0sWzEsMiwiaSIsMCx7Im9mZnNldCI6LTF9XV0=
%		\[\begin{tikzcd}
%			G && H \\
%			\\
%			{\Ab{G}}
%			\arrow["\theta", from=1-1, to=1-3]
%			\arrow["\pi"', from=1-1, to=3-1]
%			\arrow["\tilde\theta", shift left=1, from=3-1, to=1-3]
%			\arrow["i", shift left=1, from=1-3, to=3-1]
%		\end{tikzcd}\]
%		Usando que $\theta = \tilde\theta \circ \pi$, $(i\circ\theta)(g) = $
%	\end{demostracion}
%\end{proposicion}

\subsection{Transfer al centro}

\begin{proposicion}
	Sea $G$ un grupo y $H$ un subgrupo central de $G$ de índice $\ord{G:H}=n$. Entonces, el transfer de $G$ a $H$ es $\transfer G H (g) = g^{n}$.
	\begin{demostracion}
		$\,^{s_i}g^{l_i}\in H\leq Z(G)$ y conjugando por $s_i$ se llega a $g^{l_i}\in Z(G)$. Finalmente 
		\begin{equation}
			\prod_{i=1}^k \,^{s_i}g^{l_i} = \prod_{i=1}^k g^{l_i} = g^n.
		\end{equation}
	\end{demostracion}
\end{proposicion}

%Notese que en general, la función de elevar cada elemento a una potencia no es un homomorfismo salvo que $G$ sea abeliano. % blablabla soltar rollo sobre lo remarcable que es que elevar al indice del centro sí lo sea

\begin{proposicion}[Lema de Schreier]\label{prop:quotfg} % poner encima del teorema de schur
	Sea $G$ un grupo finitamente generado y $H$ un subgrupo de $G$ de índice finito $n$. Entonces $H$ es finitamente generado.
	\begin{demostracion}
		Sea $X$ un sistema generador finito de $G$ y $T=\{1=t_1,\ldots,t_n\}$ un transversal normalizada de $H$ a $G$. Definimos la función $\tau_{i}(g) = t_igt_{(i)g}^{-1}\in H$ como en \eqref{minitransfer}, de esta forma $t_i g = \tau_{i}(g)t_{(i)g}$.
		Demostramos por inducción en $k$ que todo producto $x=x_1\cdots x_k$ con $x_i \in \langle X \rangle = G$ se puede escribir de la forma $x=u_1\cdots u_k t_i$ con $u_j\in Y=\{\tau_i(x) : i=1,\ldots,n,\ x\in X\cup X^{-1}\}$ y $t_i\in T$. 
		
		Para el caso base, $k=1$, tenemos $x=x_1=t_1x_1 = \tau_1(x_1)t_{(1)x_1}$.
		
		Supongamos ahora que todo $y=y_1\cdots y_k$ con $y_1,\ldots, y_k\in \langle X \rangle$ se puede escribir como $y=u_1\cdots u_k t_i$ con $u_1\ldots, u_k \in Y$ y $t_i\in T$.
		 Sea $x=x_1\cdots x_{k+1}\in \langle X \rangle$, aplicando la hipótesis de inducción a $x_1\cdots x_k$
		\begin{equation*}
			x=u_1\cdots u_k t_i x_{k+1} = u_1\cdots u_k \tau_i(x_{k+1}) t_{(i)x_{k+1}}.
		\end{equation*}
		Y queda probado el resultado.
		
		Tomando ahora $h=x_1\cdots x_k \in H$ y aplicando el resultado, se tiene que $h=u_1\cdots u_k t_i$ y dado que $t_i = (u_1\cdots u_k)^{-1} h\in H$, debe ser $t_i=t_1=1$.
		
		
		
	
%		\begin{equation*}
%			h = t_1x_1\cdots x_k = \tau_{1}(x_1)t_{(1)x_1}x_2\cdots x_k = \prod_{i=1}^k\left[ \tau_{(1)\prod_{j=1}^{i}x_j}\Big(\prod_{j=1}^{i}x_j\Big)\right]t_{(1)h}
%		\end{equation*}
	\end{demostracion}
\end{proposicion}

\begin{corolario}[Schur]
	Sea $G$ un grupo y $Z(G)$ de índice finito $n$, entonces $G'$ es finito y $G'^n = \{1\}$.
	\begin{demostracion}
		Sea $C=Z(G)$ y $\transversal {Cg} n$ las coclases a derecha de $C$ en $G$. Sean $c_ig_i,c_jg_j\in G$, entonces $[c_ig_i,c_j,g_j] = [g_i,g_j]$ y $G'$ está finitamente generado. Por el Segundo Teorema de Isomorfía $\frac{G'}{G'\cap C} = \frac{G'C}{C}$ es finito y por la Proposición \ref{prop:quotfg}, $G'\cap C$ está finitamente generado.
		El transfer de $G$ a $C$ viene dado por $x\mapsto x^n$ y como $G'$ está contenido en el kernel, $G'^n = \{1\}$, en particular, $(G'\cap C)^n = \{1\}$. Por el teorema de clasificación de grupos abelianos finitamente generados, $G'\cap C$ es finito y como $\ord{G':G'\cap C}$ es finito se deduce que $G'$ es finito.
	\end{demostracion}
\end{corolario}


%\subsection{Transfer a un subgrupo de Sylow}
