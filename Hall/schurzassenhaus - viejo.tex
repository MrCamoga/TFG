% !TeX root=tfg.tex

% explicar que el subgrupo de orden m es un complemento de N
\begin{teorema}[Schur-Zassenhaus] Sea $G$ un grupo finito y $N\norm G$, $\ord{N}=n$ y $\ord{G:N}=m$ con $\mcd(n,m)=1$. Entonces $G$ contiene subgrupos de orden $m$ y son conjugados.
	\begin{demostracion}
		%Caso $N$ abeliano.
		% explicar que si existe un complemento de N éste debe ser isomorfo a Q
		\textit{(i) Existencia: $N$ abeliano}. Sea $Q = G/N$. Construimos una sección $\homo {s'} Q G$ que manda cada $x\in Q$ a un elemento de $x$. En principio la sección no es un homomorfismo y tenemos que $s'(xy)N = s'(x)s'(y)N$. Podemos definir una función $\homo c{Q\times Q}N$ de tal forma que $s'(x)s'(y) =s'(xy)c(x,y)$.
		
		% esto queda feo
		 Si tomamos otra sección $\homo s Q N$, tenemos $s(x)N = s'(x)N$ y por tanto $s(x)=s'(x)e(x)$, con $\homo e Q N$. Imponiendo que $s$ sea un homomorfismo:
		 
		 % esto queda aún mas feo
		 \begin{align*}
		 	s'(xy)e(xy) = s(xy) = s(x)s(y) &= s'(x)e(x)s'(y)e(y) \\ &= s'(x)s'(y)e(x)^{s'(y)}e(y) \\ &= s'(xy)c(x,y)e(x)^{s'(y)}e(y)	
		 \end{align*}
		 
		 \begin{equation}\label{eqn:rel1}
		 	c(x,y)^{-1} = e(xy)^{-1}e(x)^{s'(y)}e(y)
		 \end{equation}
		 
		 % y esto tambien
		 Aplicando la propiedad asociativa a $(s'(t)s'(x))s'(y) = s'(t)(s'(x)s'(y))$
		 
		  \begin{align*}
		 (s'(t)s'(x))s'(y) &= s'(tx)c(t,x)s'(y) \\ &= s'(tx)s'(y)c(t,x)^{s'(y)}  \\ &= s'(txy)c(tx,y)c(t,x)^{s'(y)} \\
		 s'(t)(s'(x)s'(y)) &= s'(t)s'(xy)c(x,y) \\ &= s'(txy)c(t,xy)c(x,y)
		  \end{align*}
		 
		 \begin{equation}\label{eqn:rel2}
		 	c(x,y) = c(t,xy)^{-1}c(t,x)^{s'(y)}c(tx,y)
		 \end{equation}
		 
		 % definir k
		 Podemos definir ${\displaystyle d(x) = \prod_{t\in Q} c(t,x)}$ y hacer el producto en \eqref{eqn:rel2} sobre $t$ para relacionarlo con \eqref{eqn:rel1}. Obtenemos $c(x,y)^{m} = d(xy)^{-1}d(x)^{s'(y)}d(y)$. Como $\mcd(n,m) = 1$, existe un $k\in \Z$ tal que $km\equiv 1 \mod n$. Elevando esta última ecuación a $-k$:
		 
		 \begin{equation}
		 	c(x,y)^{-1} = d(xy)^k\left(d(x)^{-k}\right)^{s'(y)}d(y)^{-k}
		 \end{equation}
		 
		 Finalmente podemos igualar $e(x) = d(x)^{-k}$ y escribir explícitamente $s$ en términos de $s'$: 
		 
		 \begin{equation}
		 	 {\displaystyle s(x) = s'(x)\prod_{t\in Q}c(t,x)^{-k} = s'(x)\prod_{t\in Q}\left(s'(t)s'(x)s'(tx)^{-1}\right)^{-k}}
		 \end{equation}
		 
		 % explicar por qué una seccion es inyectiva?
		 A partir de una sección cualquiera $s'$, hemos conseguido definir una sección $s$ que es un homomorfismo. Las secciones son inyectivas y por tanto $s(Q)$ es un subgrupo de $G$ de orden $m$. A continuación, veremos que cualesquiera dos subgrupos de orden $m$ son conjugados.
		 
		 % Conjugados ab.
		 Sean $H$ y $H^*$ dos subgrupos de orden $m$.
		 \textbf{TODO}
		 % Caso general existencia
		 
		 
		 
		 \textit{(ii) Existencia: Caso general}. Lo demostramos por inducción fuerte sobre $\ord G$. Podemos tomar como caso base $C_{p}$, que como es abeliano se cumple el resultado. Sea $p$ un divisor primo de $n$ y $P\in Syl_p(N)$. Sean $L=N_{G}(P)$ y $C = Z(P)$. 
		 Como $C$ es característico en $P$, tenemos que $C\norm L$. % explicar?
		 Por el argumento de Frattini $G=LN$.
		 Observamos que $\ord{L:N\cap L} = \ord{G:N} = m$ y $N\cap L \norm L$ por el Segundo Teorema de Isomorfía. Podemos aplicar la inducción sobre el grupo $L/C$ ya que $C\neq 1$ por ser $P$ un $p$-subgrupo.
		 
		 Por inducción, existe $H/C \leq L/C$ de orden $m$ y volviendo a $G$ tenemos el subgrupo $H$ de índice $m$ en $C$.
		 
		 Podemos aplicar el caso abeliano a $H$ y $C$ y concluir que existe un subgrupo $Q$ de orden $m$.
		 
		 \begin{tikzpicture}[node distance=2cm, scale=1, transform shape]
			\title{Diagrama}
			
			\node(G) {$G$};
			\node(N)		[below right = 1cm and 3cm of G] {$N$};
			\node(L)		[below = 1.5cm of G] {$L=N_{G}(P)$};
			\node(NcL)	[below = 1.5cm of N] {$N\cap L$};
			\node(P)		[below = 0.5cm of NcL] {$P$};
			\node(C)		[below = 1.5cm of NcL] {$C=Z(P)$};
			\node(H)		[below = 1.5cm of L] {$H$};
			\node(Q)		[below = 1.5cm of H] {$Q$};
			\node(C1)	[below = 1.5cm of C] {$1$};
			
			
			\draw(G)-- node[above]{$\norm$} node[below]{$m$}(N);
			\draw(G)--(L);
			\draw(N)--(NcL);
			\draw(L)-- node[above]{$\norm$} node[below]{$m$}(NcL);
			\draw(L)--(H);
			%\draw(NcL)--(C);
			\draw(NcL)--(P);
			\draw(P)--(C);
			\draw(H)-- node[above]{$\norm$} node[below]{$m$}(C);
			\draw(H)--(Q);
			\draw(Q)-- node[above]{$\norm$} node[below]{$m$}(C1);
			\draw(C)--(C1);

		\end{tikzpicture}
		 
	\end{demostracion}
\end{teorema}