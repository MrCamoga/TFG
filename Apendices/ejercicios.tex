% !TeX root=../tfg2.tex
\chapter{Ejercicios resueltos}

\begin{ejercicio}
	Estudiar las extensiones salvo equivalencia de $\Z_p$ por $\Z_p$. % y ver que $H^2(\Z_p,\Z_p) \cong \Z_p$.
	\begin{solucion}
		Los grupos de orden $p^2$ son abelianos y por tanto la extensión es central y tiene acción trivial.
		Observamos que si $[c]\in H^2(\Z_p,\Z_p)$, $p[c] = [0]$, por lo que $H^2(\Z_p,\Z_p)$ es un $p$-grupo abeliano de exponente $p$, es decir, isomorfo a $\Z_p^k$ para algún $k\in \N$.
		
		Contando el número de extensiones veremos que $0<k<2$ y por tanto, el número de extensiones equivalentes debe ser $p$.
		
		\textit{(Extensiones isomorfas a $\Z_p\times \Z_p$).}
		\begin{equation*}
			\extension i \pi {\Z_p} {\Z_p\times \Z_p} {\Z_p}
		\end{equation*}
		
		La proyección $\homo \pi {\Z_p\times \Z_p} {\Z_p}$ viene dada por las imágenes de los generadores $e_1=(1,0)$ y $e_2=(0,1)$. Como $\pi$ es sobreyectiva, $\pi(e_1)$ o $\pi(e_2)$ es distinto de la identidad, supongamos que $e_1\notin \Ker(\pi)$. Entonces $\langle e_1 \rangle$ es una escisión de $\pi$ y la extensión es trivial, por lo que se corresponde con el elemento neutro de $H^2$ y es única salvo equivalencia.
		
		\textit{(Extensiones isomorfas a $\Z_{p^2}$).}
		\begin{equation*}
			\extension {i_n} {\pi_n} {\Z_p} {\Z_{p^2}} {\Z_p}
		\end{equation*}
		
		$\Z_{p^2}$ es cíclico y $\Z_p$ debe incluirse en el único subgrupo de $\Z_{p^2}$ de orden $p$. 
		
		Hay $p-1$ formas de incluir $\Z_p$ en $\langle p \rangle \leq \Z_{p^2}$ que se corresponden con los distintos automorfismos de $\langle p \rangle$. Las inclusiones $i_n$ vienen dadas por $i_n(x) = pnx \mod p^2$ donde $n=1,\ldots,p-1$.
		
		Las proyecciones $\pi_n$ vienen dadas por la imagen del $1\in \Z_{p^2}$, que puede mandarse a cualquier elemento no trivial de $\Z_p$ y por tanto hay $p-1$ proyecciones distintas definidas por $\pi_n(1) = n \mod p$ para $n=1,\ldots, p-1$.
		
		En total, hay $(p-1)^2$ formas de componer las $i_n$ y $\pi_n$. Sumando la extensión trivial dan un total de $p^2-2p+2$, que es menor que $p^2$ y mayor que $1$ para todo primo $p$. 
		
		\textit{(Representantes de las extensiones y cociclos asociados).} 
		La primera extensión es única salvo equivalencia y podemos tomar la inclusión en la primera coordenada $i(x) = (x,0)$ y la proyección en la segunda $\pi(x,y) = y$. El cociclo asociado $c_0$ es trivial ya que la extension escinde.
		
		Para la segunda extensión podemos fijar la inclusión $i_1$ para simplificar cálculos. Sean $n,m\in \{1,\ldots,p-1\}$ distintos y tomemos las proyecciones $\pi_m$ y $\pi_n$. Supongamos que existe un homomorfismo $f$ que haga al siguiente diagrama conmutativo
		\[\begin{tikzcd}
			1 & {\Z_p} & {\Z_{p^2}} & {\Z_p} & 1 \\
			1 & {\Z_p} & {\Z_{p^2}} & {\Z_p} & 1
			\arrow[from=1-1, to=1-2]
			\arrow["{i_1}", from=1-2, to=1-3]
			\arrow["{\pi_m}", from=1-3, to=1-4]
			\arrow[from=1-4, to=1-5]
			\arrow[from=2-1, to=2-2]
			\arrow["{i_1}", from=2-2, to=2-3]
			\arrow["{\pi_n}", from=2-3, to=2-4]
			\arrow[from=2-4, to=2-5]
			\arrow[shift left=1, no head, from=1-2, to=2-2]
			\arrow[shift left=1, no head, from=2-2, to=1-2]
			\arrow["f"', from=1-3, to=2-3]
			\arrow[shift left=1, no head, from=1-4, to=2-4]
			\arrow[shift left=1, no head, from=2-4, to=1-4]
		\end{tikzcd}\]

		Por un lado, $f(i_1(1)) = i_1(1) = p$ y por tanto $f(p)=p$. Por otro lado, $\pi_n(f(1)) = \pi_m(1) = m$ y $f(1)\in \pi_n^{-1}(m) = n^{-1}m + \langle p\rangle$. Esto es absurdo ya que $f(p) = pf(1) = pn^{-1}m \neq p$ pero $n\neq m$. Esto prueba que las extensiones $(\Z_{p^2},i_1,\pi_n)$ son inequivalentes para todo $n=1,\ldots,p-1$.
		
		Una sección de $\pi_n$ es $s_n$ definida por $s_n(x) = n^{-1}x \mod p = \overline{n^{-1}x}\in \{0,\ldots,p-1\}$. En efecto, $\pi_n(s_n(x)) = \pi_n(\overline{n^{-1}x}) = nn^{-1}x = x$ para $x=0,\ldots,p-1$.
		El cociclo asociado a esta sección es 
		\begin{equation*}
			c_n(x,y) = i_1^{-1}(s_n(x)+s_n(y)-s_n(x+y)) = \frac{\overline{n^{-1}x}+\overline{n^{-1}y}-\overline{n^{-1}(x+y)}}{p} %= i_1^{-1}\left(\overline{n^{-1}x}+\overline{n^{-1}y}-\overline{n^{-1}(x+y)}\right) 
		\end{equation*}
		
		Como $H^2$ es cíclico, podemos tomar $c_1$, cuya expresión es sencilla, y generar el resto de los cociclos con él
		
		\[
		    c_1(x,y) = \frac{\overline{x} + \overline{y} - \overline{x+y}}{p} =  \begin{cases}
		        0 & \text{si } x+y < p\\
		        1 & \text{si } x+y \geq p
		        \end{cases}
		\]
		% comprobar si se corresponden con los pi_n o estan permutados por un n^-1 o algo asi
		\[
		    c_n(x,y) = nc_1(x,y) =  \begin{cases}
		        0 & \text{si } x+y < p\\
		        n & \text{si } x+y \geq p
		        \end{cases}
		\]
	\end{solucion}
\end{ejercicio}
