% !TeX root=../tfg2.tex
\chapter{Suma de Baer}\label{baersum}


\begin{lema}\label{prop:pullback}
	Dada una extensión $\extension i \pi A E Q$ y un homomorfismo $\homo \alpha {Q'} Q$, existe una extensión $\extension {} {} A {E_1} {Q'}$ única salvo equivalencia que hace al siguiente diagrama conmutativo.
	% https://q.uiver.app/?q=WzAsMTAsWzEsMCwiQSJdLFsyLDAsIkUiXSxbMywwLCJRIl0sWzQsMCwiMSJdLFswLDAsIjEiXSxbMCwxLCIxIl0sWzEsMSwiQSJdLFszLDEsIlEnIl0sWzQsMSwiMSJdLFsyLDEsIkUnIl0sWzQsMF0sWzAsMSwiaSJdLFsxLDIsIlxccGkiXSxbMiwzXSxbNSw2XSxbNiw5LCJpJyJdLFs5LDcsIlxccGknIl0sWzcsOF0sWzAsNiwiIiwxLHsib2Zmc2V0IjotMSwic3R5bGUiOnsiaGVhZCI6eyJuYW1lIjoibm9uZSJ9fX1dLFs2LDAsIiIsMSx7Im9mZnNldCI6LTEsInN0eWxlIjp7ImhlYWQiOnsibmFtZSI6Im5vbmUifX19XSxbNywyLCJcXGFscGhhIl0sWzksMSwiZiJdXQ==
	\[\begin{tikzcd}
		1 & A & E & Q & 1 \\
		1 & A & {E_1} & {Q'} & 1
		\arrow[from=1-1, to=1-2]
		\arrow["i", from=1-2, to=1-3]
		\arrow["\pi", from=1-3, to=1-4]
		\arrow[from=1-4, to=1-5]
		\arrow[from=2-1, to=2-2]
		\arrow["{i_1}", from=2-2, to=2-3]
		\arrow["{\pi_1}", from=2-3, to=2-4]
		\arrow[from=2-4, to=2-5]
		\arrow[shift left=1, no head, from=1-2, to=2-2]
		\arrow[shift left=1, no head, from=2-2, to=1-2]
		\arrow["\alpha", from=2-4, to=1-4]
		\arrow["f_1", from=2-3, to=1-3]
	\end{tikzcd}\]
	\begin{demostracion}
		Tomamos $E_1=E\times_Q Q' = \{(e,q)\in E\times Q \ | \ \pi(e) = \alpha(q)\}$ el pullback de $\pi$ y $\alpha$ con la inclusión $i_1$ y proyección $\pi_1$ definidas por
		\begin{gather*}
			i_1(a) = (i(a),1) \\
			\pi_1(e,q) = q
		\end{gather*}
		
		Entonces $f_1(e,q) = e$ hace al diagrama conmutativo, ya que 
		\begin{gather*}
			(f_1\circ i_1) (a) =  i(a)\\
			(\pi\circ f_1)(e,q) = \pi(e) = \alpha(q) = (\alpha\circ \pi_1)(e,q)
		\end{gather*}
			
		Finalmente comprobamos que es exacta
		\begin{equation*}
			\pi_1(i_1(a)) = \pi_1(a,1)=1
		\end{equation*}
		
		Sea $\extension {i_2} {\pi_2} A {E_2} {Q'}$ otra extensión que hace al diagrama conmutativo. Por la propiedad universal del pullback, existe $\homo f {E_2} {E_1}$ que hace al siguiente diagrama conmutativo
		% https://q.uiver.app/?q=WzAsMTYsWzQsMSwiMSJdLFswLDIsIjEiXSxbNCwyLCIxIl0sWzAsMSwiMSJdLFsxLDEsIkEiXSxbMywxLCJRIl0sWzEsMiwiQSJdLFsyLDIsIkVfMSJdLFszLDIsIlEnIl0sWzAsMCwiMSJdLFsxLDAsIkEiXSxbMiwwLCJFXzIiXSxbMywwLCJRJyJdLFs0LDAsIjEiXSxbMiwxLCJFIl0sWzcsMl0sWzMsNF0sWzEsNl0sWzYsNywiaV8xIl0sWzcsOCwiXFxwaV8xIl0sWzgsMl0sWzgsNSwiXFxhbHBoYSJdLFsxMiw1LCJcXGFscGhhIiwyXSxbOSwxMF0sWzEwLDExLCJpXzIiXSxbMTEsMTIsIlxccGlfMiJdLFsxMiwxM10sWzEwLDQsIiIsMSx7Im9mZnNldCI6LTEsInN0eWxlIjp7ImhlYWQiOnsibmFtZSI6Im5vbmUifX19XSxbNCwxMCwiIiwxLHsib2Zmc2V0IjotMSwic3R5bGUiOnsiaGVhZCI6eyJuYW1lIjoibm9uZSJ9fX1dLFs0LDYsIiIsMSx7Im9mZnNldCI6LTEsInN0eWxlIjp7ImhlYWQiOnsibmFtZSI6Im5vbmUifX19XSxbNiw0LCIiLDEseyJvZmZzZXQiOi0xLCJzdHlsZSI6eyJoZWFkIjp7Im5hbWUiOiJub25lIn19fV0sWzQsMTQsImkiXSxbMTQsNSwiXFxwaSJdLFs3LDE0LCJmXzEiXSxbMTEsMTQsImZfMiIsMl0sWzUsMF0sWzExLDcsImYiLDEseyJsYWJlbF9wb3NpdGlvbiI6MjAsImN1cnZlIjotMn1dLFsxMiw4LCIiLDEseyJvZmZzZXQiOi0xLCJjdXJ2ZSI6LTIsInN0eWxlIjp7ImhlYWQiOnsibmFtZSI6Im5vbmUifX19XSxbOCwxMiwiIiwxLHsib2Zmc2V0IjotMSwiY3VydmUiOjIsInN0eWxlIjp7ImhlYWQiOnsibmFtZSI6Im5vbmUifX19XV0=
		\[\begin{tikzcd}
			1 & A & {E_2} & {Q'} & 1 \\
			1 & A & E & Q & 1 \\
			1 & A & {E_1} & {Q'} & 1
			\arrow[from=2-1, to=2-2]
			\arrow[from=3-1, to=3-2]
			\arrow["{i_1}", from=3-2, to=3-3]
			\arrow["{\pi_1}", from=3-3, to=3-4]
			\arrow[from=3-4, to=3-5]
			\arrow["\alpha", from=3-4, to=2-4]
			\arrow["\alpha"', from=1-4, to=2-4]
			\arrow[from=1-1, to=1-2]
			\arrow["{i_2}", from=1-2, to=1-3]
			\arrow["{\pi_2}", from=1-3, to=1-4]
			\arrow[from=1-4, to=1-5]
			\arrow[shift left=1, no head, from=1-2, to=2-2]
			\arrow[shift left=1, no head, from=2-2, to=1-2]
			\arrow[shift left=1, no head, from=2-2, to=3-2]
			\arrow[shift left=1, no head, from=3-2, to=2-2]
			\arrow["i", from=2-2, to=2-3]
			\arrow["\pi", from=2-3, to=2-4]
			\arrow["{f_1}", from=3-3, to=2-3]
			\arrow["{f_2}"', from=1-3, to=2-3]
			\arrow[from=2-4, to=2-5]
			\arrow["f"{description, pos=0.2}, curve={height=-12pt}, from=1-3, to=3-3]
			\arrow[shift left=1, curve={height=-12pt}, no head, from=1-4, to=3-4]
			\arrow[shift left=1, curve={height=12pt}, no head, from=3-4, to=1-4]
		\end{tikzcd}\]
		y $E_1$ y $E_2$ son extensiones equivalentes.
	\end{demostracion}
\end{lema}

\begin{lema}\label{prop:pushout}
	Dada una extensión $\extension i \pi A E Q$ y un homomorfismo $\homo \beta {A} {A'}$, existe una extensión $\extension {} {} {A'} {E_1} Q$ única salvo equivalencia que hace al siguiente diagrama conmutativo.
	% https://q.uiver.app/?q=WzAsMTAsWzEsMCwiQSJdLFsyLDAsIkUiXSxbMywwLCJRIl0sWzQsMCwiMSJdLFswLDAsIjEiXSxbMCwxLCIxIl0sWzEsMSwiQSciXSxbMywxLCJRIl0sWzQsMSwiMSJdLFsyLDEsIkUnIl0sWzQsMF0sWzAsMSwiaSJdLFsxLDIsIlxccGkiXSxbMiwzXSxbNSw2XSxbNiw5LCJpJyJdLFs5LDcsIlxccGknIl0sWzcsOF0sWzEsOSwiZiJdLFs3LDIsIiIsMSx7Im9mZnNldCI6LTEsInN0eWxlIjp7ImhlYWQiOnsibmFtZSI6Im5vbmUifX19XSxbMiw3LCIiLDEseyJvZmZzZXQiOi0xLCJzdHlsZSI6eyJoZWFkIjp7Im5hbWUiOiJub25lIn19fV0sWzAsNiwiXFxiZXRhIl1d
	\[\begin{tikzcd}
		1 & A & E & Q & 1 \\
		1 & {A'} & {E_1} & Q & 1
		\arrow[from=1-1, to=1-2]
		\arrow["i", from=1-2, to=1-3]
		\arrow["\pi", from=1-3, to=1-4]
		\arrow[from=1-4, to=1-5]
		\arrow[from=2-1, to=2-2]
		\arrow["{i_1}", from=2-2, to=2-3]
		\arrow["{\pi_1}", from=2-3, to=2-4]
		\arrow[from=2-4, to=2-5]
		\arrow["{f_1}", from=1-3, to=2-3]
		\arrow[shift left=1, no head, from=2-4, to=1-4]
		\arrow[shift left=1, no head, from=1-4, to=2-4]
		\arrow["\beta", from=1-2, to=2-2]
	\end{tikzcd}\]
	\begin{demostracion}
		Tomamos $E_1=E\sqcup_A A' = \frac{E\times A'}{\{(i(a),-\beta(a))\in E\times A' \ : \ a\in A\}}$ el pushout de $i$ y $\beta$ con la inclusión $i_1$ y proyección $\pi_1$ definidas por
		\begin{gather*}
			i_1(a) = \overline{(1,a)} \\
			\pi_1(\overline{(e,a)}) = \pi(e)
		\end{gather*}
		
		Entonces $f_1(e) = \overline{(e,0)}$ hace al diagrama conmutativo, ya que 
		\begin{gather*}
			(i_1\circ \beta)(a) = \overline{(1,\beta(a))} = \overline{(i(a),0)} = (f_1\circ i)(a)\\
			(\pi_1\circ f_1)(e) = \pi(e)
		\end{gather*}
		
		Finalmente se prueba que la sucesión es exacta
		\begin{equation*}
			(\pi\circ i_1)(a) = \pi'(\overline{(1,a)}) = \pi(1) = 1
		 \end{equation*}
		 
		Para probar que es única salvo equivalencia, tomamos otra extensión $\extension {i_2} {\pi_2} {A'} {E_2} Q$ y por la propiedad universal del pushout, existe $\homo f {E_1}{E_2}$ tal que el siguiente diagrama conmuta
		% https://q.uiver.app/?q=WzAsMTYsWzAsMCwiMSJdLFswLDEsIjEiXSxbMCwyLCIxIl0sWzQsMiwiMSJdLFsxLDAsIkEnIl0sWzEsMSwiQSJdLFsyLDEsIkUiXSxbMywxLCJRIl0sWzMsMCwiUSJdLFszLDIsIlEiXSxbMiwwLCJFXzIiXSxbMSwyLCJBJyJdLFsyLDIsIkVfMSJdLFszLDNdLFs0LDAsIjEiXSxbNCwxLCIxIl0sWzEsNV0sWzUsNiwiaSJdLFs2LDcsIlxccGkiXSxbMCw0XSxbNCwxMCwiaV8yIl0sWzEwLDgsIlxccGlfMiJdLFsyLDExXSxbMTEsMTIsImlfMSJdLFsxMiw5LCJcXHBpXzEiXSxbOSwzXSxbNSw0LCJcXGJldGEiLDJdLFs1LDExLCJcXGJldGEiXSxbNiwxMCwiZl8yIiwyXSxbNiwxMiwiZl8xIl0sWzcsOCwiIiwxLHsib2Zmc2V0IjotMSwic3R5bGUiOnsiaGVhZCI6eyJuYW1lIjoibm9uZSJ9fX1dLFs4LDcsIiIsMSx7Im9mZnNldCI6LTEsInN0eWxlIjp7ImhlYWQiOnsibmFtZSI6Im5vbmUifX19XSxbNyw5LCIiLDEseyJvZmZzZXQiOi0xLCJzdHlsZSI6eyJoZWFkIjp7Im5hbWUiOiJub25lIn19fV0sWzksNywiIiwxLHsib2Zmc2V0IjotMSwic3R5bGUiOnsiaGVhZCI6eyJuYW1lIjoibm9uZSJ9fX1dLFs4LDE0LCIiLDAseyJvZmZzZXQiOjF9XSxbNywxNSwiIiwwLHsib2Zmc2V0IjoxfV0sWzEyLDEwLCJmIiwxLHsibGFiZWxfcG9zaXRpb24iOjIwLCJjdXJ2ZSI6LTJ9XSxbNCwxMSwiIiwxLHsib2Zmc2V0IjotMSwiY3VydmUiOjIsInN0eWxlIjp7ImhlYWQiOnsibmFtZSI6Im5vbmUifX19XSxbMTEsNCwiIiwxLHsib2Zmc2V0IjotMSwiY3VydmUiOi0yLCJzdHlsZSI6eyJoZWFkIjp7Im5hbWUiOiJub25lIn19fV1d
		\[\begin{tikzcd}
			1 & {A'} & {E_2} & Q & 1 \\
			1 & A & E & Q & 1 \\
			1 & {A'} & {E_1} & Q & 1
			\arrow[from=2-1, to=2-2]
			\arrow["i", from=2-2, to=2-3]
			\arrow["\pi", from=2-3, to=2-4]
			\arrow[from=1-1, to=1-2]
			\arrow["{i_2}", from=1-2, to=1-3]
			\arrow["{\pi_2}", from=1-3, to=1-4]
			\arrow[from=3-1, to=3-2]
			\arrow["{i_1}", from=3-2, to=3-3]
			\arrow["{\pi_1}", from=3-3, to=3-4]
			\arrow[from=3-4, to=3-5]
			\arrow["\beta"', from=2-2, to=1-2]
			\arrow["\beta", from=2-2, to=3-2]
			\arrow["{f_2}"', from=2-3, to=1-3]
			\arrow["{f_1}", from=2-3, to=3-3]
			\arrow[shift left=1, no head, from=2-4, to=1-4]
			\arrow[shift left=1, no head, from=1-4, to=2-4]
			\arrow[shift left=1, no head, from=2-4, to=3-4]
			\arrow[shift left=1, no head, from=3-4, to=2-4]
			\arrow[shift right=1, from=1-4, to=1-5]
			\arrow[shift right=1, from=2-4, to=2-5]
			\arrow["f"{description, pos=0.2}, curve={height=-12pt}, from=3-3, to=1-3]
			\arrow[shift left=1, curve={height=12pt}, no head, from=1-2, to=3-2]
			\arrow[shift left=1, curve={height=-12pt}, no head, from=3-2, to=1-2]
		\end{tikzcd}\]
		y $E_1$ y $E_2$ son extensiones equivalentes.
	\end{demostracion}
\end{lema}

\begin{teorema}
	Dadas dos extensiones $\extension{i_j} {\pi_j} A {E_j} Q$, para $j=1,2$, podemos expresar la suma de extensiones dada por la estructura aditiva de $H^2(Q,A)$ descrita en la Proposición \ref{extsum} con el siguiente diagrama conmutativo. La última fila se conoce como la suma de Baer de las extensiones $E_1$ y $E_2$.
	
	% https://q.uiver.app/?q=WzAsMTUsWzEsNCwiQSJdLFszLDQsIlxcZnJhY3tFXzFcXHRpbWVzX1EgRV8yfXtLZXIoKyl9Il0sWzUsNCwiUSJdLFszLDAsIkVfMVxcdGltZXMgRV8yIl0sWzEsMiwiQVxcdGltZXMgQSJdLFs1LDIsIlEiXSxbMSwwLCJBXFx0aW1lcyBBIl0sWzUsMCwiUVxcdGltZXMgUSJdLFszLDIsIkVfMVxcdGltZXNfUSBFXzIiXSxbMCw0LCIxIl0sWzAsMiwiMSJdLFswLDAsIjEiXSxbNiw0LCIxIl0sWzYsMiwiMSJdLFs2LDAsIjEiXSxbNiwzLCJpXzFcXHRpbWVzIGlfMiIsMCx7InN0eWxlIjp7InRhaWwiOnsibmFtZSI6Imhvb2siLCJzaWRlIjoidG9wIn19fV0sWzMsNywiXFxwaV8xXFx0aW1lcyBcXHBpXzIiLDAseyJvZmZzZXQiOi0xLCJzdHlsZSI6eyJoZWFkIjp7Im5hbWUiOiJlcGkifX19XSxbNCw4LCJcXHRpbGRlXFxpbWF0aCIsMCx7InN0eWxlIjp7InRhaWwiOnsibmFtZSI6Imhvb2siLCJzaWRlIjoidG9wIn19fV0sWzgsNSwiXFx0aWxkZVxccGkiLDAseyJvZmZzZXQiOi0xLCJzdHlsZSI6eyJoZWFkIjp7Im5hbWUiOiJlcGkifX19XSxbNCwwLCIrIiwyLHsic3R5bGUiOnsiaGVhZCI6eyJuYW1lIjoiZXBpIn19fV0sWzAsMSwiaV8zIiwwLHsic3R5bGUiOnsidGFpbCI6eyJuYW1lIjoiaG9vayIsInNpZGUiOiJ0b3AifX19XSxbMSwyLCJcXHBpXzMiLDAseyJvZmZzZXQiOi0xLCJzdHlsZSI6eyJoZWFkIjp7Im5hbWUiOiJlcGkifX19XSxbNywzLCJzXzFcXHRpbWVzIHNfMiIsMCx7Im9mZnNldCI6LTEsInN0eWxlIjp7InRhaWwiOnsibmFtZSI6Imhvb2siLCJzaWRlIjoidG9wIn19fV0sWzIsMSwic18zIiwwLHsib2Zmc2V0IjotMSwic3R5bGUiOnsidGFpbCI6eyJuYW1lIjoiaG9vayIsInNpZGUiOiJ0b3AifX19XSxbNSw3LCJcXERlbHRhIiwwLHsic3R5bGUiOnsidGFpbCI6eyJuYW1lIjoiaG9vayIsInNpZGUiOiJ0b3AifX19XSxbNSw4LCJcXHRpbGRlIHMiLDAseyJvZmZzZXQiOi0xLCJzdHlsZSI6eyJ0YWlsIjp7Im5hbWUiOiJob29rIiwic2lkZSI6InRvcCJ9fX1dLFs4LDMsImkiLDAseyJzdHlsZSI6eyJ0YWlsIjp7Im5hbWUiOiJob29rIiwic2lkZSI6InRvcCJ9fX1dLFs2LDQsIiIsMSx7Im9mZnNldCI6LTEsInN0eWxlIjp7ImhlYWQiOnsibmFtZSI6Im5vbmUifX19XSxbNCw2LCIiLDEseyJvZmZzZXQiOi0xLCJzdHlsZSI6eyJoZWFkIjp7Im5hbWUiOiJub25lIn19fV0sWzIsNSwiIiwyLHsib2Zmc2V0IjotMSwic3R5bGUiOnsiaGVhZCI6eyJuYW1lIjoibm9uZSJ9fX1dLFs1LDIsIiIsMix7Im9mZnNldCI6LTEsInN0eWxlIjp7ImhlYWQiOnsibmFtZSI6Im5vbmUifX19XSxbOCwxLCJcXHBpX3tLZXIoKyl9IiwyLHsic3R5bGUiOnsiaGVhZCI6eyJuYW1lIjoiZXBpIn19fV0sWzExLDZdLFsxMCw0XSxbOSwwXSxbNywxNF0sWzUsMTNdLFsyLDEyXV0=
		\begin{tikzcd}
			1 & {A\times A} && {E_1\times E_2} && {Q\times Q} & 1 \\
			\\
			1 & {A\times A} && {E_1\times_Q E_2} && Q & 1 \\
			\\
			1 & A && {\frac{E_1\times_Q E_2}{\Ker(+)}} && Q & 1
			\arrow["{i_1\times i_2}", hook, from=1-2, to=1-4]
			\arrow["{\pi_1\times \pi_2}", shift left=1, two heads, from=1-4, to=1-6]
			\arrow["\tilde\imath", hook, from=3-2, to=3-4]
			\arrow["\tilde\pi", shift left=1, two heads, from=3-4, to=3-6]
			\arrow["{+}"', two heads, from=3-2, to=5-2]
			\arrow["{i_3}", hook, from=5-2, to=5-4]
			\arrow["{\pi_3}", shift left=1, two heads, from=5-4, to=5-6]
			\arrow["{s_1\times s_2}", shift left=1, hook, from=1-6, to=1-4]
			\arrow["{s_3}", shift left=1, hook, from=5-6, to=5-4]
			\arrow["\Delta", hook, from=3-6, to=1-6]
			\arrow["{\tilde s}", shift left=1, hook, from=3-6, to=3-4]
			\arrow["i", hook, from=3-4, to=1-4]
			\arrow[shift left=1, no head, from=1-2, to=3-2]
			\arrow[shift left=1, no head, from=3-2, to=1-2]
			\arrow[shift left=1, no head, from=5-6, to=3-6]
			\arrow[shift left=1, no head, from=3-6, to=5-6]
			\arrow["{\pi_{\Ker(+)}}"', two heads, from=3-4, to=5-4]
			\arrow[from=1-1, to=1-2]
			\arrow[from=3-1, to=3-2]
			\arrow[from=5-1, to=5-2]
			\arrow[from=1-6, to=1-7]
			\arrow[from=3-6, to=3-7]
			\arrow[from=5-6, to=5-7]
		\end{tikzcd}
		
		donde $\Delta$ es la inclusión diagonal, $+$ es la proyección a través de la suma de las componentes y $E_1\times_Q E_2$ es el pullback de $E_1$ y $E_2$.
		\begin{gather*}
			+(a_1,a_2) = a_1+a_2 \\
			\Delta(q) = (q,q) \\
			E_1\times_Q E_2 = \{(e_1,e_2)\in E_1\times E_2\ |\ \pi_1(e_1)=\pi_2(e_2)\}
		\end{gather*}
		
		\begin{demostracion}
			A partir de las extensiones $E_1$ y $E_2$ se construye la extensión del producto directo tomando la inclusión y proyección coordenada a coordenada. El objetivo será utilizar los cociclos $c_1$ y $c_2$ asociados a las extensiones $E_1$ y $E_2$ respectivamente para construir una sucesión exacta $\extension {i_3} {\pi_3} A {E_3} Q$ cuyo cociclo asociado sea $c_3=c_1+c_2$.
		\begin{equation}
			1\xrightarrow{} A\times A \xrightarrow{i_1\times i_2} E_1\times E_2\xrightarrow{\pi_1\times\pi_2} Q\times Q \xrightarrow{} 1 % comprobar que es exacta
		\end{equation}
		La sección $s_1\times s_2$ de $\pi_1\times \pi_2$ tiene como cociclo asociado
		\begin{align}
			(c_1\times c_2)\colon (Q\times Q)\times (Q\times Q) &\to A\times A \nonumber\\
			                              ((q_{11},q_{12}),(q_{21},q_{22})) &\mapsto (c_1(q_{11},q_{12}),c_2(q_{21},q_{22})) % comprobar el orden de la operacion
		\end{align}
		Proyectando $A\times A$ sobre $A$ haciendo la suma de componentes movemos $(c_1\times c_2)((q_{11},q_{12}),(q_{21},q_{22}))$ a $c_1(q_{11},q_{12}) + c_2(q_{21},q_{22})$. Ahora identificando $q_{11}$ con $q_{21}$ y $q_{12}$ con $q_{22}$ mediante la inclusión diagonal $\homo \Delta Q {Q\times Q}$, el cociclo asociado a la ultima fila es $c_3=+\circ (c_1\times c_2)\circ \Delta=c_1+c_2$
		
		%en verdad seria \Delta\times\Delta
		\begin{equation}
			Q\times Q \xrightarrow{\Delta} (Q\times Q)\times (Q\times Q) \xrightarrow{c_1\times c_2} A\times A\xrightarrow{+} A
		\end{equation}
		
		A continuación, buscamos el grupo que hace que la extensión de $Q$ por $A$ conmute con el diagrama.
		
		Por el Lema \ref{prop:pullback}, tenemos el siguiente diagrama conmutativo
		% https://q.uiver.app/?q=WzAsMTAsWzEsMCwiQVxcdGltZXMgQSJdLFsyLDAsIkVfMVxcdGltZXMgRV8yIl0sWzMsMCwiUVxcdGltZXMgUSJdLFs0LDAsIjEiXSxbMCwwLCIxIl0sWzAsMSwiMSJdLFsxLDEsIkFcXHRpbWVzIEEiXSxbMywxLCJRIl0sWzQsMSwiMSJdLFsyLDEsIihFXzFcXHRpbWVzIEVfMilcXHRpbWVzX3tRXFx0aW1lcyBRfVEiXSxbNCwwXSxbMCwxLCJpXzFcXHRpbWVzIGlfMiJdLFsxLDIsIlxccGlfMVxcdGltZXMgXFxwaV8yIl0sWzIsM10sWzUsNl0sWzYsOSwiXFx0aWxkZSBpIl0sWzcsOF0sWzAsNiwiIiwxLHsib2Zmc2V0IjotMSwic3R5bGUiOnsiaGVhZCI6eyJuYW1lIjoibm9uZSJ9fX1dLFs2LDAsIiIsMSx7Im9mZnNldCI6LTEsInN0eWxlIjp7ImhlYWQiOnsibmFtZSI6Im5vbmUifX19XSxbNywyLCJcXERlbHRhIl0sWzksMSwiaSJdLFs5LDcsIlxcdGlsZGVcXHBpIl1d
		\[\begin{tikzcd}
			1 & {A\times A} & {E_1\times E_2} & {Q\times Q} & 1 \\
			1 & {A\times A} & {(E_1\times E_2)\times_{Q\times Q}Q} & {Q} & 1
			\arrow[from=1-1, to=1-2]
			\arrow["{i_1\times i_2}", from=1-2, to=1-3]
			\arrow["{\pi_1\times \pi_2}", from=1-3, to=1-4]
			\arrow[from=1-4, to=1-5]
			\arrow[from=2-1, to=2-2]
			\arrow["{\tilde \imath}", from=2-2, to=2-3]
			\arrow[from=2-4, to=2-5]
			\arrow[shift left=1, no head, from=1-2, to=2-2]
			\arrow[shift left=1, no head, from=2-2, to=1-2]
			\arrow["\Delta", from=2-4, to=1-4]
			\arrow["i", from=2-3, to=1-3]
			\arrow["\tilde\pi", from=2-3, to=2-4]
		\end{tikzcd}\]
		
		Observese que el pullback de $\pi_1\times \pi_2$ y $\Delta$ es igual que el pullback de $\pi_1$y $\pi_2$.
		\begin{align*}
			(E_1\times E_2)\times_{Q\times Q}Q 
			&= \{(e_1,e_2,q)\in E_1\times E_2\times Q \ | \ (\pi_1\times \pi_2)(e_1,e_2) = (q,q)\} \\
			&= \{(e_1,e_2)\in E_1\times E_2 \ | \ \pi_1(e_1)=\pi_2(e_2)\} \\
			&= E_1\times_Q E_2
		\end{align*}
		
		Por el Lema \ref{prop:pushout}, tenemos el siguiente diagrama conmutativo
		
		\end{demostracion}
\end{teorema}