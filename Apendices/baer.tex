% !TeX root=../tfg2.tex
\chapter{Suma de Baer y propiedades functoriales de $H^2$}\label{baersum}

En este apéndice se introduce la suma de Baer de extensiones de grupos. Se probará que esta suma de extensiones es equivalente a la vista en la Proposición \ref{extsum}, se darán algunas propiedades functoriales de $H^2$ y relaciones entre ambas. Para este apéndice se usan nociones de teoría de categorías en los que no se entrará en detalle, se recomienda consultar \cite[\S1.1, \S1.3 y \S3.1]{Riehl}.

\begin{definicion}
	Dados dos homomorfismos de grupos $B \xrightarrow{f} A \xleftarrow{g} C$, se define el pullback de $f$ y $g$ como el triple $(P,b,c)$ donde $P$ es un grupo y $\homo b P B$ y $\homo c P C$ son homomorfismos de grupos que hacen al siguiente diagrama conmutativo
	% https://q.uiver.app/?q=WzAsNCxbMCwwLCJQIl0sWzEsMCwiQyJdLFswLDEsIkIiXSxbMSwxLCJBIl0sWzAsMiwiYiIsMl0sWzAsMSwiYyJdLFsxLDMsImYiXSxbMiwzLCJnIiwyXV0=
		\[\begin{tikzcd}
			P & C \\
			B & A
			\arrow["b"', from=1-1, to=2-1]
			\arrow["c", from=1-1, to=1-2]
			\arrow["g", from=1-2, to=2-2]
			\arrow["f"', from=2-1, to=2-2]
		\end{tikzcd}\]
	y verifica la siguiente propiedad universal. Dado otro triple $(D,b',c')$ que hace al cuadrado conmutativo, existe un único homomorfismo $\homo h D P$ que hace al siguiente diagrama conmutativo.
	% https://q.uiver.app/?q=WzAsNSxbMSwxLCJQIl0sWzIsMSwiQyJdLFsxLDIsIkIiXSxbMiwyLCJBIl0sWzAsMCwiRCJdLFswLDIsImIiLDJdLFswLDEsImMiXSxbMSwzLCJmIl0sWzIsMywiZyIsMl0sWzQsMSwiYyciLDAseyJjdXJ2ZSI6LTJ9XSxbNCwyLCJiJyIsMix7ImN1cnZlIjoyfV0sWzQsMCwiXFxleGlzdHMhaCIsMCx7InN0eWxlIjp7ImJvZHkiOnsibmFtZSI6ImRhc2hlZCJ9fX1dXQ==
	\[\begin{tikzcd}
		D \\
		& P & C \\
		& B & A
		\arrow["b"', from=2-2, to=3-2]
		\arrow["c", from=2-2, to=2-3]
		\arrow["g", from=2-3, to=3-3]
		\arrow["f"', from=3-2, to=3-3]
		\arrow["{c'}", curve={height=-12pt}, from=1-1, to=2-3]
		\arrow["{b'}"', curve={height=12pt}, from=1-1, to=3-2]
		\arrow["{\exists!h}", dashed, from=1-1, to=2-2]
	\end{tikzcd}\]
\end{definicion}

\begin{definicion}
	El pushout se define de manera dual como el triple $(P,b,c)$ que completa al diagrama $B \xleftarrow{f} A \xrightarrow{g} C$ y verifica la siguiente propiedad universal. Dado otro triple $(D,b',c')$ que completa al diagrama, existe un único homomorfismo $\homo h P D$ que hace al siguiente diagrama conmutativo
	% https://q.uiver.app/?q=WzAsNSxbMSwxLCJQIl0sWzEsMCwiQyJdLFswLDEsIkIiXSxbMCwwLCJBIl0sWzIsMiwiRCJdLFsyLDAsImIiLDJdLFsxLDAsImMiXSxbMSw0LCJjJyIsMCx7ImN1cnZlIjotMn1dLFsyLDQsImInIiwyLHsiY3VydmUiOjJ9XSxbMCw0LCJcXGV4aXN0cyFoIiwyLHsic3R5bGUiOnsiYm9keSI6eyJuYW1lIjoiZGFzaGVkIn19fV0sWzMsMiwiZiJdLFszLDEsImciLDJdXQ==
	\[\begin{tikzcd}
		A & C \\
		B & P \\
		&& D
		\arrow["b"', from=2-1, to=2-2]
		\arrow["c", from=1-2, to=2-2]
		\arrow["{c'}", curve={height=-12pt}, from=1-2, to=3-3]
		\arrow["{b'}"', curve={height=12pt}, from=2-1, to=3-3]
		\arrow["{\exists!h}"', dashed, from=2-2, to=3-3]
		\arrow["f", from=1-1, to=2-1]
		\arrow["g"', from=1-1, to=1-2]
	\end{tikzcd}\]
\end{definicion}


\begin{lema}\label{prop:pullback}
	Dada una extensión $\extension i \pi A E Q$ y un homomorfismo $\homo \alpha {Q'} Q$, existe una extensión $\extension {} {} A {E_1} {Q'}$ única salvo equivalencia que hace al siguiente diagrama conmutativo.
	% https://q.uiver.app/?q=WzAsMTAsWzEsMCwiQSJdLFsyLDAsIkUiXSxbMywwLCJRIl0sWzQsMCwiMSJdLFswLDAsIjEiXSxbMCwxLCIxIl0sWzEsMSwiQSJdLFszLDEsIlEnIl0sWzQsMSwiMSJdLFsyLDEsIkUnIl0sWzQsMF0sWzAsMSwiaSJdLFsxLDIsIlxccGkiXSxbMiwzXSxbNSw2XSxbNiw5LCJpJyJdLFs5LDcsIlxccGknIl0sWzcsOF0sWzAsNiwiIiwxLHsib2Zmc2V0IjotMSwic3R5bGUiOnsiaGVhZCI6eyJuYW1lIjoibm9uZSJ9fX1dLFs2LDAsIiIsMSx7Im9mZnNldCI6LTEsInN0eWxlIjp7ImhlYWQiOnsibmFtZSI6Im5vbmUifX19XSxbNywyLCJcXGFscGhhIl0sWzksMSwiZiJdXQ==
	\[\begin{tikzcd}
		1 & A & E & Q & 1 \\
		1 & A & {E_1} & {Q'} & 1
		\arrow[from=1-1, to=1-2]
		\arrow["i", from=1-2, to=1-3]
		\arrow["\pi", from=1-3, to=1-4]
		\arrow[from=1-4, to=1-5]
		\arrow[from=2-1, to=2-2]
		\arrow["{i_1}", from=2-2, to=2-3]
		\arrow["{\pi_1}", from=2-3, to=2-4]
		\arrow[from=2-4, to=2-5]
		\arrow[shift left=1, no head, from=1-2, to=2-2]
		\arrow[shift left=1, no head, from=2-2, to=1-2]
		\arrow["\alpha", from=2-4, to=1-4]
		\arrow["{\alpha_*}", from=2-3, to=1-3]
	\end{tikzcd}\]
	\begin{demostracion}
		Tomamos $E_1$ el pullback de $\pi$ y $\alpha$. El homomorfismo $\homo 0 {A} Q'$ completa al diagrama $\extension i \pi A E Q$ y lo hace conmutativo. Por la propiedad universal del pullback, existe un homomorfismo de grupos $\homo {i_1} {A} {E_1}$ y $\extension {i_1} {\pi_1} A {E_1} {Q'}$ es una extensión de grupos. 
		
		Sea $\extension {i_2} {\pi_2} A {E_2} {Q'}$ otra extensión que hace al diagrama conmutativo. De nuevo, por la propiedad universal del pullback, existe $\homo f {E_2} {E_1}$ que hace al siguiente diagrama conmutativo
		% https://q.uiver.app/?q=WzAsMTYsWzQsMSwiMSJdLFswLDIsIjEiXSxbNCwyLCIxIl0sWzAsMSwiMSJdLFsxLDEsIkEiXSxbMywxLCJRIl0sWzEsMiwiQSJdLFsyLDIsIkVfMSJdLFszLDIsIlEnIl0sWzAsMCwiMSJdLFsxLDAsIkEiXSxbMiwwLCJFXzIiXSxbMywwLCJRJyJdLFs0LDAsIjEiXSxbMiwxLCJFIl0sWzcsMl0sWzMsNF0sWzEsNl0sWzYsNywiaV8xIl0sWzcsOCwiXFxwaV8xIl0sWzgsMl0sWzgsNSwiXFxhbHBoYSJdLFsxMiw1LCJcXGFscGhhIiwyXSxbOSwxMF0sWzEwLDExLCJpXzIiXSxbMTEsMTIsIlxccGlfMiJdLFsxMiwxM10sWzEwLDQsIiIsMSx7Im9mZnNldCI6LTEsInN0eWxlIjp7ImhlYWQiOnsibmFtZSI6Im5vbmUifX19XSxbNCwxMCwiIiwxLHsib2Zmc2V0IjotMSwic3R5bGUiOnsiaGVhZCI6eyJuYW1lIjoibm9uZSJ9fX1dLFs0LDYsIiIsMSx7Im9mZnNldCI6LTEsInN0eWxlIjp7ImhlYWQiOnsibmFtZSI6Im5vbmUifX19XSxbNiw0LCIiLDEseyJvZmZzZXQiOi0xLCJzdHlsZSI6eyJoZWFkIjp7Im5hbWUiOiJub25lIn19fV0sWzQsMTQsImkiXSxbMTQsNSwiXFxwaSJdLFs3LDE0LCJmXzEiXSxbMTEsMTQsImZfMiIsMl0sWzUsMF0sWzExLDcsImYiLDEseyJsYWJlbF9wb3NpdGlvbiI6MjAsImN1cnZlIjotMn1dLFsxMiw4LCIiLDEseyJvZmZzZXQiOi0xLCJjdXJ2ZSI6LTIsInN0eWxlIjp7ImhlYWQiOnsibmFtZSI6Im5vbmUifX19XSxbOCwxMiwiIiwxLHsib2Zmc2V0IjotMSwiY3VydmUiOjIsInN0eWxlIjp7ImhlYWQiOnsibmFtZSI6Im5vbmUifX19XV0=
		\[\begin{tikzcd}
			1 & A & {E_2} & {Q'} & 1 \\
			1 & A & E & Q & 1 \\
			1 & A & {E_1} & {Q'} & 1
			\arrow[from=2-1, to=2-2]
			\arrow[from=3-1, to=3-2]
			\arrow["{i_1}", from=3-2, to=3-3]
			\arrow["{\pi_1}", from=3-3, to=3-4]
			\arrow[from=3-4, to=3-5]
			\arrow["\alpha", from=3-4, to=2-4]
			\arrow["\alpha"', from=1-4, to=2-4]
			\arrow[from=1-1, to=1-2]
			\arrow["{i_2}", from=1-2, to=1-3]
			\arrow["{\pi_2}", from=1-3, to=1-4]
			\arrow[from=1-4, to=1-5]
			\arrow[shift left=1, no head, from=1-2, to=2-2]
			\arrow[shift left=1, no head, from=2-2, to=1-2]
			\arrow[shift left=1, no head, from=2-2, to=3-2]
			\arrow[shift left=1, no head, from=3-2, to=2-2]
			\arrow["i", from=2-2, to=2-3]
			\arrow["\pi", from=2-3, to=2-4]
			\arrow["{\alpha_*}", from=3-3, to=2-3]
			\arrow["{\tilde\alpha_*}"', from=1-3, to=2-3]
			\arrow[from=2-4, to=2-5]
			\arrow["f"{description, pos=0.2}, curve={height=-12pt}, from=1-3, to=3-3]
			\arrow[shift left=1, curve={height=-12pt}, no head, from=1-4, to=3-4]
			\arrow[shift left=1, curve={height=12pt}, no head, from=3-4, to=1-4]
		\end{tikzcd}\]
		y $f$ es una equivalencia de extensiones.
	\end{demostracion}
\end{lema}

\begin{lema}\label{prop:pushout}
	Dada una extensión $\extension i \pi A E Q$ y un homomorfismo $\homo \beta {A} {A'}$, existe una extensión $\extension {} {} {A'} {E_1} Q$ única salvo equivalencia que hace al siguiente diagrama conmutativo.
	% https://q.uiver.app/?q=WzAsMTAsWzEsMCwiQSJdLFsyLDAsIkUiXSxbMywwLCJRIl0sWzQsMCwiMSJdLFswLDAsIjEiXSxbMCwxLCIxIl0sWzEsMSwiQSciXSxbMywxLCJRIl0sWzQsMSwiMSJdLFsyLDEsIkUnIl0sWzQsMF0sWzAsMSwiaSJdLFsxLDIsIlxccGkiXSxbMiwzXSxbNSw2XSxbNiw5LCJpJyJdLFs5LDcsIlxccGknIl0sWzcsOF0sWzEsOSwiZiJdLFs3LDIsIiIsMSx7Im9mZnNldCI6LTEsInN0eWxlIjp7ImhlYWQiOnsibmFtZSI6Im5vbmUifX19XSxbMiw3LCIiLDEseyJvZmZzZXQiOi0xLCJzdHlsZSI6eyJoZWFkIjp7Im5hbWUiOiJub25lIn19fV0sWzAsNiwiXFxiZXRhIl1d
	\[\begin{tikzcd}
		1 & A & E & Q & 1 \\
		1 & {A'} & {E_1} & Q & 1
		\arrow[from=1-1, to=1-2]
		\arrow["i", from=1-2, to=1-3]
		\arrow["\pi", from=1-3, to=1-4]
		\arrow[from=1-4, to=1-5]
		\arrow[from=2-1, to=2-2]
		\arrow["{i_1}", from=2-2, to=2-3]
		\arrow["{\pi_1}", from=2-3, to=2-4]
		\arrow[from=2-4, to=2-5]
		\arrow["{\beta_*}", from=1-3, to=2-3]
		\arrow[shift left=1, no head, from=2-4, to=1-4]
		\arrow[shift left=1, no head, from=1-4, to=2-4]
		\arrow["\beta", from=1-2, to=2-2]
	\end{tikzcd}\]
	\begin{demostracion}
		Tomamos $E_1$ el pushout de $i$ y $\beta$. El homomorfismo $\homo 0 {A'} Q$ completa al diagrama $\extension i \pi A E Q$ y lo hace conmutativo. Por la propiedad universal del pushout, existe un homomorfismo $\homo {\pi_1} {E_1} Q$ que hace al diagrama conmutativo y $\extension {i_1} {\pi_1} {A'} {E_1} Q$ es una extensión de grupos.
		 
		Para probar que es única salvo equivalencia, tomamos otra extensión $\extension {i_2} {\pi_2} {A'} {E_2} Q$ y por la propiedad universal del pushout, existe $\homo f {E_1}{E_2}$ tal que el siguiente diagrama conmuta
		% https://q.uiver.app/?q=WzAsMTYsWzAsMCwiMSJdLFswLDEsIjEiXSxbMCwyLCIxIl0sWzQsMiwiMSJdLFsxLDAsIkEnIl0sWzEsMSwiQSJdLFsyLDEsIkUiXSxbMywxLCJRIl0sWzMsMCwiUSJdLFszLDIsIlEiXSxbMiwwLCJFXzIiXSxbMSwyLCJBJyJdLFsyLDIsIkVfMSJdLFszLDNdLFs0LDAsIjEiXSxbNCwxLCIxIl0sWzEsNV0sWzUsNiwiaSJdLFs2LDcsIlxccGkiXSxbMCw0XSxbNCwxMCwiaV8yIl0sWzEwLDgsIlxccGlfMiJdLFsyLDExXSxbMTEsMTIsImlfMSJdLFsxMiw5LCJcXHBpXzEiXSxbOSwzXSxbNSw0LCJcXGJldGEiLDJdLFs1LDExLCJcXGJldGEiXSxbNiwxMCwiZl8yIiwyXSxbNiwxMiwiZl8xIl0sWzcsOCwiIiwxLHsib2Zmc2V0IjotMSwic3R5bGUiOnsiaGVhZCI6eyJuYW1lIjoibm9uZSJ9fX1dLFs4LDcsIiIsMSx7Im9mZnNldCI6LTEsInN0eWxlIjp7ImhlYWQiOnsibmFtZSI6Im5vbmUifX19XSxbNyw5LCIiLDEseyJvZmZzZXQiOi0xLCJzdHlsZSI6eyJoZWFkIjp7Im5hbWUiOiJub25lIn19fV0sWzksNywiIiwxLHsib2Zmc2V0IjotMSwic3R5bGUiOnsiaGVhZCI6eyJuYW1lIjoibm9uZSJ9fX1dLFs4LDE0LCIiLDAseyJvZmZzZXQiOjF9XSxbNywxNSwiIiwwLHsib2Zmc2V0IjoxfV0sWzEyLDEwLCJmIiwxLHsibGFiZWxfcG9zaXRpb24iOjIwLCJjdXJ2ZSI6LTJ9XSxbNCwxMSwiIiwxLHsib2Zmc2V0IjotMSwiY3VydmUiOjIsInN0eWxlIjp7ImhlYWQiOnsibmFtZSI6Im5vbmUifX19XSxbMTEsNCwiIiwxLHsib2Zmc2V0IjotMSwiY3VydmUiOi0yLCJzdHlsZSI6eyJoZWFkIjp7Im5hbWUiOiJub25lIn19fV1d
		\[\begin{tikzcd}
			1 & {A'} & {E_2} & Q & 1 \\
			1 & A & E & Q & 1 \\
			1 & {A'} & {E_1} & Q & 1
			\arrow[from=2-1, to=2-2]
			\arrow["i", from=2-2, to=2-3]
			\arrow["\pi", from=2-3, to=2-4]
			\arrow[from=1-1, to=1-2]
			\arrow["{i_2}", from=1-2, to=1-3]
			\arrow["{\pi_2}", from=1-3, to=1-4]
			\arrow[from=3-1, to=3-2]
			\arrow["{i_1}", from=3-2, to=3-3]
			\arrow["{\pi_1}", from=3-3, to=3-4]
			\arrow[from=3-4, to=3-5]
			\arrow["\beta"', from=2-2, to=1-2]
			\arrow["\beta", from=2-2, to=3-2]
			\arrow["{\tilde\beta_*}"', from=2-3, to=1-3]
			\arrow["{\beta_*}", from=2-3, to=3-3]
			\arrow[shift left=1, no head, from=2-4, to=1-4]
			\arrow[shift left=1, no head, from=1-4, to=2-4]
			\arrow[shift left=1, no head, from=2-4, to=3-4]
			\arrow[shift left=1, no head, from=3-4, to=2-4]
			\arrow[shift right=1, from=1-4, to=1-5]
			\arrow[shift right=1, from=2-4, to=2-5]
			\arrow["f"{description, pos=0.2}, curve={height=-12pt}, from=3-3, to=1-3]
			\arrow[shift left=1, curve={height=12pt}, no head, from=1-2, to=3-2]
			\arrow[shift left=1, curve={height=-12pt}, no head, from=3-2, to=1-2]
		\end{tikzcd}\]
		y $E_1$ y $E_2$ son extensiones equivalentes.
	\end{demostracion}
\end{lema}

\begin{observacion}
	El homomorfismo $\homo \beta A {A'}$ es necesariamente un homomorfismo de $Q$-módulos.
	Dadas $s$ y $s_1$ secciones de $\pi$ y $\pi_1$ respectivamente, y $a\in A$ y $q\in Q$ se tiene $$\beta(q\cdot a) = (i_1^{-1}\circ \beta_*\circ i)(a\cdot q) = (i_1^{-1}\circ \beta_*)(\,^{s(q)}i(a)) = i_1^{-1}(\,^{s_1(q)}i_1(\beta(a))) = q\cdot \beta(a).$$
\end{observacion}

% propiedades functoriales de H^2
\begin{observacion}
	Los homomorfismos $\homo \alpha {Q'} Q$ y $\homo \beta A {A'}$ determinan a partir de la extensión $\extension {}{} A E Q$, una única extensión salvo equivalencia de $Q'$ por $A$ y $Q$ por $A'$ respectivamente. Por tanto, inducen funciones sobre las clases de extensiones equivalentes o, por el Teorema \ref{h2}, sobre los segundos grupos de cohomología
	\begin{align*}
		H^2(\alpha,A) \colon H^2(Q,A) &\to H^2(Q',A) \\
						   [E] &\mapsto [\alpha_*(E)] = [E'], \\
		H^2(Q,\beta) \colon H^2(Q,A) &\to H^2(Q,A') \\
						   [E] &\mapsto [\beta_*(E)] = [E'].
	\end{align*}
	Estas funciones preservan la composición de homomorfismos y el homomorfismo identidad, es decir, dados $\homo {f_1} {Q''} {Q'}$, $\homo {f_2} {Q'} {Q}$, $\homo {g_1} A {A'}$ y $\homo {g_2} {A'} {A''}$, se verifica
	\begin{gather*}
		H^2(f_2\circ f_1,A) = H^2(f_1,A)\circ H^2(f_2,A), \\
		H^2(Q,g_2\circ g_1) = H^2(Q,g_2)\circ H^2(Q,g_1), \\
		H^2(1_Q,A) = 1_{H^2(Q,A)}, \\
		H^2(Q,1_A) = 1_{H^2(Q,A)}.
	\end{gather*}
	y por tanto $H^2(-,A)$ es un functor contravariante de la categoría de grupos a la categoría de grupos abelianos y $H^2(Q,-)$ es un functor covariante de la categoría de $Q$-módulos a la categoría de grupos abelianos. % explicar contra/co-variante
		
\end{observacion}

\begin{teorema}\label{thm:baer}
	Dadas dos extensiones $\extension{i_j} {\pi_j} A {E_j} Q$, para $j=1,2$, podemos expresar la suma de extensiones dada por la estructura aditiva de $H^2(Q,A)$ descrita en la Proposición \ref{extsum} con el siguiente diagrama conmutativo. La última fila se conoce como la suma de Baer de las extensiones $E_1$ y $E_2$.
	
	% https://q.uiver.app/?q=WzAsMTUsWzEsNCwiQSJdLFszLDQsIkVfMStFXzIiXSxbNSw0LCJRIl0sWzMsMCwiRV8xXFx0aW1lcyBFXzIiXSxbMSwyLCJBXFx0aW1lcyBBIl0sWzUsMiwiUSJdLFsxLDAsIkFcXHRpbWVzIEEiXSxbNSwwLCJRXFx0aW1lcyBRIl0sWzMsMiwiRV8xXFx0aW1lc19RIEVfMiJdLFswLDQsIjEiXSxbMCwyLCIxIl0sWzAsMCwiMSJdLFs2LDQsIjEiXSxbNiwyLCIxIl0sWzYsMCwiMSJdLFs2LDMsImlfMVxcdGltZXMgaV8yIiwwLHsic3R5bGUiOnsidGFpbCI6eyJuYW1lIjoiaG9vayIsInNpZGUiOiJ0b3AifX19XSxbMyw3LCJcXHBpXzFcXHRpbWVzIFxccGlfMiIsMCx7Im9mZnNldCI6LTEsInN0eWxlIjp7ImhlYWQiOnsibmFtZSI6ImVwaSJ9fX1dLFs0LDgsIlxcdGlsZGVcXGltYXRoIiwwLHsic3R5bGUiOnsidGFpbCI6eyJuYW1lIjoiaG9vayIsInNpZGUiOiJ0b3AifX19XSxbOCw1LCJcXHRpbGRlXFxwaSIsMCx7InN0eWxlIjp7ImhlYWQiOnsibmFtZSI6ImVwaSJ9fX1dLFs0LDAsIlxcbmFibGEiLDIseyJzdHlsZSI6eyJoZWFkIjp7Im5hbWUiOiJlcGkifX19XSxbMCwxLCJpXzMiLDAseyJzdHlsZSI6eyJ0YWlsIjp7Im5hbWUiOiJob29rIiwic2lkZSI6InRvcCJ9fX1dLFsxLDIsIlxccGlfMyIsMCx7Im9mZnNldCI6LTEsInN0eWxlIjp7ImhlYWQiOnsibmFtZSI6ImVwaSJ9fX1dLFs3LDMsInNfMVxcdGltZXMgc18yIiwwLHsib2Zmc2V0IjotMSwic3R5bGUiOnsidGFpbCI6eyJuYW1lIjoiaG9vayIsInNpZGUiOiJ0b3AifX19XSxbMiwxLCJzXzMiLDAseyJvZmZzZXQiOi0xLCJzdHlsZSI6eyJ0YWlsIjp7Im5hbWUiOiJob29rIiwic2lkZSI6InRvcCJ9fX1dLFs1LDcsIlxcRGVsdGEiLDAseyJzdHlsZSI6eyJ0YWlsIjp7Im5hbWUiOiJob29rIiwic2lkZSI6InRvcCJ9fX1dLFs4LDMsIlxcRGVsdGFfKiIsMCx7InN0eWxlIjp7InRhaWwiOnsibmFtZSI6Imhvb2siLCJzaWRlIjoidG9wIn19fV0sWzYsNCwiIiwxLHsib2Zmc2V0IjotMSwic3R5bGUiOnsiaGVhZCI6eyJuYW1lIjoibm9uZSJ9fX1dLFs0LDYsIiIsMSx7Im9mZnNldCI6LTEsInN0eWxlIjp7ImhlYWQiOnsibmFtZSI6Im5vbmUifX19XSxbMiw1LCIiLDIseyJvZmZzZXQiOi0xLCJzdHlsZSI6eyJoZWFkIjp7Im5hbWUiOiJub25lIn19fV0sWzUsMiwiIiwyLHsib2Zmc2V0IjotMSwic3R5bGUiOnsiaGVhZCI6eyJuYW1lIjoibm9uZSJ9fX1dLFs4LDEsIlxcbmFibGFfKiIsMix7InN0eWxlIjp7ImhlYWQiOnsibmFtZSI6ImVwaSJ9fX1dLFsxMSw2XSxbMTAsNF0sWzksMF0sWzcsMTRdLFs1LDEzXSxbMiwxMl1d
	\begin{tikzcd}
		1 & {A\times A} && {E_1\times E_2} && {Q\times Q} & 1 \\
		\\
		1 & {A\times A} && {E_1\times_Q E_2} && Q & 1 \\
		\\
		1 & A && {E_1+E_2} && Q & 1
		\arrow["{i_1\times i_2}", hook, from=1-2, to=1-4]
		\arrow["{\pi_1\times \pi_2}", shift left=1, two heads, from=1-4, to=1-6]
		\arrow["\tilde\imath", hook, from=3-2, to=3-4]
		\arrow["\tilde\pi", two heads, from=3-4, to=3-6]
		\arrow["\nabla"', two heads, from=3-2, to=5-2]
		\arrow["{i_3}", hook, from=5-2, to=5-4]
		\arrow["{\pi_3}", shift left=1, two heads, from=5-4, to=5-6]
		\arrow["{s_1\times s_2}", shift left=1, hook, from=1-6, to=1-4]
		\arrow["{s_3}", shift left=1, hook, from=5-6, to=5-4]
		\arrow["\Delta", hook, from=3-6, to=1-6]
		\arrow["{\Delta_*}", hook, from=3-4, to=1-4]
		\arrow[shift left=1, no head, from=1-2, to=3-2]
		\arrow[shift left=1, no head, from=3-2, to=1-2]
		\arrow[shift left=1, no head, from=5-6, to=3-6]
		\arrow[shift left=1, no head, from=3-6, to=5-6]
		\arrow["{\nabla_*}"', two heads, from=3-4, to=5-4]
		\arrow[from=1-1, to=1-2]
		\arrow[from=3-1, to=3-2]
		\arrow[from=5-1, to=5-2]
		\arrow[from=1-6, to=1-7]
		\arrow[from=3-6, to=3-7]
		\arrow[from=5-6, to=5-7]
	\end{tikzcd}
		
		donde 
		\begin{gather*}
			\nabla(a_1,a_2) = a_1+a_2, \\
			\Delta(q) = (q,q), \\
			E_1\times_Q E_2 = \{(e_1,e_2)\in E_1\times E_2 \ : \ \pi_1(e_1)=\pi_2(e_2)\}, \\
			E_1+E_2 = E_1\times_Q E_2 /\{(i_1(a),-i_2(a)) \ : \ a\in A\}.
		\end{gather*}
		
		\begin{demostracion}
			A partir de las extensiones $E_1$ y $E_2$ se construye la extensión del producto directo tomando la inclusión y proyección coordenada a coordenada
		\begin{equation}
			1\xrightarrow{} A\times A \xrightarrow{i_1\times i_2} E_1\times E_2\xrightarrow{\pi_1\times\pi_2} Q\times Q \xrightarrow{} 1.
		\end{equation} 
		El objetivo será utilizar los cociclos $c_1$ y $c_2$ asociados a las extensiones $E_1$ y $E_2$ respectivamente para construir una sucesión exacta $\extension {i_3} {\pi_3} A {E_3} Q$ cuyo cociclo asociado sea $c_3=c_1+c_2$.
		
		La sección $s_1\times s_2$ de $\pi_1\times \pi_2$ tiene como cociclo asociado
		\begin{align}
			(c_1\times c_2)\colon (Q\times Q)\times (Q\times Q) &\to A\times A \nonumber\\
			                              ((q_{11},q_{12}),(q_{21},q_{22})) &\mapsto (c_1(q_{11},q_{12}),c_2(q_{21},q_{22})).
		\end{align}
		Proyectando $A\times A$ sobre $A$ haciendo la suma de componentes movemos $(c_1\times c_2)((q_{11},q_{12}),(q_{21},q_{22}))$ a $c_1(q_{11},q_{12}) + c_2(q_{21},q_{22})$. Ahora identificando $q_{11}$ con $q_{21}$ y $q_{12}$ con $q_{22}$ mediante la inclusión diagonal $\homo \Delta Q {Q\times Q}$, el cociclo asociado a la ultima fila es $c_3=\nabla\circ (c_1\times c_2)\circ \Delta=c_1+c_2$
		
		%en verdad seria \Delta\times\Delta
		\begin{equation}
			Q\times Q \xrightarrow{\Delta} (Q\times Q)\times (Q\times Q) \xrightarrow{c_1\times c_2} A\times A\xrightarrow{\nabla} A.
		\end{equation}
		
		A continuación, probamos que los grupos centrales hacen al diagrama conmutativo.
		Para la fila central, definimos las funciones de la siguiente forma
		\begin{gather*}
		\tilde\imath(a_1,a_2) = (i_1(a_1),i_2(a_2)), \\
		\Delta_*(e_1,e_2) = (e_1,e_2), \\
		\tilde\pi(e_1,e_2) = \pi_1(e_1)=\pi_2(e_2).
		\end{gather*}
		
		Se verifica trivialmente que conmuta con la fila superior y que la fila es exacta
		\begin{gather*}
			(\Delta_* \circ \tilde\imath)(a_1,a_2) = (i_1(a_1),i_2(a_2))=(i_1\times i_2)(a_1,a_2),    \\
			(\Delta \circ \tilde\pi)(e_1,e_2) = (\pi_1(e_1),\pi_1(e_1)) = (\pi_1(e_1),\pi_2(e_2)) = ((\pi_1\times\pi_2)\circ \Delta_*)(e_1,e_2), \\
			(\tilde\pi \circ \tilde\imath) (a_1,a_2) = \pi_1(i_1(a_1)) = \pi_2(i_2(a_2))=1.
		\end{gather*}
		
		Para la fila inferior, definimos las funciones
		\begin{gather*}
			i_3(a) = \overline{(i_1(a),1)}= \overline{(1,i_2(a))}, \\
			\nabla_*(e_1,e_2) = \overline{(e_1,e_2)}, \\
			\pi_3(\overline{(e_1,e_2)}) = \pi_1(e_1) = \pi_2(e_2).
		\end{gather*}
		$\pi_3$ está bien definida ya que $\pi_1(e_1) = \pi_1(e_1i_1(a))$ para todo $a\in A$.
		
		De nuevo, estas funciones hacen conmutativas las filas central e inferior y esta última es exacta
		\begin{gather*}
			(i_3\circ \nabla)(a_1,a_2) = \overline{(i_1(a_1+a_2),1)} = \overline{(i_1(a_1),i_2(a_2))}= (\nabla_*\circ \tilde\imath)(a_1,a_2), \\
			(\pi_3\circ \nabla_*)(e_1,e_2) = \overline{(e_1,e_2)} = \pi_1(e_1) = \tilde\pi(e_1,e_2), \\
			(\pi_3\circ i_3) (a) = \pi_3(\overline{(i_1(a),1)})=\pi_1(i_1(a)) = 1.
		\end{gather*}
		
		Por los Lemas \ref{prop:pullback} y \ref{prop:pushout}, estas extensiones son únicas salvo equivalencia y la suma de Baer está bien definida y coincide con la suma de extensiones dada anteriormente.

		\end{demostracion}
\end{teorema}

\begin{corolario}
	Los functores $\alpha_* = H^2(\alpha,A)$ y $\beta_* = H^2(Q,\beta)$ son homomorfismos de grupos, es decir, para todo par de extensiones $\extension {}{} A {E_i} Q$, $i=1,2$ se tiene
	\begin{gather*}
		%\alpha_*(-E) = -\alpha_*(E) \\
		%\beta_*(-E) = -\beta_*(E) \\
		\alpha_*(E_1+E_2) = \alpha_*(E_1)+\alpha_*(E_2),\\
		\beta_*(E_1+E_2) = \beta_*(E_1)+\beta_*(E_2),
	\end{gather*}
	lo que hacen a $H^2$ un functor aditivo.
	\begin{demostracion}
		Lo probamos para $\alpha_*$, para $\beta_*$ se hace de manera análoga usando el lema \ref{prop:pushout}. Basta mirar el siguiente diagrama.
		% https://q.uiver.app/?q=WzAsMzAsWzMsNCwiQSJdLFs1LDIsIkVfMSdcXHRpbWVzX3tRJ31FXzInIl0sWzIsMSwiQVxcdGltZXMgQSJdLFs0LDEsIkVfMVxcdGltZXMgRV8yIl0sWzYsMSwiUVxcdGltZXMgUSJdLFs0LDMsIkVfMVxcdGltZXNfUSBFXzIiXSxbNiwzLCJRIl0sWzIsMywiQVxcdGltZXMgQSJdLFsyLDUsIkEiXSxbNCw1LCJcXGZyYWN7RV8xXFx0aW1lc19RIEVfMn17S2VyKCspfSJdLFs2LDUsIlEiXSxbNywyLCJRJyJdLFs3LDAsIlEnXFx0aW1lcyBRJyJdLFs1LDAsIkVfMSdcXHRpbWVzIEVfMiciXSxbMywwLCJBIFxcdGltZXMgQSJdLFszLDIsIkFcXHRpbWVzIEEiXSxbNSw0LCJcXGZyYWN7RV8xJ1xcdGltZXNfe1EnfSBFXzInfXtLZXIoKyl9Il0sWzcsNCwiUSciXSxbMSwwLCIxIl0sWzAsMSwiMSJdLFsxLDIsIjEiXSxbMCwzLCIxIl0sWzEsNCwiMSJdLFswLDUsIjEiXSxbOCw1LCIxIl0sWzksNCwiMSJdLFs5LDIsIjEiXSxbOCwzLCIxIl0sWzgsMSwiMSJdLFs5LDAsIjEiXSxbNiwxMSwiXFxhbHBoYSIsMV0sWzQsMTIsIlxcYWxwaGFcXHRpbWVzXFxhbHBoYSIsMV0sWzMsMTMsIlxcYWxwaGFfKlxcdGltZXNcXGFscGhhXyoiLDFdLFsyLDE0LCIiLDEseyJvZmZzZXQiOi0xLCJzdHlsZSI6eyJoZWFkIjp7Im5hbWUiOiJub25lIn19fV0sWzE0LDIsIiIsMSx7Im9mZnNldCI6LTEsInN0eWxlIjp7ImhlYWQiOnsibmFtZSI6Im5vbmUifX19XSxbMiw3LCIiLDEseyJvZmZzZXQiOjEsInN0eWxlIjp7ImhlYWQiOnsibmFtZSI6Im5vbmUifX19XSxbNywyLCIiLDEseyJvZmZzZXQiOjEsInN0eWxlIjp7ImhlYWQiOnsibmFtZSI6Im5vbmUifX19XSxbNSwzXSxbNSw5XSxbNyw4LCIrIiwxLHsibGFiZWxfcG9zaXRpb24iOjcwfV0sWzYsMTAsIiIsMSx7Im9mZnNldCI6LTEsInN0eWxlIjp7ImhlYWQiOnsibmFtZSI6Im5vbmUifX19XSxbMTAsNiwiIiwxLHsib2Zmc2V0IjotMSwic3R5bGUiOnsiaGVhZCI6eyJuYW1lIjoibm9uZSJ9fX1dLFsxMSwxMiwiXFxEZWx0YSciLDEseyJsYWJlbF9wb3NpdGlvbiI6MzB9XSxbMSwxM10sWzUsNiwiXFx0aWxkZVxccGkiLDEseyJsYWJlbF9wb3NpdGlvbiI6NzB9XSxbNyw1LCJcXHRpbGRlXFxpbWF0aCIsMSx7ImxhYmVsX3Bvc2l0aW9uIjo3MH1dLFs4LDksImkiLDFdLFs5LDEwLCJcXHBpIiwxXSxbMTMsMTIsIlxccGlfMSdcXHRpbWVzXFxwaV8yJyIsMV0sWzIsMywiaV8xXFx0aW1lcyBpXzIiLDEseyJsYWJlbF9wb3NpdGlvbiI6NzB9XSxbMyw0LCJcXHBpXzFcXHRpbWVzXFxwaV8yIiwxLHsibGFiZWxfcG9zaXRpb24iOjcwfV0sWzE0LDEzLCJpXzEnXFx0aW1lcyBpXzInIiwxXSxbMTQsMTUsIiIsMSx7Im9mZnNldCI6LTEsInN0eWxlIjp7ImhlYWQiOnsibmFtZSI6Im5vbmUifX19XSxbMTUsMTQsIiIsMSx7Im9mZnNldCI6LTEsInN0eWxlIjp7ImhlYWQiOnsibmFtZSI6Im5vbmUifX19XSxbNywxNSwiIiwxLHsib2Zmc2V0IjotMSwic3R5bGUiOnsiaGVhZCI6eyJuYW1lIjoibm9uZSJ9fX1dLFsxNSw3LCIiLDEseyJvZmZzZXQiOi0xLCJzdHlsZSI6eyJoZWFkIjp7Im5hbWUiOiJub25lIn19fV0sWzE1LDEsIlxcdGlsZGVcXGltYXRoJyIsMSx7ImxhYmVsX3Bvc2l0aW9uIjo3MH1dLFsxLDExLCJcXHRpbGRlXFxwaSciLDEseyJsYWJlbF9wb3NpdGlvbiI6NzB9XSxbNiw0LCJcXERlbHRhIiwxLHsibGFiZWxfcG9zaXRpb24iOjMwfV0sWzAsMTYsImknIiwxLHsibGFiZWxfcG9zaXRpb24iOjcwfV0sWzE2LDE3LCJcXHBpJyIsMSx7ImxhYmVsX3Bvc2l0aW9uIjo3MH1dLFsxMSwxNywiIiwxLHsib2Zmc2V0IjotMSwic3R5bGUiOnsiaGVhZCI6eyJuYW1lIjoibm9uZSJ9fX1dLFsxNywxMSwiIiwxLHsib2Zmc2V0IjotMSwic3R5bGUiOnsiaGVhZCI6eyJuYW1lIjoibm9uZSJ9fX1dLFsxLDE2XSxbMTUsMCwiKyIsMSx7ImxhYmVsX3Bvc2l0aW9uIjo3MH1dLFs4LDAsIiIsMSx7Im9mZnNldCI6LTEsInN0eWxlIjp7ImhlYWQiOnsibmFtZSI6Im5vbmUifX19XSxbMCw4LCIiLDEseyJvZmZzZXQiOi0xLCJzdHlsZSI6eyJoZWFkIjp7Im5hbWUiOiJub25lIn19fV0sWzUsMSwiXFxhbHBoYV8qIiwxXSxbOSwxNiwiXFxhbHBoYV8qIiwxXSxbMTAsMTcsIlxcYWxwaGEiLDFdLFsxOCwxNF0sWzE5LDJdLFsyMCwxNV0sWzIxLDddLFsyMiwwXSxbMjMsOF0sWzEwLDI0XSxbMTcsMjVdLFs2LDI3XSxbMTEsMjZdLFs0LDI4XSxbMTIsMjldXQ==
		\[\adjustbox{width=\textwidth}{
		
		% https://q.uiver.app/?q=WzAsMzAsWzMsNCwiQSJdLFs1LDIsIlxcYWxwaGFfKihFXzEpXFx0aW1lc197USd9XFxhbHBoYV8qKEVfMikiXSxbMiwxLCJBXFx0aW1lcyBBIl0sWzQsMSwiRV8xXFx0aW1lcyBFXzIiXSxbNiwxLCJRXFx0aW1lcyBRIl0sWzQsMywiRV8xXFx0aW1lc19RIEVfMiJdLFs2LDMsIlEiXSxbMiwzLCJBXFx0aW1lcyBBIl0sWzIsNSwiQSJdLFs0LDUsIkVfMStFXzIiXSxbNiw1LCJRIl0sWzcsMiwiUSciXSxbNywwLCJRJ1xcdGltZXMgUSciXSxbNSwwLCJcXGFscGhhXyooRV8xKVxcdGltZXMgXFxhbHBoYV8qKEVfMikiXSxbMywwLCJBIFxcdGltZXMgQSJdLFszLDIsIkFcXHRpbWVzIEEiXSxbNSw0LCJcXGFscGhhXyooRV8xKStcXGFscGhhXyooRV8yKSJdLFs3LDQsIlEnIl0sWzksMCwiMSJdLFs4LDEsIjEiXSxbOSwyLCIxIl0sWzgsMywiMSJdLFs5LDQsIjEiXSxbOCw1LCIxIl0sWzAsNSwiMSJdLFsxLDQsIjEiXSxbMCwzLCIxIl0sWzEsMiwiMSJdLFswLDEsIjEiXSxbMSwwLCIxIl0sWzExLDYsIlxcYWxwaGEiLDFdLFsxMiw0LCJcXGFscGhhXFx0aW1lc1xcYWxwaGEiLDFdLFsxMywzLCJcXGFscGhhXypcXHRpbWVzXFxhbHBoYV8qIiwxXSxbMiwxNCwiIiwxLHsib2Zmc2V0IjotMSwic3R5bGUiOnsiaGVhZCI6eyJuYW1lIjoibm9uZSJ9fX1dLFsxNCwyLCIiLDEseyJvZmZzZXQiOi0xLCJzdHlsZSI6eyJoZWFkIjp7Im5hbWUiOiJub25lIn19fV0sWzIsNywiIiwxLHsib2Zmc2V0IjoxLCJzdHlsZSI6eyJoZWFkIjp7Im5hbWUiOiJub25lIn19fV0sWzcsMiwiIiwxLHsib2Zmc2V0IjoxLCJzdHlsZSI6eyJoZWFkIjp7Im5hbWUiOiJub25lIn19fV0sWzUsMywiXFxEZWx0YV8qIiwxLHsibGFiZWxfcG9zaXRpb24iOjMwfV0sWzUsOSwiXFxuYWJsYV8qIiwxLHsibGFiZWxfcG9zaXRpb24iOjcwfV0sWzcsOCwiXFxuYWJsYSIsMSx7ImxhYmVsX3Bvc2l0aW9uIjo3MH1dLFs2LDEwLCIiLDEseyJvZmZzZXQiOi0xLCJzdHlsZSI6eyJoZWFkIjp7Im5hbWUiOiJub25lIn19fV0sWzEwLDYsIiIsMSx7Im9mZnNldCI6LTEsInN0eWxlIjp7ImhlYWQiOnsibmFtZSI6Im5vbmUifX19XSxbMTEsMTIsIlxcRGVsdGEnIiwxLHsibGFiZWxfcG9zaXRpb24iOjMwfV0sWzEsMTMsIlxcRGVsdGEnXyoiLDEseyJsYWJlbF9wb3NpdGlvbiI6MzB9XSxbNSw2LCJcXHRpbGRlXFxwaSIsMSx7ImxhYmVsX3Bvc2l0aW9uIjo3MH1dLFs3LDUsIlxcdGlsZGVcXGltYXRoIiwxLHsibGFiZWxfcG9zaXRpb24iOjcwfV0sWzgsOSwiaSIsMV0sWzksMTAsIlxccGkiLDFdLFsxMywxMiwiXFxwaV8xJ1xcdGltZXNcXHBpXzInIiwxXSxbMiwzLCJpXzFcXHRpbWVzIGlfMiIsMSx7ImxhYmVsX3Bvc2l0aW9uIjo3MH1dLFszLDQsIlxccGlfMVxcdGltZXNcXHBpXzIiLDEseyJsYWJlbF9wb3NpdGlvbiI6NzB9XSxbMTQsMTMsImlfMSdcXHRpbWVzIGlfMiciLDFdLFsxNCwxNSwiIiwxLHsib2Zmc2V0IjotMSwic3R5bGUiOnsiaGVhZCI6eyJuYW1lIjoibm9uZSJ9fX1dLFsxNSwxNCwiIiwxLHsib2Zmc2V0IjotMSwic3R5bGUiOnsiaGVhZCI6eyJuYW1lIjoibm9uZSJ9fX1dLFs3LDE1LCIiLDEseyJvZmZzZXQiOi0xLCJzdHlsZSI6eyJoZWFkIjp7Im5hbWUiOiJub25lIn19fV0sWzE1LDcsIiIsMSx7Im9mZnNldCI6LTEsInN0eWxlIjp7ImhlYWQiOnsibmFtZSI6Im5vbmUifX19XSxbMTUsMSwiXFx0aWxkZVxcaW1hdGgnIiwxLHsibGFiZWxfcG9zaXRpb24iOjcwfV0sWzEsMTEsIlxcdGlsZGVcXHBpJyIsMSx7ImxhYmVsX3Bvc2l0aW9uIjo3MH1dLFs2LDQsIlxcRGVsdGEiLDEseyJsYWJlbF9wb3NpdGlvbiI6MzB9XSxbMCwxNiwiaSciLDEseyJsYWJlbF9wb3NpdGlvbiI6NzB9XSxbMTYsMTcsIlxccGknIiwxLHsibGFiZWxfcG9zaXRpb24iOjcwfV0sWzExLDE3LCIiLDEseyJvZmZzZXQiOi0xLCJzdHlsZSI6eyJoZWFkIjp7Im5hbWUiOiJub25lIn19fV0sWzE3LDExLCIiLDEseyJvZmZzZXQiOi0xLCJzdHlsZSI6eyJoZWFkIjp7Im5hbWUiOiJub25lIn19fV0sWzEsMTYsIlxcbmFibGFfKiIsMSx7ImxhYmVsX3Bvc2l0aW9uIjo3MH1dLFsxNSwwLCJcXG5hYmxhIiwxLHsibGFiZWxfcG9zaXRpb24iOjcwfV0sWzgsMCwiIiwxLHsib2Zmc2V0IjotMSwic3R5bGUiOnsiaGVhZCI6eyJuYW1lIjoibm9uZSJ9fX1dLFswLDgsIiIsMSx7Im9mZnNldCI6LTEsInN0eWxlIjp7ImhlYWQiOnsibmFtZSI6Im5vbmUifX19XSxbMSw1LCJcXGFscGhhXyoiLDFdLFsxNiw5LCJcXGFscGhhXyoiLDFdLFsxNywxMCwiXFxhbHBoYSIsMV0sWzEyLDE4XSxbNCwxOV0sWzExLDIwXSxbNiwyMV0sWzE3LDIyXSxbMTAsMjNdLFsyNCw4XSxbMjUsMF0sWzI2LDddLFsyNywxNV0sWzI4LDJdLFsyOSwxNF1d
		\begin{tikzcd}
			& 1 && {A \times A} && {\alpha_*(E_1)\times \alpha_*(E_2)} && {Q'\times Q'} && 1 \\
			1 && {A\times A} && {E_1\times E_2} && {Q\times Q} && 1 \\
			& 1 && {A\times A} && {\alpha_*(E_1)\times_{Q'}\alpha_*(E_2)} && {Q'} && 1 \\
			1 && {A\times A} && {E_1\times_Q E_2} && Q && 1 \\
			& 1 && A && {\alpha_*(E_1)+\alpha_*(E_2)} && {Q'} && 1 \\
			1 && A && {E_1+E_2} && Q && 1
			\arrow["\alpha"{description}, from=3-8, to=4-7]
			\arrow["\alpha\times\alpha"{description}, from=1-8, to=2-7]
			\arrow["{\alpha_*\times\alpha_*}"{description}, from=1-6, to=2-5]
			\arrow[shift left=1, no head, from=2-3, to=1-4]
			\arrow[shift left=1, no head, from=1-4, to=2-3]
			\arrow[shift right=1, no head, from=2-3, to=4-3]
			\arrow[shift right=1, no head, from=4-3, to=2-3]
			\arrow["{\Delta_*}"{description, pos=0.3}, from=4-5, to=2-5]
			\arrow["{\nabla_*}"{description, pos=0.7}, from=4-5, to=6-5]
			\arrow["\nabla"{description, pos=0.7}, from=4-3, to=6-3]
			\arrow[shift left=1, no head, from=4-7, to=6-7]
			\arrow[shift left=1, no head, from=6-7, to=4-7]
			\arrow["{\Delta'}"{description, pos=0.3}, from=3-8, to=1-8]
			\arrow["{\Delta'_*}"{description, pos=0.3}, from=3-6, to=1-6]
			\arrow["\tilde\pi"{description, pos=0.7}, from=4-5, to=4-7]
			\arrow["\tilde\imath"{description, pos=0.7}, from=4-3, to=4-5]
			\arrow["i"{description}, from=6-3, to=6-5]
			\arrow["\pi"{description}, from=6-5, to=6-7]
			\arrow["{\pi_1'\times\pi_2'}"{description}, from=1-6, to=1-8]
			\arrow["{i_1\times i_2}"{description, pos=0.7}, from=2-3, to=2-5]
			\arrow["{\pi_1\times\pi_2}"{description, pos=0.7}, from=2-5, to=2-7]
			\arrow["{i_1'\times i_2'}"{description}, from=1-4, to=1-6]
			\arrow[shift left=1, no head, from=1-4, to=3-4]
			\arrow[shift left=1, no head, from=3-4, to=1-4]
			\arrow[shift left=1, no head, from=4-3, to=3-4]
			\arrow[shift left=1, no head, from=3-4, to=4-3]
			\arrow["{\tilde\imath'}"{description, pos=0.7}, from=3-4, to=3-6]
			\arrow["{\tilde\pi'}"{description, pos=0.7}, from=3-6, to=3-8]
			\arrow["\Delta"{description, pos=0.3}, from=4-7, to=2-7]
			\arrow["{i'}"{description, pos=0.7}, from=5-4, to=5-6]
			\arrow["{\pi'}"{description, pos=0.7}, from=5-6, to=5-8]
			\arrow[shift left=1, no head, from=3-8, to=5-8]
			\arrow[shift left=1, no head, from=5-8, to=3-8]
			\arrow["{\nabla_*}"{description, pos=0.7}, from=3-6, to=5-6]
			\arrow["\nabla"{description, pos=0.7}, from=3-4, to=5-4]
			\arrow[shift left=1, no head, from=6-3, to=5-4]
			\arrow[shift left=1, no head, from=5-4, to=6-3]
			\arrow["{\alpha_*}"{description}, from=3-6, to=4-5]
			\arrow["{\alpha_*}"{description}, from=5-6, to=6-5]
			\arrow["\alpha"{description}, from=5-8, to=6-7]
			\arrow[from=1-8, to=1-10]
			\arrow[from=2-7, to=2-9]
			\arrow[from=3-8, to=3-10]
			\arrow[from=4-7, to=4-9]
			\arrow[from=5-8, to=5-10]
			\arrow[from=6-7, to=6-9]
			\arrow[from=6-1, to=6-3]
			\arrow[from=5-2, to=5-4]
			\arrow[from=4-1, to=4-3]
			\arrow[from=3-2, to=3-4]
			\arrow[from=2-1, to=2-3]
			\arrow[from=1-2, to=1-4]
		\end{tikzcd}
		}\]
		
		El plano frontal y trasero son las sumas de Baer $E_1+E_2$ y $\alpha_*(E_1)+\alpha_*(E_2)$ respectivamente, y por el Lema \ref{prop:pullback} la fila superior frontal conmuta con la trasera porque está definida componente a componente. Las filas centrales conmutan observando que $\Delta,\Delta',\Delta_*$ y $\Delta_*'$ son inyectivas y definiendo $\alpha_*$ como $\Delta_*^{-1}\circ (\alpha_*\times\alpha_*)\circ \Delta_*'$. Por último, las filas inferiores son cocientes de las filas centrales y definimos $\alpha_*$ como el homomorfismo inducido por el $\alpha_*$ de la fila central.
		
		La flecha $\homo {\alpha_*} {\alpha_*(E_1)+\alpha_*(E_2)} {E_1+E_2}$ dice que $\alpha_*(E_1+E_2) = \alpha_*(E_1)+\alpha_*(E_2)$.
		
	\end{demostracion}
\end{corolario}
