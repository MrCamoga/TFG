% !TeX root=tfg2.tex

\chapter{Introducción}

% extensiones de grupos
El problema de las extensiones de grupos trata de determinar, dados dos grupos $N$ y $Q$, cuáles son todos los grupos $E$ que tienen un subgrupo normal isomorfo a $N$ y cociente $E/N$ isomorfo a $Q$. En este trabajo se estudiará el caso en que $N$ es un grupo abeliano para eventualmente concluir que hay una biyección entre el segundo grupo de cohomología $H^2(Q,N)$ y las clases de extensiones equivalentes de $Q$ por $N$.

% teoremas de hall: cuándo existen subgrupos de cierto orden
Por otro lado, se estudiarán los grupos de acuerdo a la existencia de ciertos subgrupos. Se definirán los $\pi$-subgrupos de Hall como una generalización de los subgrupos de Sylow, cuyo orden es múltiplo de primos del conjunto $\pi$ y cuyo índice no es divisible por ningún primo de $\pi$. Se verá que estos subgrupos comparten muchas semejanzas con los subgrupos de Sylow y se probará para grupos resolubles un resultado análogo al primer y segundo teorema de Sylow, que garantiza la existencia de subgrupos de Hall y que todos ellos son conjugados.

El trabajo está dividido en cuatro capítulos.

% estructura
Se comenzará dando una introducción a los complejos de cocadenas y a la cohomología de grupos. Se probarán diversos resultados que serán necesarios para la clasificación de las extensiones con subgrupo normal abeliano. Una vez caracterizadas a través de la cohomología de grupos, se hará uso del Teorema de Schur-Zassenhaus para demostrar en el Teorema de Hall que todos los $\pi$-subgrupos de Hall son conjugados. Seguidamente, daremos un teorema recíproco que caracteriza a los grupos resolubles mediante la existencia de $\pi$-subgrupos de Hall para todo subconjunto de primos $\pi$. Finalmente introduciremos el homomorfismo del transfer, una herramienta muy útil para el estudio de la resolubilidad de los grupos.