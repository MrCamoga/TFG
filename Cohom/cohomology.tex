% !TeX root=../tfg2.tex

\chapter{Cohomología}

% introduccion explicando que es util para el estudio de extensiones etc
Introduccion 

Comenzamos repasando las definiciones y algunas propiedades de los módulos y de las sucesiones exactas.

\section{G-módulos}

\begin{definicion}
	Sea $G$ un grupo. Un $G$-módulo izquierdo es un grupo abeliano $A$ junto a una acción de grupo a izquierda $\homo \varphi {G\times A} A$, que escribiremos como $\varphi(g,a)=g\cdot a$, compatible con la operación de grupo de $A$. Es decir, para todo $g,g_1,g_2\in G$ y $a,a_1,a_2\in A$ se verifica
	\begin{enumerate}
		\item $1\cdot a=a$
		\item $g_1\cdot(g_2\cdot a)=(g_1g_2)\cdot a$
		\item $g\cdot (a_1+a_2)=g\cdot a_1+g\cdot a_2$
	\end{enumerate} 
	
	Se sigue de (iii) que $g\cdot 0 = 0$ y que $g\cdot (-a) = - g\cdot a$ para todo $g\in G$ y $a\in A$.
\end{definicion}

\begin{definicion}
	Dados dos $G$-módulos $A$ y $B$, un homomorfismo de $G$-módulos o $G$-homomorfismo es una función $\homo f A B$ que verifica
	\begin{enumerate}
		\item $f(a_1+a_2) = f(a_1)+f(a_2)$
		\item $f(g\cdot a) = g\cdot f(a)$
	\end{enumerate}
	
	Denotamos por $\FHom G A B$ al conjunto de $G$-homomorfismos de $A$ a $B$.
\end{definicion}

% demostrar o dejar como ejercicio
%\begin{ejercicio}
%	$\FHom G A B$ tiene estructura de $G$-módulo.
%\end{ejercicio}

%
%\begin{observacion}
%	$\FHom Q A B$ con la operación suma $(f+g)(a) = f(a)+g(a)$ y la acción de $Q$ dada por $(q\cdot f)(a) = q\cdot f(a)$ hacen a $\FHom Q A B$ un $Q$-módulo.
%	\begin{demostracion}
%		La función $f(a)=0$ es la identidad
%	\end{demostracion}
%\end{observacion}

\section{Anillos de grupos}

Dado un grupo $G$ definimos el anillo $\Z[G]$ como el $\Z$-módulo libre generado por los elementos de $G$. Es decir, un elemento en $\Z[G]$ es una suma finita
\begin{equation*}
	\sum_{i=1}^n a_ig_i
\end{equation*}
donde $g_i\in G$ y $a_i\in \Z$.

Definiendo el producto en $\Z[G]$ de la siguiente forma
\begin{equation*}
	\left(\sum_i a_ig_i\right) \left(\sum_jb_jh_j\right) = \sum_{i,j} a_ib_jg_ih_j
\end{equation*}
$\Z[G]$ tiene estructura de anillo con unidad. Notese que este anillo será conmutativo si y solo si $G$ es abeliano.

\begin{observacion}
	Una estructura de $G$-módulo sobre un grupo abeliano $A$ extiende de manera única a un $\Z[G]$-módulo por linealidad. Esto es, dado $n\in\N$, $g\in G$ y $a\in A$ definimos la acción de $ng$ sobre $a$ como $(ng)\cdot a = g\cdot (na)$ y $(-n g)\cdot a = (n g)\cdot (-a)$. Recíprocamente, a partir de una estructura de $\Z[G]$-módulo en $A$, ésta da una acción de $G$ en $A$ restringiendo la acción al subgrupo $G$.
	
	De igual manera, un homomorfismo de $G$-módulos extiende a un homomorfismo de $\Z[G]$-módulos.
	Por ello, a partir de ahora hablaremos indistintamente de $G$ y $\Z[G]$-módulos.
\end{observacion}

\section{Sucesiones exactas}

\begin{definicion}
	Una sucesión de grupos $G_i$ y homomorfismos $f_i$
	\begin{equation*}
		\cdots \xrightarrow{f_{n-2}} G_{n-1} \xrightarrow{f_{n-1}} G_n \xrightarrow{f_{n}} G_{n+1} \xrightarrow{f_{n+1}} \cdots
	\end{equation*}
	se dice que es exacta en $G_n$ si $\Image(f_{n-1}) = \Ker(f_{n})$. Diremos que es la sucesión es exacta si lo es para todo $G_n$.
	
	Diremos que es una sucesión exacta corta cuando es de la forma
	\begin{equation}\label{eq:sec}
		1\xrightarrow{} G_1 \xrightarrow{f_1} G_2\xrightarrow{f_2} G_3 \xrightarrow{} 1
	\end{equation}
\end{definicion}

\begin{observacion}\label{obs:exact}
	La definición de sucesión exacta corta dice que $f_2$ es inyectiva, $f_3$ sobreyectiva y $G_2/\Image(f_2)\cong G_3$.
	 
	Observemos que el recíproco también es cierto. % reciproco de q explicar mejor
	Dados dos grupos $G$ y $H$ y $\homo f G H$ un epimorfismo, se puede construir la sucesión exacta corta $\extension i f {\Ker(f)} G H$.
	De igual manera, si $N \norm G$, entonces $\extension i \pi N G {G/N}$ es una sucesión exacta corta.
\end{observacion}
%	\begin{demostracion}
%		$\Ker(f_2)=\Image(f_1)=\{1\} \implies f_2$ es inyectiva.
%		
%		$\Image(f_3)=\Ker(f_4)=G_3 \implies f_3$ es sobreyectiva.
%		
%		$\Ker(f_3)=\Image(f_2) \cong G_1 \implies {G_2}/{f_2(G_1)}\cong G_3$ por el Primer Teorema de Isomorfía.
%	\end{demostracion}

\begin{proposicion}[Lema corto de los cinco]\label{prop:sflem}
	Si las filas del siguiente diagrama conmutativo son exactas y $g$ y $h$ son isomorfismos entonces $f$ es un isomorfismo.
	% https://q.uiver.app/?q=WzAsMTAsWzEsMCwiQSJdLFsxLDEsIkEnIl0sWzIsMCwiQiJdLFsyLDEsIkInIl0sWzMsMCwiQyJdLFszLDEsIkMnIl0sWzQsMCwiMSJdLFs0LDEsIjEiXSxbMCwwLCIxIl0sWzAsMSwiMSJdLFswLDEsImYiXSxbMCwyLCJpIl0sWzIsNCwiXFxwaSJdLFsyLDMsImciXSxbMSwzLCJpJyIsMl0sWzMsNSwiXFxwaSciLDJdLFs0LDUsImgiXSxbNCw2XSxbOSwxXSxbOCwwXSxbNSw3XV0=
	\[\begin{tikzcd}
		1 & {A_1} & {B_1} & {C_1} & 1 \\
		1 & {A_2} & {B_2} & {C_2} & 1
		\arrow["g", from=1-2, to=2-2]
		\arrow["{i_1}", from=1-2, to=1-3]
		\arrow["{\pi_1}", from=1-3, to=1-4]
		\arrow["f", from=1-3, to=2-3]
		\arrow["{i_2}"', from=2-2, to=2-3]
		\arrow["{\pi_2}"', from=2-3, to=2-4]
		\arrow["h", from=1-4, to=2-4]
		\arrow[from=1-4, to=1-5]
		\arrow[from=2-1, to=2-2]
		\arrow[from=1-1, to=1-2]
		\arrow[from=2-4, to=2-5]
	\end{tikzcd}\]
	\begin{demostracion}
		Para verlo, demostramos que $f$ es inyectiva y sobreyectiva.
		
		Comenzamos viendo que $f$ es inyectiva. Sea $x\in \Ker(f)$, por la conmutatividad del diagrama, $h(\pi_1(x))=\pi_2(f(x)) = \pi_2(1) = 1$. Entonces, $\pi_1(x)\in \Ker(h) = \{1\}$ y $x\in \Ker(\pi_1)$, por tanto existe $\tilde x\in A_1$ tal que $i_1(\tilde x)=x$, de nuevo por conmutatividad, $i_2(g(\tilde x))=f(i_1(\tilde x))=f(x)=1$ y como $i_2$ y $g$ son inyectivas, $\tilde x = 1$ y $x = i_1(\tilde x)=1$.
		
		Ahora vemos la sobreyectividad. Sea $x\in B_2$, como $\pi_1$ y $h$ son sobreyectivas existe $ x_1\in B_1$ tal que $\pi_2(x)=h(\pi_1(x_1)) = \pi_2(f(x_1))$. Por tanto, $x f(x_1)^{-1} \in \Ker(\pi_2)= \Image(i_2)$ y existe $\tilde x\in A_2$ tal que $i_2(\tilde x)=x f(x_1)^{-1}$. Usando que $g$ es isomorfismo, $i_2(\tilde x) = f(i_1(g^{-1}(\tilde x)))$. Despejando $x = f(i_1(g^{-1}(\tilde x)) x_1)$, $f$ es sobreyectiva.
	\end{demostracion}
\end{proposicion}



\section{Complejos de cadenas}

Antes de tratar el caso de la cohomología de grupos, introducimos en general los complejos de cadenas y cocadenas.
	
%Sea $R$ un anillo. Un $R$-módulo graduado es una sucesión $C=\{C_n\}_{n\in \Z}$ de $R$-módulos. 
%Una función de grado $r\in\Z$ entre dos $R$-módulos graduados $C$ y $C'$ es una sucesión de homomorfismos de $R$-módulos $f=\{\homo {f_n} {C_n} {C_{n+r}}\}_{n\in\Z}$.

%Un complejo de cadenas es un par $(C,d)$ donde $C$ es un $R$-módulo graduado y $d$ es un 

Sea $R$ un anillo con unidad. Un $R$-módulo graduado es una sucesión de $R$-módulos $C=\{C_n\}_{n\in\Z}$. Diremos que los elementos del $R$-módulo $C_n$ tienen grado $n$.
Definimos un mapa de grado $r$ entre dos $R$-módulos graduados $C$ y $C'$ como una sucesión de homomorfismos de $R$-módulos $f=\{\homo {f_n} {C_n} {C_{n+r}'}\}_{n\in\Z}$.

Un complejo de cadenas sobre $R$ es el par $(C,d)$ donde $C$ es un $R$-módulo graduado y $\homo d C C$ un mapa de grado $-1$ llamado operador de borde que verifica $d_{n}\circ d_{n+1}=0$ para todo $n\in\Z$, escrito sin subíndices como $d^2 = 0$. Equivalentemente, el operador de borde verifica $\Image(d_{n+1})\subseteq \Ker(d_{n})$.

\begin{equation}
	\cdots  \xrightarrow{} C_{n+1} \xrightarrow{d_{n+1}} C_n \xrightarrow{d_n} C_{n-1} \xrightarrow{} \cdots 
\end{equation}

Definimos los $n$-ciclos y $n$-bordes como $Z_n(C)=\Ker(d_{n})$ y $B_n(C)=\Image(d_{n+1})$ respectivamente. Asimismo, definimos el $n$-ésimo grupo de homología del complejo $C$ como $H_n(C) = \frac{Z_n(C)}{B_n(C)}$. El complejo de cadenas $C$ será exacto si y solo si $H_n(C) = 0$ para todo $n\in\Z$. Así, los grupos de homología miden cómo de lejos está el complejo de ser exacto.

En el caso en que el operador $d$ tenga grado $1$ en vez de $-1$, se dirá que $(C,d)$ es un complejo de cocadenas y $d$ un operador de coborde. Se escribirá todo con superíndices y con el prefijo \textit{co-} para distinguirlo de un complejo de cadenas. De manera análoga definimos los $n$-cociclos, $n$-cobordes y el $n$-ésimo grupo de cohomología como $Z^n(C)=\Ker(d^n)$, $B^n(C)=\Image(d^{n-1})$ y $H^n(C)=\frac{Z^n(C)}{B^n(C)}$.

Notese que no hay ninguna diferencia esencial entre complejos de cadenas y cocadenas y siempre podemos pasar de uno a otro cambiando los índices $C^n = C_{-n}$. En la práctica, para que no sean equivalentes, se suele requerir que los complejos de cadenas verifiquen $C_n = 0$ para todo $n>0$ y los complejos de cocadenas $C^n=0$ para $n<0$.

\section{Cohomología de grupos}

Ahora nos centraremos en la cohomología de grupos. Para comenzar, fijaremos un $Q$-módulo $A$ y definiremos unos grupos abelianos a partir de $Q$ y $A$ sobre los que construiremos un operador coborde. Los resultados principales de esta sección probarán que el mapa define un complejo de cocadenas y que la cohomología de éste es trivial cuando $Q$ y $A$ tienen órdenes coprimos. 
%Estos serán de gran ayuda en el estudio de las extensiones,  %%El estudio de la cohomología de grupos es de gran interés en numerosas areas de las matématicas. P

\begin{definicion}
	Sea $A$ un $G$-módulo con acción $\varphi$ y sea $n\in \N$. Denotamos por $C^n_\varphi(G,A)$ al conjunto de funciones de $G^n$ a $A$. Este conjunto junto con la operación $(f+g)(g_1,\ldots,g_n) = f(g_1,\ldots,g_n) + g(g_1,\ldots,g_n)$ para $f,g\in C^n$ y la acción de $G$ dada por $(g\cdot f)(g_1,\ldots,g_n) = g\cdot f(g_1,\ldots,g_n)$ para $f\in C^n$ y $g\in G$ hacen a $C^n_\varphi(G,A)$ un $G$-módulo. % de hecho es un ZQ-modulo de anillos 
	%(f+g)(q_1,...,q_n) = f(q_1,...,q_n)+g(q_1,...,q_n)
	% (qf)(q_1,...,q_n) = q f(q_1,...,q_n)
	% ((q_1+q_2)f)(q_1,...,q_n) = q_1f(q_1,...,q_n) + q_2f(q_1,...,q_n)
	A los elementos de $C^n_\varphi(G,A)$ los llamaremos $n$-cocadenas. A menudo escribiremos $C^n$ cuando esté claro de qué $G$-módulo estamos hablando. % esto suena cringe

	%$C=\bigoplus_{n=0}^\infty C^n$ para $f\in C^r$ y $g\in C^s$ definimos el producto de $f$ y $g$ como $(fg)(q_1,\ldots,q_{r+s}) = f()$
\end{definicion}


\begin{definicion}
	Dado $n\in \N$, el operador coborde $\homo {\partial^n} {C^n} {C^{n+1}}$ se define como
	\begin{equation*}
		\partial^n f = \sum\limits_{i=0}^{n+1} (-1)^{i} d_i f
	\end{equation*}
	donde el operador $d_i$ viene dado por 
	\[
		(d_if)(g_1,\ldots,g_n) = 
		\begin{cases*} % confuso porque en la formula de arriba f \in C^n y aqui f\in C^{n-1}
			g_1\cdot f(g_2,\ldots,g_n) 				& si $ i=0$ \\
			f(g_1,\ldots,g_ig_{i+1},\ldots,g_n) 	& si $ 0 < i <n$ \\ %f(q_1,\ldots,q_{i-1},q_iq_{i+1},q_{i+2},\ldots,q_{n+1}) & si $ 0 < i <n+1$ \\
			f(g_1,\ldots,g_{n-1}) 					& si $ i=n$
		\end{cases*}
	\]
	Diremos que $\partial f$ es el coborde de $f$. % definir el operador \partial C -> C
\end{definicion}

%Es obvio que el operador coborde es homomorfismo de Q-modulos

\begin{lema}\label{lem:coborde}
	Sea $f\in C^{n-1}$, entonces para todo $i=0,\ldots,n+1$ y $j=i,\ldots,n$
	\begin{equation}
		d_id_jf=d_{j+1}d_if
	\end{equation}
	\begin{demostracion}
		Para simplificar los calculos, extendemos $(g_1,\ldots,g_{n+1})$ a $(g_0,\ldots,g_{n+3})$ con $g_0=g_{n+2}=g_{n+3}=1$.
		Definimos la operación de $d_i$ sobre la tupla como $d_i(g_0,\ldots,g_{n+3}) = (g_0,\ldots,g_ig_{i+1},\ldots,g_{n+3})$. La tupla obtenida tras aplicar $d_i$ y $d_j$, se aplicará a $f$ actuando el primer elemento a la izquierda con la acción de $G$-módulo en $A$ y omitiendo los dos últimos elementos (o actuándolos a la derecha con acción trivial). Así tenemos la siguiente forma de calcular $d_i(d_jf)$
		\begin{align}
			f\cdot (\tilde g_0,\ldots,\tilde g_{n+1}) = \tilde g_0 \cdot f(\tilde g_1,\ldots,\tilde g_{n-1})  \\
			f\cdot (d_j(d_i(g_0,\ldots,g_{n+3}))) = d_i(d_jf)(g_1,\ldots,g_{n+1})	\label{eq:homcohom}
		\end{align}
		De esta forma, reducimos los casos frontera a un mismo caso homogéneo que estudiaremos a continuación.
		
		\begin{itemize}
			\item Caso $j=i$:
				\begin{align*}
					d_i(d_i(g_0,\ldots,g_{n+3})) 
					&= d_i(g_0,\ldots,g_ig_{i+1},\ldots,g_{n+3}) \\
					&= (g_0,\ldots,g_ig_{i+1}g_{i+2},\ldots,g_{n+3}) \\
					&= d_i(g_0,\ldots,g_{i+1}g_{i+2},\ldots,g_{n+3}) \\
					&= d_i(d_{i+1}(g_0,\ldots,g_{n+3}))
				\end{align*}
			\item Caso $j\geq i+1$:
				\begin{align*}
					d_j(d_i(g_0,\ldots,g_{n+3})) 
					&= d_j(g_0,\ldots,g_ig_{i+1},\ldots,g_{n+3}) \\
					&= (g_0,\ldots,g_ig_{i+1},\ldots,g_{j+1}g_{j+2},\ldots,g_{n+3}) \\
					&= d_i(g_0,\ldots,g_{j+1}g_{j+2},\ldots,g_{n+3}) \\
					&= d_i(d_{j+1}(g_0,\ldots,g_{n+3}))
				\end{align*}
		\end{itemize}
		
		Usando \eqref{eq:homcohom}, hemos probado el resultado.
%		\textit{Caso $j=i$}
%		\begin{align*}
%			&(d_i(d_i))(1,1,g_1,\ldots,g_{n+2},1,1) = \\
%			&d_i(\ldots,1,g_1,\ldots,g_ig_{i+1},\ldots,g_n,1,\ldots) = \\
%			&(\ldots,1,g_1,\ldots,g_ig_{i+1}g_{i+2},\ldots,g_n,1,\ldots)
%		\end{align*}
%		\begin{align*}
%			&(d_{i+1}(d_i))(1,1,g_1,\ldots,g_{n+2},1,1) = \\
%			&d_i(\ldots,1,g_1,\ldots,g_{i+1}g_{i+2},\ldots,g_n,1,\ldots) = \\
%			&(\ldots,1,g_1,\ldots,g_ig_{i+1}g_{i+2},\ldots,g_n,1,\ldots)
%		\end{align*}
%		Son iguales
%		
%		\textit{Caso $j=i+1$}
%		\begin{align*}
%			&(d_i(d_{i+1}))(\ldots,1,g_1,\ldots,g_n,1,\ldots) = \\
%			&d_{i+1}(\ldots,1,g_1,\ldots,g_ig_{i+1},\ldots,g_n,1,\ldots) = \\
%			&(\ldots,1,g_1,\ldots,g_ig_{i+1},g_{i+2}g_{i+3},\ldots,g_n,1,\ldots)
%		\end{align*}
%		\begin{align*}
%			&(d_{i+2}(d_i))(\ldots,1,g_1,\ldots,g_n,1,\ldots) = \\
%			&d_i(\ldots,1,g_1,\ldots,g_{i+2}g_{i+3},\ldots,g_n,1,\ldots) = \\
%			&(\ldots,1,g_1,\ldots,g_ig_{i+1},g_{i+2}g_{i+3},\ldots,g_n,1,\ldots)
%		\end{align*}
%		Son iguales
%		
%		\textit{Caso $j>i+1$}
%		\begin{align*}
%			&(d_{j+1}(d_i))(\ldots,1,g_1,\ldots,g_n,1,\ldots) = \\
%			&d_i(\ldots,1,g_1,\ldots,g_{j+1}g_{j+2},\ldots,g_n,1,\ldots) = \\
%			&(\ldots,1,g_1,\ldots,g_ig_{i+1},\ldots,g_{j+1}g_{j+2},\ldots,g_n,1,\ldots)
%		\end{align*}
%		
%		\begin{align*}
%			(\partial^2(\partial^1f))(g_1,g_2,g_3) 
%			& = g_1\cdot (\partial^1f)(g_2,g_3) - (\partial^1f)(g_1g_2,g_3) + (\partial^1f)(g_1,g_2g_3) - (\partial^1f)(g_1,g_2) \\
%			& = g_1g_2\cdot f(g_3) - g_1\cdot f(g_2g_3) + g_1\cdot f(g_2) \\
%			& - g_1g_2\cdot f(g_3) + f(g_1g_2g_3) - f(g_1g_2) \\
%			& + g_1\cdot f(g_2g_3) - f(g_1g_2g_3) + f(g_1) \\
%			& - g_1\cdot f(g_2) + f(g_1g_2) - f(g_1)
%		\end{align*}
	\end{demostracion}
\end{lema}

\begin{teorema}\label{prop:cochaincomplex}
	Sea $n\in \N$, entonces $\partial^{n+1}\partial^n = 0$, lo que hace a
%	\begin{equation*}
%		C^0 \xrightarrow{\partial^0} C^1 \xrightarrow{\partial^1} C^2\xrightarrow{\partial^2} \cdots
%	\end{equation*}
	$(C,\partial)$ un complejo de cocadenas.
	\begin{demostracion}
		
		Para probarlo separaremos la siguiente suma en dos triangulos,  $j<i$ y $j\geq i$, aplicaremos el Lema \ref{lem:coborde} al segundo y finalmente reescribiremos los índices del sumatorio. 
		Sea $f\in C^n$ %y $g_1,\ldots,g_{n+2}\in G$, entonces
		\begin{align*}
		\partial^{n+1}(\partial^{n} f) 
			&= \sum_{i=0}^{n+2} (-1)^i d_i(\partial^n f) = \sum_{i=0}^{n+2}\sum_{j=0}^{n+1} (-1)^{i+j} d_i(d_jf)\\
			&= \sum_{0\leq j < i \leq n+2} (-1)^{i+j} d_i(d_jf)  + \sum_{0\leq i \leq j \leq n+1} (-1)^{i+j} d_i(d_jf)\\
			&= \sum_{0\leq j < i \leq n+2} (-1)^{i+j} d_i(d_jf)  + \sum_{0\leq i \leq j \leq n+1} (-1)^{i+j} d_{j+1}(d_if) \\
			&= \sum_{0\leq j < i \leq n+2} (-1)^{i+j} d_i(d_jf)  + \sum_{0\leq j < i \leq n+2} (-1)^{i+j-1} d_{i}(d_jf) \\
			&= \sum_{0\leq j < i \leq n+2} (-1)^{i+j}(d_i(d_jf)-d_i(d_jf)) = 0
		%(\partial^{n+1}(\partial_{n} f))(g_1,\ldots, g_{n+2}) = \sum\limits_{i=0}^{n+2}\sum\limits_{j=0}^{n+1} (-1)^{i+j} (d_i(d_jf))(g_1,\ldots, g_{n+2})
		\end{align*}
	\end{demostracion}
\end{teorema}

\begin{definicion}
	La proposición anterior nos permite definir los grupos de $n$-cobordes y $n$-cociclos como 
	\begin{align*}
		B^n_\varphi(G,A) &= \Image(\partial^{n-1}) \\
		Z^n_\varphi(G,A) &= \Ker(\partial^{n})
	\end{align*}
	
	Con ellos podemos definir entonces el $n$-ésimo grupo de cohomología como 
	\begin{equation}
		H^n_\varphi(G,A) = \frac{Z^n_\varphi(G,A)}{B^n_\varphi(G,A)} = \frac{\Image(\partial^{n-1})}{\Ker(\partial^{n})}
	\end{equation}
	
	Dado $f\in Z^n(G,A)$, denotaremos su clase de equivalencia por $[f]\in H^n(G,A)$. Diremos que dos cociclos que pertenecen a la misma clase son cohomólogos. 
\end{definicion}

\begin{observacion}
	$C^0(G,A)$ es el conjunto de funciones de $G^0=1$ a $A$ que es naturalmente isomorfo a $A$. El complejo de cocadenas $(C,\partial)$ puede extenderse a la izquierda con un $0$
	\begin{equation*}
		0 \xrightarrow{\partial^{-1}} A \xrightarrow{\partial^0} C^{1}(G,A) \xrightarrow{\partial^1} C^2(G,A) \xrightarrow{\partial^2} \cdots
	\end{equation*}
	lo que permite definir el grupo de cohomología $H^0$ como $\Ker(\partial^0)/\Image(\partial^{-1}) = \{a\in A \ : \ g\cdot a = a \ \forall g\in G \} = A^G$.
\end{observacion}

% H^n(G,A) es un Z_m-modulo

\begin{teorema}\label{thm:trivialH}
	Sea $A$ un $G$-módulo con $\mcd(\ord{G}, \ord{A}) = 1$. Entonces $H^n(G,A)$ es trivial para todo $n\in\Z^+$.
	\begin{demostracion}
		Sea $c\in Z^n(Q,A)$
		\begin{equation}\label{eq:ncocycle}
			\sum\limits_{i=0}^{n+1} (-1)^i(d_ic)(g_1,\ldots,g_{n+1}) = 0
		\end{equation}
		
		Definiendo $\tilde c(g_1,\hdots,g_{n-1}) = {\displaystyle \sum_{g_n\in G} c(g_1,\hdots,g_{n})}$ y haciendo la suma en \eqref{eq:ncocycle} para cada $g_{n+1} \in G$

		\begin{multline*}
			\sum\limits_{g_{n+1}\in G}\sum\limits_{i=0}^{n+1} (-1)^i(d_ic)(g_1,\ldots,g_{n+1}) = \\ 
			= \sum\limits_{i=0}^{n} (-1)^i(d_i\tilde c)(g_1,\ldots,g_n) + (-1)^{n+1}\ord{G} c(g_1,\ldots,g_n) = 0
		\end{multline*}
		
		Como $\mcd(\ord{G}, \ord{A}) = 1$, por la identidad de Bezout existe $k\in \N$ tal que $k \ord{G} \equiv (-1)^{n} \mod \ord{A}$. Multiplicando por $k$
		\begin{equation*}
			c(g_1,\hdots,g_n) = \sum\limits_{i=0}^n (-1)^i(d_i k\tilde c)(g_1,\ldots,g_n)
		\end{equation*}
		
		El cociclo $c$ es el coborde de la función $k \tilde c$ y por tanto $H^n(G,A)$ es trivial.
	\end{demostracion}
\end{teorema}

\section{Cociclos normalizados y $H^2$}

En esta sección daremos un isomorfismo entre el segundo grupo de cohomología y un grupo más pequeño en el que los $2$-cociclos verifican cierta condición de normalización que definiremos a continuación. Este isomorfismo será de gran ayuda para simplificar la clasificación de extensiones en el siguiente capítulo.

%\begin{definicion}% en general un n-cociclo f es normalizado cuando f(q_1,...,q_n) = 0 si cualquiera de los q_i = 0
%	Un $2$-cociclo $c\in Z^2(Q,A)$ se dice que es normalizado cuando $c(1,1)= 0$
%\end{definicion}
%
%% esto tb es cierto para un cociclo no normalizado con accion trivial
%\begin{proposicion}
%	Sea $c\in Z^2(Q,A)$ un $2$-cociclo normalizado. Entonces para todo $q\in Q$
%	\begin{equation*}
%		c(1,q)=0=c(q,1)
%	\end{equation*}
%	\begin{demostracion}
%		Evaluando la condición de un $2$-cociclo en $(1,1,q)\in Q^3$
%		\begin{equation*}
%			1\cdot c(1,q) - c(1,q) + c(1,q) - c(1,1) = c(1,q) =0
%		\end{equation*} 
%		De igual manera, evaluando en $(q,1,1)\in Q^3$
%		\begin{equation*}
%			q\cdot c(1,1) - c(q,1) + c(q,1) - c(q,1) = -c(q,1) = 0
%		\end{equation*}
%	\end{demostracion}
%\end{proposicion}

\begin{definicion}% en general un n-cociclo f es normalizado cuando f(q_1,...,q_n) = 0 si cualquiera de los q_i = 0
	Un $2$-cociclo $c\in Z^2(G,A)$ se dice que es normalizado cuando para todo $g\in G$
	\begin{equation*}
		c(1,g)= 0 = c(g,1)
	\end{equation*}
	La suma de dos cociclos normalizados es también un cociclo normalizado. Definimos los grupos de cociclos y cobordes normalizados como $Z^2_N(G,A)$ y $B^2_N(G,A)$ respectivamente.
\end{definicion}

\begin{proposicion}
	$c\in Z^2_N(G,A)$ si y solo si $c(1,1)=0$.
	\begin{demostracion}
		La implicación a la derecha viene de la definición. Para la implicación no trivial, tomamos $c\in Z^2(G,A)$ tal que $c(1,1) = 0$. 
		Evaluando $\partial^2 c$ en $(1,1,g)\in G^3$
		\begin{equation*}
			1\cdot c(1,g) - c(1,g) + c(1,g) - c(1,1) = c(1,g) =0
		\end{equation*} 
		De igual manera, evaluando en $(g,1,1)\in G^3$
		\begin{equation*}
			g\cdot c(1,1) - c(g,1) + c(g,1) - c(g,1) = -c(g,1) = 0
		\end{equation*}
	\end{demostracion}
\end{proposicion}

\begin{lema}\label{prop:normcoc}
	Todo $2$-cociclo $c\in Z^2(G,A)$ es cohomólogo a un $2$-cociclo normalizado. %Esto es, $\frac{Z^{*n}_\varphi(Q,A)}{B^n_\varphi(Q,A)} = \frac{Z^n_\varphi(Q,A)}{B^n_\varphi(Q,A)}$
	\begin{demostracion}
		Tomamos una función $\homo \phi G A$ tal que $\phi(1) = -c(1,1)$ y construímos el $2$-coborde b de $\phi$
		\begin{equation*}
			b(g_1,g_2) = (\partial^1 \phi)(g_1,g_2) =  q_1\cdot \phi(g_2) - \phi(g_1g_2) + \phi(g_1)
		\end{equation*}
		$b$ verifica que $b(1,1) = \phi(1) = -c(1,1)$ y por tanto el cociclo $\tilde c = c+b$ es un $2$-cociclo normalizado.
	\end{demostracion}
\end{lema}

\begin{teorema}\label{thm:h2hn2}
	Tenemos el siguiente isomorfismo entre los grupos de cohomología
	\begin{equation}
		H^2_N(G,A) = \frac{Z^2_N(G,A)}{B^2_N(G,A)} \cong \frac{Z^2(G,A)}{B^2(G,A)} = H^2(G,A)
	\end{equation}
	\begin{demostracion}
		Por el Lema \ref{prop:normcoc}, tenemos el siguiente epimorfismo
		\begin{align*}
			\Phi \ \colon \ Z^2_N(G,A) &\twoheadrightarrow H^2(G,A) \\
			c \ &\mapsto [c]
		\end{align*}
		
		El kernel de $\Phi$ es precisamente la intersección de $Z^2_N(G,A)$ y $B^2(G,A)$, esto es, $B^2_N(G,A)$. Por el Primer Teorema de Isomorfía, se sigue el resultado.
	\end{demostracion}
\end{teorema}

