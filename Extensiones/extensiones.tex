% !TeX root=../tfg2.tex

% TODO
% G-modulos

\chapter{Extensiones de grupos}

\begin{definicion}
	Una sucesión de grupos $G_i$ y homomorfismos $f_i$
	\begin{equation*}
		G_0\xrightarrow{f_1} G_1 \xrightarrow{f_2} G_2\xrightarrow{f_3} \cdots \xrightarrow{f_n} G_n \xrightarrow{f_{n+1}} \cdots
	\end{equation*}
	se dice que es exacta si $\Ker(f_{i+1})=\Image(f_i)$ para todo $i$. Diremos que es una sucesión exacta corta cuando es de la forma
	\begin{equation}\label{eq:sec}
		1\xrightarrow{} G_1 \xrightarrow{f_2} G_2\xrightarrow{f_3} G_3 \xrightarrow{} 1
	\end{equation}
\end{definicion}

% poner mas bonito o redactar sin separar por lineas
\begin{observacion}
	Una sucesión exacta corta como \eqref{eq:sec} es equivalente a decir que $f_2$ es inyectiva, $f_3$ sobreyectiva y $G_2/f_2(G_1)\cong G_3$
	\begin{demostracion}
		$\Ker(f_2)=\Image(f_1)=\{1\} \implies f_2$ es inyectiva.
		
		$\Image(f_3)=\Ker(f_4)=G_3 \implies f_3$ es sobreyectiva.
		
		$\Ker(f_3)=\Image(f_2) \cong G_1 \implies {G_2}/{f_2(G_1)}\cong G_3$ por el Primer Teorema de Isomorfía.
	\end{demostracion}
\end{observacion}

\begin{definicion}
	Una extensión $E$ de un grupo $Q$ por un grupo $N$ es una sucesión exacta corta
	\begin{equation}\label{eq:ext}
		\extension i \pi N E Q 
	\end{equation}
	Decimos que otra extensión $\extension {i'} {\pi'} N {E'} Q $ es equivalente si existe un homomorfismo $\homo f E {E'}$ tal que el siguiente diagrama conmuta:

%	\begin{figure}[h!] % quitar figure?
%		\centering	
		% https://q.uiver.app/?q=WzAsMTAsWzEsMCwiTiJdLFsyLDAsIkUiXSxbMywwLCJRIl0sWzQsMCwiMSJdLFswLDAsIjEiXSxbMCwxLCIxIl0sWzEsMSwiTiJdLFszLDEsIlEiXSxbNCwxLCIxIl0sWzIsMSwiRSciXSxbNCwwXSxbMCwxLCJpIl0sWzEsMiwiXFxwaSJdLFsyLDNdLFs1LDZdLFs2LDksImknIl0sWzksNywiXFxwaSciXSxbNyw4XSxbMiw3LCIiLDEseyJvZmZzZXQiOi0xLCJzdHlsZSI6eyJoZWFkIjp7Im5hbWUiOiJub25lIn19fV0sWzcsMiwiIiwxLHsib2Zmc2V0IjotMSwic3R5bGUiOnsiaGVhZCI6eyJuYW1lIjoibm9uZSJ9fX1dLFsxLDksImYiXSxbMCw2LCIiLDEseyJvZmZzZXQiOi0xLCJzdHlsZSI6eyJoZWFkIjp7Im5hbWUiOiJub25lIn19fV0sWzYsMCwiIiwxLHsib2Zmc2V0IjotMSwic3R5bGUiOnsiaGVhZCI6eyJuYW1lIjoibm9uZSJ9fX1dXQ==
\begin{equation}\begin{tikzcd}
	1 & N & E & Q & 1 \\
	1 & N & {E'} & Q & 1
	\arrow[from=1-1, to=1-2]
	\arrow["i", from=1-2, to=1-3]
	\arrow["\pi", from=1-3, to=1-4]
	\arrow[from=1-4, to=1-5]
	\arrow[from=2-1, to=2-2]
	\arrow["{i'}", from=2-2, to=2-3]
	\arrow["{\pi'}", from=2-3, to=2-4]
	\arrow[from=2-4, to=2-5]
	\arrow[shift left=1, no head, from=1-4, to=2-4]
	\arrow[shift left=1, no head, from=2-4, to=1-4]
	\arrow["f", from=1-3, to=2-3]
	\arrow[shift left=1, no head, from=1-2, to=2-2]
	\arrow[shift left=1, no head, from=2-2, to=1-2]
\end{tikzcd}
\end{equation}
	%\end{figure}
	% observar que necesariamente f es iso
\end{definicion}

% limpiar
\begin{observacion}\label{prop:eqiso}Dos extensiones equivalentes $\extension {i_{j}} {\pi_j} N {E_j} Q$ para $j=1,2$, son isomorfas.
	\begin{demostracion}
		Para verlo, demostramos que $f$ es inyectiva y sobreyectiva.
		
		\textit{(i) Inyectiva. }Sea $x\in \Ker(f)$, por la conmutatividad del diagrama, $\pi_1(x)=\pi_2(f(x)) = \pi_2(1) = 1$. Entonces, $x\in \Ker(\pi_1) = \Image(i_1)$, por tanto existe $\tilde x\in N$ tal que $i_1(\tilde x)=x$, de nuevo por conmutatividad, $i_2(\tilde x)=f(i_1(\tilde x))=f(x)=1$ y como $i_2$ es inyectiva, $\tilde x = 1$ y $x = i_1(\tilde x)=1$.
		
		\textit{(ii) Sobreyectiva. } Sea $x\in \Image(f)$, como $\pi_1$ es sobreyectiva existe $ x_1\in E_1$ tal que $\pi_2(x)=\pi_1(x_1) = \pi_2(f(x_1))$. Por tanto, $x f(x_1)^{-1} \in \Ker(\pi_2)= \Image(i_2)$ y existe $\tilde x\in N$ tal que $f(i_1(\tilde x))=i_2(\tilde x)=x f(x_1)^{-1}$. Despejando $x = f(i_1(\tilde x) x_1)$, $f$ es sobreyectiva.
	\end{demostracion}
\end{observacion}

% define isomorfismo de extensiones, reeescribir demostracion
\begin{observacion}
	Ser una equivalencia de extensiones es más débil que ser un isomorfismo, como se ve en el siguiente ejemplo.
	% https://q.uiver.app/?q=WzAsMTAsWzAsMCwiMSJdLFsxLDAsIlxcbWF0aGJie1p9XzMiXSxbMiwwLCJcXG1hdGhiYntafV85Il0sWzMsMCwiXFxtYXRoYmJ7Wn1fMyJdLFs0LDAsIjEiXSxbMCwxLCIxIl0sWzEsMSwiXFxtYXRoYmJ7Wn1fMyJdLFsyLDEsIlxcbWF0aGJie1p9XzkiXSxbMywxLCJcXG1hdGhiYntafV8zIl0sWzQsMSwiMSJdLFswLDFdLFsxLDIsIlxcdGltZXMgMyJdLFsyLDMsIlxcdGltZXMgMSJdLFszLDRdLFs1LDZdLFs2LDcsIlxcdGltZXMgMyJdLFs3LDgsIlxcdGltZXMgMiJdLFs4LDldLFsxLDYsIiIsMSx7Im9mZnNldCI6LTEsInN0eWxlIjp7ImhlYWQiOnsibmFtZSI6Im5vbmUifX19XSxbNiwxLCIiLDEseyJvZmZzZXQiOi0xLCJzdHlsZSI6eyJoZWFkIjp7Im5hbWUiOiJub25lIn19fV0sWzIsNywiZiIsMl0sWzMsOCwiIiwxLHsib2Zmc2V0IjotMSwic3R5bGUiOnsiaGVhZCI6eyJuYW1lIjoibm9uZSJ9fX1dLFs4LDMsIiIsMSx7Im9mZnNldCI6LTEsInN0eWxlIjp7ImhlYWQiOnsibmFtZSI6Im5vbmUifX19XV0=
\[\begin{tikzcd}
	1 & {\mathbb{Z}_3} & {\mathbb{Z}_9} & {\mathbb{Z}_3} & 1 \\
	1 & {\mathbb{Z}_3} & {\mathbb{Z}_9} & {\mathbb{Z}_3} & 1
	\arrow[from=1-1, to=1-2]
	\arrow["{\times 3}", from=1-2, to=1-3]
	\arrow["{\times 1}", from=1-3, to=1-4]
	\arrow[from=1-4, to=1-5]
	\arrow[from=2-1, to=2-2]
	\arrow["{\times 3}", from=2-2, to=2-3]
	\arrow["{\times 2}", from=2-3, to=2-4]
	\arrow[from=2-4, to=2-5]
	\arrow[shift left=1, no head, from=1-2, to=2-2]
	\arrow[shift left=1, no head, from=2-2, to=1-2]
	\arrow["f"', from=1-3, to=2-3]
	\arrow[shift left=1, no head, from=1-4, to=2-4]
	\arrow[shift left=1, no head, from=2-4, to=1-4]
\end{tikzcd}\]
	\begin{demostracion}
		Un automorfismo $f$ de $\mathbb{Z}_9$ viene dado por $f(x) = kx$ con $x\in \mathbb{Z}_9$ y $k\in \mathbb{Z}_9^{\times} = \{1,2,4,5,7,8\}$		 
		
		Para que el diagrama conmute a la derecha, $(\times2\circ f)(x) = 2kx = x \mod 3$, $k \equiv 2 \mod 3 \implies k = 2,5,8$
		
		Por otro lado, para que conmute a la izquierda, $(f\circ \times3)(x) = 3kx = 3x \mod 9$, por lo que $k = 1,4,8$.
		
		Por tanto, no existe un isomorfismo $f$ que haga al diagrama conmutativo y las extensiones no son equivalentes.
	\end{demostracion}
\end{observacion}

% definir E_j, \pi_j e i_j?
\begin{proposicion}
	La equivalencia de extensiones es una relación de equivalencia.
	\begin{demostracion}
		\begin{enumerate}
			\item Reflexiva: $E$ es equivalente a sí misma tomando $f=1_E$
			\item Simétrica: Si $\homo f {E_1} {E_2}$ es una equivalencia, por la Observación \eqref{prop:eqiso}, $\homo {f^{-1}} {E_2} {E_1}$ es una equivalencia. 
			\item Transitiva: Si $\homo f {E_1} {E_2}$ y $\homo g {E_2} {E_3}$ son equivalencias, $g\circ f\circ i_1 = g\circ i_2 = i_3$ y $\pi_1 \circ g \circ f = \pi_2 \circ f = \pi_3$, entonces $\homo {g\circ f} {E_1} {E_3}$ es una equivalencia.
		\end{enumerate}
	\end{demostracion}
\end{proposicion}

\begin{definicion}
	Sea $\homo \pi A B$ un homomorfismo de grupos, una sección $s$ de $\pi$ es una inversa a la derecha de $\pi$, esto es, $\homo s B A$ tal que $\pi \circ s = 1_B$.
\end{definicion}

\begin{observacion}\label{extact}
	Una extensión $\extension i \pi N E Q$ determina, por conjugación por elementos de $E$, un homomorfismo $\homo \alpha E {\Aut(N)}$ definido por % 
	\begin{equation*}
		\alpha(g)(n) = n^g = g^{-1}ng
	\end{equation*}
	
	Entonces, $\alpha(N) = \Inn(N)$ y $\alpha$ induce un homomorfismo $\homo {\tilde\alpha} {E/N} {\Out(N)}$ % explicar que esta bien definida
	\begin{equation*}
		\tilde\alpha(gN) = \overline{\alpha(g)} % esto es el abstract kernel de la extension
	\end{equation*}
	
	El homomorfismo $\tilde\alpha$ se conoce como el kernel abstracto de la extensión. % no se usa
	
	Fijando una sección $s$ de $\pi$, para todo $q\in Q$, la conjugación por $s(q)$ determina un automorfismo $\varphi(s(q))$ de $N$ definido por $\varphi(s(q))(n) = \alpha(s(q))(n)$. Notese que la función $\homo \varphi Q {\Aut(N)}$ no es necesariamente un homomorfismo de grupos, pero sí lo es salvo automorfismos internos. En particular, si la sección $s$ es un homomorfismo o el grupo de automorfismos internos de $N$ es trivial, como se estudia en las secciones \ref{sec:split} y \ref{sec:ab}, entonces $\varphi$ sí es un homomorfismo y podremos hablar de la acción de la extensión. %%?????????????????
\end{observacion}

%\begin{observacion}% extensiones equivalentes dan lugar al mismo kernel abstracto
%	Dos extensiones $E$ y $E'$ equivalentes vienen dadas por una misma acción de $Q$ en $N$. Por ello, para estudiar las extensiones salvo equivalencia podemos fijar una acción $\homo \varphi Q {\Aut(N)}$ y estudiar las extensiones que dan lugar a esa acción.
%	\begin{demostracion}
%	 	$f\left(i(n)^{s(q)}\right) = f(i(n))^{f(s(q))} = i'(n)^{s'(q)}$
%	 \end{demostracion}
%\end{observacion}
