% !TeX root=../tfg2.tex

\chapter{Sucio}

%Esto son cosas a sucio para meter en el capítulo de extensiones o en uno previo de cohomología.

\begin{definicion}
	Un $2$-cociclo $\homo c {Q\times Q} A$ se dice que es normalizado cuando $c(1,1)= 0$
\end{definicion}

% esto tb es cierto para un cociclo no normalizado con accion trivial
\begin{proposicion}
	Sea $\homo c {Q\times Q} A$ un $2$-cociclo normalizado. Entonces para todo $q\in Q$
	\begin{equation*}
		c(1,q)=0=c(q,1)
	\end{equation*}
	\begin{demostracion}
		Evaluando la condición de un $2$-cociclo en $(1,1,q)\in Q^3$
		\begin{equation*}
			1\cdot c(1,q) - c(1,q) + c(1,q) - c(1,1) = c(1,q) =0
		\end{equation*} 
		De igual manera, evaluando en $(q,1,1)\in Q^3$
		\begin{equation*}
			q\cdot c(1,1) - c(q,1) + c(q,1) - c(q,1) = -c(q,1) = 0
		\end{equation*}
		
	\end{demostracion}
\end{proposicion}

\begin{proposicion}\label{prop:normcoc}
	Todo $2$-cociclo $\homo c {Q\times Q} A$ es cohomologo a un $2$-cociclo normalizado. %Esto es, $\frac{Z^{*n}_\varphi(Q,A)}{B^n_\varphi(Q,A)} = \frac{Z^n_\varphi(Q,A)}{B^n_\varphi(Q,A)}$
	\begin{demostracion}
		Sea $c(1,1) = x$. Tomamos una función $\homo \phi Q A$ tal que $\phi(1) = -x$ y construímos el $2$-coborde asociado a $\phi$
		\begin{equation*}
			b(q_1,q_2) = q_1\cdot \phi(q_2) - \phi(q_1q_2) + \phi(q_1)
		\end{equation*}
		$b$ verifica que $b(1,1) = \phi(1) = -x$ y por tanto el cociclo $\tilde c = c+b$ es un cociclo normalizado.
	\end{demostracion}
\end{proposicion}



\begin{ejercicio}
	Estudiar las extensiones salvo equivalencia de $\Z_p$ por $\Z_p$. % y ver que $H^2(\Z_p,\Z_p) \cong \Z_p$.
	
		Los grupos de orden $p^2$ son abelianos y por tanto $\Z_p$ es central en la extensión y la acción es trivial.
		Observamos que si $[c]\in H^2(\Z_p,\Z_p)$ $p[c] = [0]$, por lo que $H^2(\Z_p,\Z_p)$ es un $p$-grupo abeliano de exponente $p$, es decir, isomorfo a $\Z_p^k$ para algún $k\in \N$.
		
		Contando el número de extensiones veremos que $0<k<2$ y por tanto, el número de extensiones equivalentes debe ser $p$.
		
		\textit{(Extensiones isomorfas a $\Z_p\times \Z_p$).}
		\begin{equation*}
			\extension i \pi {\Z_p} {\Z_p\times \Z_p} {\Z_p}
		\end{equation*}
		
		La proyección $\homo \pi {\Z_p\times \Z_p} {\Z_p}$ viene dada por las imágenes de los generadores $e_1=(1,0)$ y $e_2=(0,1)$. Como $\pi$ es sobreyectiva, $\pi(e_1)$ o $\pi(e_2)$ es distinto de la identidad, supongamos que $e_1\notin \Ker(\pi)$. Entonces $\langle e_1 \rangle$ es una escisión de $\pi$ y la extensión es trivial, por lo que se corresponde con el elemento neutro de $H^2$ y es única salvo equivalencia.
		
		\textit{(Extensiones isomorfas a $\Z_{p^2}$).}
		\begin{equation*}
			\extension {i_n} {\pi_n} {\Z_p} {\Z_{p^2}} {\Z_p}
		\end{equation*}
		
		$\Z_{p^2}$ es cíclico y $\Z_p$ debe incluirse en el único subgrupo de $\Z_{p^2}$ de orden $p$. 
		
		Hay $p-1$ formas de incluir $\Z_p$ en $\langle p \rangle \leq \Z_{p^2}$ que se corresponden con los distintos automorfismos de $\langle p \rangle$. Las inclusiones $i_n$ vienen dadas por $i_n(x) = pnx \mod p^2$
		
		Las proyecciones $\pi_n$ vienen dadas por la imagen del $1\in \Z_{p^2}$, que puede mandarse a cualquier elemento no trivial de $\Z_p$ y por tanto hay $p-1$ proyecciones distintas definidas por $\pi_n(1) = n \mod p$ para $n=1\ldots p-1$.
		
		En total, hay $(p-1)^2$ formas de componer las $i_n$ y $\pi_n$. Sumando la extensión trivial dan un total de $p^2-2p+2$, que es menor que $p^2$ y mayor que $1$ para todo primo $p$. 
		
		\textit{(Representantes de las extensiones y cociclos asociados).} 
		La primera extensión es única salvo equivalencia y podemos tomar la inclusión en la primera coordenada $i(x) = (x,0)$ y la proyección en la segunda $\pi(x,y) = y$. El cociclo asociado $c_0$ es trivial ya que la extension escinde.
		
		Para la segunda extensión podemos fijar la inclusión $i_1$ para simplificar cálculos. Sean $n,m\in \{1,\ldots,p-1\}$ distintos y tomemos las proyecciones $\pi_m$ y $\pi_n$. Supongamos que existe un homomorfismo $f$ que haga al siguiente diagrama conmutativo
		\[\begin{tikzcd}
			1 & {\Z_p} & {\Z_{p^2}} & {\Z_p} & 1 \\
			1 & {\Z_p} & {\Z_{p^2}} & {\Z_p} & 1
			\arrow[from=1-1, to=1-2]
			\arrow["{i_1}", from=1-2, to=1-3]
			\arrow["{\pi_m}", from=1-3, to=1-4]
			\arrow[from=1-4, to=1-5]
			\arrow[from=2-1, to=2-2]
			\arrow["{i_1}", from=2-2, to=2-3]
			\arrow["{\pi_n}", from=2-3, to=2-4]
			\arrow[from=2-4, to=2-5]
			\arrow[shift left=1, no head, from=1-2, to=2-2]
			\arrow[shift left=1, no head, from=2-2, to=1-2]
			\arrow["f"', from=1-3, to=2-3]
			\arrow[shift left=1, no head, from=1-4, to=2-4]
			\arrow[shift left=1, no head, from=2-4, to=1-4]
		\end{tikzcd}\]

		Por un lado, $f(i_1(1)) = i_1(1) = p$ y por tanto $f(p)=p$. Por otro lado, $\pi_n(f(1)) = \pi_m(1) = m$ y $f(1)\in \pi_n^{-1}(m) = n^{-1}m + \langle p\rangle$. Esto es absurdo ya que $f(p) = pf(1) = pn^{-1}m \neq p$ pero $n\neq m$. Esto prueba que las extensiones $(\Z_{p^2},i_1,\pi_n)$ son inequivalentes para todo $n=1,\ldots,p-1$.
		
		Una sección de $\pi_n$ es $s_n$ definida por $s_n(x) = n^{-1}x \mod p = \overline{n^{-1}x}$. En efecto, $\pi_n(s_n(x)) = \pi_n(\overline{n^{-1}x}) = nn^{-1}x = x$ para $x=0,\ldots,p-1$.
		El cociclo asociado a esta sección es 
		\begin{equation*}
			c_n(x,y) = i_1^{-1}(s_n(x)+s_n(y)-s_n(x+y)) = \frac{\overline{n^{-1}x}+\overline{n^{-1}y}-\overline{n^{-1}(x+y)}}{p} %= i_1^{-1}\left(\overline{n^{-1}x}+\overline{n^{-1}y}-\overline{n^{-1}(x+y)}\right) 
		\end{equation*}
		
		Como $H^2$ es cíclico, podemos tomar $c_1$, cuya expresión es sencilla, y generar el resto de los cociclos con él
		
		\[
		    c_1(x,y) = \frac{\overline{x} + \overline{y} - \overline{x+y}}{p} =  \begin{cases}
		        0 & \text{si } x+y < p\\
		        1 & \text{si } x+y \geq p
		        \end{cases}
		\]
		% comprobar si se corresponden con los pi_n o estan permutados por un n^-1 o algo asi
		\[
		    c_n(x,y) = nc_1(x,y) =  \begin{cases}
		        0 & \text{si } x+y < p\\
		        n & \text{si } x+y \geq p
		        \end{cases}
		\]

\end{ejercicio}


%\begin{proposicion}
%	\begin{demostracion}
%		A partir de las extensiones $E_1$ y $E_2$ se construye la extensión del producto directo tomando la inclusión y proyección coordenada a coordenada. El objetivo será utilizar los cociclos $c_1$ y $c_2$ para construir una sucesión exacta $1\xrightarrow{} A\xrightarrow{i_3} E_3\xrightarrow{\pi_3} Q \xrightarrow{} 1$ cuyo cociclo asociado sea $c_3=c_1+c_2$.
%		\begin{equation}
%			1\xrightarrow{} A\times A \xrightarrow{i_1\times i_2} E_1\times E_2\xrightarrow{\pi_1\times\pi_2} Q\times Q \xrightarrow{} 1 % comprobar que es exacta
%		\end{equation}
%		La sección $s_1\times s_2$ de $\pi_1\times \pi_2$ tiene como cociclo asociado
%		\begin{align}
%			(c_1\times c_2)\colon (Q\times Q)\times (Q\times Q) &\to A\times A \\
%			                              ((q_{11},q_{12}),(q_{21},q_{22})) &\mapsto (c_1(q_{11},q_{12}),c_2(q_{21},q_{22})) % comprobar el orden de la operacion
%		\end{align}
%		Proyectando $A\times A$ sobre $A$ y haciendo la suma de componentes en $A$ movemos $(c_1\times c_2)((q_{11},q_{12}),(q_{21},q_{22}))$ a $c_1(q_{11},q_{12}) + c_2(q_{21},q_{22})$. Basta identificar $q_{11}$ con $q_{21}$ y $q_{12}$ con $q_{22}$ mediante $\homo \Delta Q {Q\times Q}$ definido por $\Delta(q)=(q,q)$. Notese que el morfismo diagonal está definido para las secciones $s_1$ y $s_2$, por tanto, para los cociclos estará definido de $Q\times Q$ en $(Q\times Q)\times (Q\times Q)$.
%		
%		\begin{equation}
%			Q\times Q \xrightarrow{\Delta} (Q\times Q)\times (Q\times Q) \xrightarrow{c_1\times c_2} A\times A\xrightarrow{+} A
%		\end{equation}
%		El cociclo que buscamos es $c_3=+\circ (c_1\times c_2)\circ \Delta=c_1+c_2$
%		
%		Completando el diagrama de $\Delta$ y $\pi_1 \times \pi_2$ con $\homo i {\tilde E} {E_1\times E_2}$ y $\homo {\tilde\pi} E Q$, para $x\in \tilde E$ $(\pi_1(i(x)),\pi_2(i(x)))=(\tilde\pi(x),\tilde\pi(x))$ lo que implica que $\pi_1(i(x))=\pi_2(i(x))$ y por tanto $i(x)\in E_1\times_Q E_2$. Lo natural es tomar $\tilde E = E_1\times_Q E_2$, $i$ la inclusión y $\tilde\pi(e_1,e_2)=\pi_1(e_1)=\pi_2(e_2)$.
%		
%		% https://q.uiver.app/?q=WzAsNCxbMCwwLCJFXzFcXHRpbWVzIEVfMiJdLFsyLDAsIlFcXHRpbWVzIFEiXSxbMiwxLCJRIl0sWzAsMSwiXFx0aWxkZSBFIl0sWzMsMCwiaSJdLFsxLDAsInNfMVxcdGltZXMgc18yIl0sWzIsMSwiXFxEZWx0YSJdLFsyLDMsIlxcdGlsZGUgcyJdXQ==
%		\[\begin{tikzcd}
%			{E_1\times E_2} && {Q\times Q} \\
%			{\tilde E} && Q
%			\arrow["i", from=2-1, to=1-1]
%			\arrow["{\pi_1\times \pi_2}", from=1-1, to=1-3]
%			\arrow["\Delta", from=2-3, to=1-3]
%			\arrow["{\tilde \pi}", from=2-1, to=2-3]
%		\end{tikzcd}\]
%		
%		$\Ker(\tilde \pi) = \{(e_1,e_2)\in E_1\times_Q E_2 \colon \tilde\pi(e_1,e_2)=0\}=\{(e_1,e_2)\in E_1\times_Q E_2 \colon \pi_1(e_1)=\pi_2(e_2)=0\} = A\times A$
%	
%		Por tanto, la sucesión $1\xrightarrow{} A\times A \xrightarrow{\tilde\imath} E_1\times_Q E_2\xrightarrow{\tilde\pi} Q \xrightarrow{} 1$ es exacta.
%		
%		$i_3(a_1+a_2)=\pi(a_1,a_2)=\pi(a_1+a_2,0)=\pi(0,a_1+a_2)$, para que esté bien definida, tenemos que cocientar $E_1\times_Q E_2$ por $Ker(+)=\{(a_1,a_2)\in A\times A \colon a_1+a_2=0\}=\{(a,-a) \colon a\in A\}$
%		
%		
%		% Fin
%		El cociclo correspondiente a la extensión es por tanto $c_1+c_2$ y por la proposición \ref{extsum} la suma está bien definida en clases de extensiones equivalentes.
%	\end{demostracion}
%\end{proposicion}
