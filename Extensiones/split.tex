% !TeX root=../tfg2.tex

\section{Extensiones separables}\label{sec:split}

\begin{definicion} % mover antes de explicar 
	Sea $\homo \pi A B$ un homomorfismo de grupos, una sección $s$ de $\pi$ es una inversa a la derecha de $\pi$, esto es, $\homo s B A$ tal que $\pi \circ s = 1_B$.
\end{definicion}

\begin{definicion}	
	Decimos que una extensión $\extension i \pi N E Q$ es separable cuando existe una sección $\homo s Q E$ de $\pi$ que es un homomorfismo.
\end{definicion}

%G tiene un subgrupo isomorfo a Q, interseca trivialmente con N por ser una sucesion exacta y por tanto es un producto semidirecto...
\begin{teorema}\label{splitext}
	Una extensión $\extension i \pi N E Q$ es separable sí y solo sí E es equivalente a $\extension {} {} N {Q\ltimes_{\varphi} N} Q$, donde $\homo \varphi Q \Aut(N)$ es una acción de grupos de $Q$ en $N$. % explicar que es una accion y no una funcion
	\begin{demostracion}
		% =>
		Si $E$ es separable, existe una sección $\homo s Q E$ que es homomorfismo. $s$ es inyectiva porque es una inversa por la derecha de $\pi$ y, por tanto, $Q\cong s(Q) \leq E$.
		$\pi (i(N)) = \{1\}$ y $s(q)\in i(N) \iff q = 1$, por lo que $s(Q)\cap i(N) = {1_E}$ y $E$ es isomorfo a un producto semidirecto externo de $Q$ por $N$.
		
		% <=
		En la otra dirección, definiendo para $n\in N, \ q\in Q$ $i(n) = (1_Q,n)$ y $\pi((n,q)) = qN$ % TODO
	\end{demostracion}
\end{teorema}

\begin{teorema}
	Una extension $\extension i \pi N E Q$ separable es equivalente
\end{teorema}
% Splittings (Escisiones)
\subsection{Clasificación de las escisiones}

Para clasificar las escisiones consideraremos la extensión separable canónica 

\begin{equation}\label{extsplit}
	\extension i \pi A {Q \ltimes A} Q
\end{equation}

% escisiones vienen dadas por 1-cociclos
\begin{proposicion}
	Las escisiones de \eqref{extsplit} son homomorfismos de la forma $s(q) = (q,c(q))$ donde $\homo c Q A$ es un $1$-cociclo.
	\begin{proof}
		Una sección $s$ de $\pi$ tiene la forma $s(q) = (q,c(q))$ donde $c$ es una función $\homo c Q A$. Imponiendo que la sección sea un homomorfismo
		\begin{align*}
			s(q_1)s(q_2) &= (q_1q_2,c(q_1)\cdot q_2 + c(q_2)) \\
			s(q_1q_2) &= (q_1q_2,c(q_1q_2))
		\end{align*}
		la función $c$ tiene que verificar la ecuación de un $1$-cociclo para que $s$ sea una escisión.
		\begin{equation}
			c(q_1q_2) = c(q_1)\cdot q_2 + c(q_2)
		\end{equation}
	\end{proof}
\end{proposicion}

\begin{definicion}
	Dos escisiones $s_1$ y $s_2$ se dice que son $A$-conjugadas si existe un $a\in A$ tal que $s_1(q)=s_2(q)^{i(a)}$ para todo $q\in Q$
\end{definicion}

\begin{proposicion}
	Sean $s_1$ y $s_2$ dos escisiones y $c_1$ y $c_2$ los $1$-cociclos asociados. Entonces $s_1$ y $s_2$ son $A$-conjugadas si $c_1$ y $c_2$ se diferencian en un $1$-coborde.
	\begin{demostracion}
		Si existe un $a \in A$ tal que $s_1(q) = s_2(q)^{i(a)}$ para todo $q\in Q$, 
		\begin{align*}
		(q,c_1(q)) &= (1,-a)(q,c_2(q))(1,a) \\
					&= (q,-a \cdot q +  c_2(q) + a)
		\end{align*}
		
		La diferencia entre $c_1$ y $c_2$ verifica la ecuación de un $1$-coborde
		
		\begin{equation}
			c_2(q) - c_1(q) = a\cdot q - a
		\end{equation}
	\end{demostracion}
\end{proposicion}

%\begin{observacion}
%	Las definiciones de $1$-cociclos y $1$-cobordes dadas se corresponden con las definiciones usuales en el caso en que $N$ es abeliano.
%	\begin{align}
%		c(q_1q_2) = c(q_1)^{q_2}c(q_2) = c(q_1)\cdot q_2 + c(q_2) \\
%		c_1(q) = n^{-q}c_2(q)n \iff c_2(q)-c_1(q) = n\cdot q - n
%	\end{align}
%\end{observacion}

% explicar que se corresponde con el grupo de cohomologia en el caso abeliano
\begin{teorema}\label{h1}
	Sea $E$ una extensión separable de $Q$ por $A$. Entonces, las clases de escisiones $A$-conjugadas están en correspondencia uno a uno con los elementos de $H^1(Q,A)$
\end{teorema}

% se comprueba trivialmente que es una relacion de equivalencia