\documentclass[11pt,a4paper,twoside,spanish]{book}

\usepackage[utf8]{inputenc}
\usepackage{pkgs/TFG}

\usepackage{mathtools}
\usepackage{hyperref}
\usepackage{tikz,pgfplots,pkgs/quiver}
\usepackage{adjustbox}
\usetikzlibrary{positioning, babel}
\pgfplotsset{compat=1.18}

\usepackage{verbatim}


%\newcommand{\Z}{\mathbb{Z}}
%\newcommand{\N}{\mathbb{N}}


% dark theme
\usepackage{xcolor}
%\pagecolor[rgb]{0,0,0}
%\color[rgb]{1,1,1}



\DeclareMathOperator{\SylowSubgroup}{Syl}
\DeclareMathOperator{\HallSubgroup}{Hall}
\DeclareMathOperator{\OSubgroup}{O}
\DeclareMathOperator{\Normalizer}{N}
\DeclareMathOperator{\Center}{Z}
\DeclareMathOperator{\mcd}{mcd}
\DeclareMathOperator{\mcm}{mcm}
%\DeclareMathOperator{\Ker}{Ker}
\DeclareMathOperator{\Image}{Im}
%\DeclareMathOperator{\Hom}{Hom}
\DeclareMathOperator{\Aut}{Aut}
\DeclareMathOperator{\Inn}{Inn}
\DeclareMathOperator{\Out}{Out}
\DeclareMathOperator{\Ext}{Ext}
\DeclareMathOperator{\SL}{SL}
\DeclareMathOperator{\GL}{GL}
\DeclareMathOperator{\PGL}{PGL}
\DeclareMathOperator{\PSL}{PSL}
\DeclareMathOperator{\car}{car}
\DeclareMathOperator{\sgn}{sgn}
%\DeclareMathOperator{\det}{det}
%\DeclareMathOperator{\HomFunctor}{Hom}
\DeclareMathOperator{\ExtFunctor}{Ext}
\DeclareMathOperator{\TorFunctor}{Tor}
\DeclareMathOperator{\AbFunctor}{Ab}
\DeclareMathOperator{\Pletra}{P}

\newcommand{\FHom}[3]{\Hom_#1(#2,#3)}
\newcommand{\FTor}[3]{\TorFunctor_#1(#2,#3)}
\newcommand{\FExt}[3]{\Ext_#1(#2,#3)}
\newcommand{\Syl}[2]{\SylowSubgroup_#1(#2)}
\newcommand{\Hall}[2]{\HallSubgroup_#1(#2)}
\newcommand{\Core}[2]{\OSubgroup_#1(#2)}
\newcommand{\Ab}[1]{#1^\text{ab}}
\newcommand{\gensub}[1]{\left\langle#1\right\rangle}
\newcommand{\Norm}[2]{\Normalizer_#1(#2)}
\newcommand{\norm}{\trianglelefteq}
\newcommand{\ord}[1]{\left|#1\right|}%\vert
\newcommand{\homo}[3]{#1\colon #2\to #3}
\newcommand{\extension}[5]{1\xrightarrow{} #3 \xrightarrow{#1} #4\xrightarrow{#2} #5 \xrightarrow{} 1}
\newcommand{\transversal}[2]{\{#1_1,\ldots,#1_#2\}}
\newcommand{\transfer}[2]{\tau_{#1/ #2}}
\newcommand{\pretransfer}[2]{P_{#1/ #2}}
\newcommand{\ptransfer}[2]{\tilde\tau_{#1/ #2}}


\begin{document}
\renewcommand{\tablename}{Tabla}
\frontmatter

\titulo{Extensiones de grupos y teoremas de Hall}
\nombre{Carlos Moya García}
\director{Jon González Sánchez}
\otrodirector{}
\fecha{22/06/2022}

\maketitle
\thispagestyle{empty}
\pagestyle{plain}
\tableofcontents
\clearpage{\pagestyle{empty}\cleardoublepage}
\pagestyle{fancy}
%\fancyhf{}
\renewcommand{\chaptermark}[1]{\markboth{#1}{}}
\fancyhead[LO]{\slshape\nouppercase{\leftmark}}
\fancyhead[RO]{\thepage}
\fancyhead[LE]{\thepage}
\fancyhead[RE]{\slshape\nouppercase{\rightmark}}

%\incluir{Introduccion}

\mainmatter
\renewcommand{\chaptermark}[1]{\markboth{\chaptername\ \thechapter. #1}{}}

% !TeX root=../tfg2.tex

\chapter{Cohomología}

% introduccion explicando que es util para el estudio de extensiones etc
Introduccion 

%\section{No sé dónde meter la definicion de sucesiones exactas}% preeliminares

\section{G-módulos}

\begin{definicion}
	Sea $G$ un grupo. Un $G$-módulo izquierdo es un grupo abeliano $A$ junto a una acción de grupo a izquierda $\homo \varphi {G\times A} A$, que escribiremos como $\varphi(g,a)=g\cdot a$, compatible con la operación de grupo de $A$. Es decir, para todo $g,g_1,g_2\in G$ y $a,a_1,a_2\in A$ se verifica
	\begin{enumerate}
		\item $1\cdot a=a$
		\item $g_1\cdot(g_2\cdot a)=(g_1g_2)\cdot a$
		\item $g\cdot (a_1+a_2)=g\cdot a_1+g\cdot a_2$
	\end{enumerate} 
	
	% esto igual aqui no pinta mucho pero bueno nose
	Se sigue de (iii) que $g\cdot 0 = 0$ y que $q\cdot (-a) = - q\cdot a$ para todo $g\in G$ y $a\in A$.
\end{definicion}

\begin{definicion}
	Dados dos $G$-módulos $A$ y $B$, un homomorfismo de $G$-módulos o $G$-homomorfismo es una función $\homo f A B$ que verifica
	\begin{enumerate}
		\item $f(a_1+a_2) = f(a_1)+f(a_2)$
		\item $f(g\cdot a) = g\cdot f(a)$
	\end{enumerate}
	
	Denotamos por $\FHom G A B$ al conjunto de $G$-homomorfismos de $A$ a $B$.
\end{definicion}

% demostrar o dejar como ejercicio
\begin{ejercicio}
	$\FHom G A B$ tiene estructura de $G$-módulo.
\end{ejercicio}

%
%\begin{observacion}
%	$\FHom Q A B$ con la operación suma $(f+g)(a) = f(a)+g(a)$ y la acción de $Q$ dada por $(q\cdot f)(a) = q\cdot f(a)$ hacen a $\FHom Q A B$ un $Q$-módulo.
%	\begin{demostracion}
%		La función $f(a)=0$ es la identidad
%	\end{demostracion}
%\end{observacion}

\section{Anillos de grupos}

Dado un grupo $G$ definimos el anillo $\Z[G]$ como el $\Z$-módulo libre generado por los elementos de $G$. Es decir, un elemento en $\Z[G]$ es de la forma
\begin{equation*}
	\sum_i a_ig_i
\end{equation*}
donde $g_i\in G$ y $a_i\in \Z$ y $a_i=0$ para casi todo $i$.

Definiendo el producto en $\Z[G]$ de la siguiente forma
\begin{equation*}
	\left(\sum_i a_ig_i\right) \left(\sum_jb_jh_j\right) = \sum_i\sum_j a_ib_jg_ih_j
\end{equation*}
$\Z[G]$ tiene estructura de anillo con unidad. Notese que este anillo será conmutativo si y solo si $G$ es abeliano.

\begin{observacion}
	Una estructura de $G$-módulo sobre un grupo abeliano $A$ extiende de manera única a un $\Z[G]$-módulo por linealidad.
	%De igual manera, un homomorfismo de $G$-módulos extiende a un homomorfismo de $\Z[G]$-módulos.
	Por ello, a partir de ahora hablaremos indistintamente de $G$ y $\Z[G]$-módulos.
\end{observacion}

\section{Sucesiones exactas}

% introducir aqui las sucesiones exactas
\begin{definicion}
	Una sucesión de grupos $G_i$ y homomorfismos $f_i$
	\begin{equation*}
		\cdots \xrightarrow{f_{n-2}} G_{n-1} \xrightarrow{f_{n-1}} G_n \xrightarrow{f_{n}} G_{n+1} \xrightarrow{f_{n+1}} \cdots
	\end{equation*}
	se dice que es exacta en $G_n$ si $\Image(f_{n-1}) = \Ker(f_{n})$. Diremos que es la sucesión es exacta si lo es para todo $G_n$.
	
	Diremos que es una sucesión exacta corta cuando es de la forma
	\begin{equation}\label{eq:sec}
		1\xrightarrow{} G_1 \xrightarrow{f_1} G_2\xrightarrow{f_2} G_3 \xrightarrow{} 1
	\end{equation}
\end{definicion}

% poner mas bonito o redactar sin separar por lineas
\begin{observacion}\label{obs:exact}
	La definición de sucesión exacta corta dice que $f_2$ es inyectiva, $f_3$ sobreyectiva y $G_2/\Image(f_2)\cong G_3$.
	 
	Observemos que el recíproco también es cierto. % reciproco de q explicar mejor
	Dados dos grupos $G$ y $H$ y $\homo f G H$ un epimorfismo, se puede construir la sucesión exacta corta $\extension i f {\Ker(f)} G H$.
	De igual manera, si $N \norm G$, entonces $\extension i \pi N G {G/N}$ es una sucesión exacta corta.
\end{observacion}
%	\begin{demostracion}
%		$\Ker(f_2)=\Image(f_1)=\{1\} \implies f_2$ es inyectiva.
%		
%		$\Image(f_3)=\Ker(f_4)=G_3 \implies f_3$ es sobreyectiva.
%		
%		$\Ker(f_3)=\Image(f_2) \cong G_1 \implies {G_2}/{f_2(G_1)}\cong G_3$ por el Primer Teorema de Isomorfía.
%	\end{demostracion}

%\begin{definicion}
%	Sean $\extension {}{} A B C$ y $\extension {}{} {A'} {B'} {C'}$ dos sucesiones exactas. Un homomorfismo de sucesiones exactas
%\end{definicion}

\begin{proposicion}[Lema corto de los cinco]\label{prop:sflem}
	Si las filas del siguiente diagrama conmutativo son exactas y $g$ y $h$ son isomorfismos entonces $f$ es un isomorfismo.
	% https://q.uiver.app/?q=WzAsMTAsWzEsMCwiQSJdLFsxLDEsIkEnIl0sWzIsMCwiQiJdLFsyLDEsIkInIl0sWzMsMCwiQyJdLFszLDEsIkMnIl0sWzQsMCwiMSJdLFs0LDEsIjEiXSxbMCwwLCIxIl0sWzAsMSwiMSJdLFswLDEsImYiXSxbMCwyLCJpIl0sWzIsNCwiXFxwaSJdLFsyLDMsImciXSxbMSwzLCJpJyIsMl0sWzMsNSwiXFxwaSciLDJdLFs0LDUsImgiXSxbNCw2XSxbOSwxXSxbOCwwXSxbNSw3XV0=
	\[\begin{tikzcd}
		1 & {A_1} & {B_1} & {C_1} & 1 \\
		1 & {A_2} & {B_2} & {C_2} & 1
		\arrow["g", from=1-2, to=2-2]
		\arrow["{i_1}", from=1-2, to=1-3]
		\arrow["{\pi_1}", from=1-3, to=1-4]
		\arrow["f", from=1-3, to=2-3]
		\arrow["{i_2}"', from=2-2, to=2-3]
		\arrow["{\pi_2}"', from=2-3, to=2-4]
		\arrow["h", from=1-4, to=2-4]
		\arrow[from=1-4, to=1-5]
		\arrow[from=2-1, to=2-2]
		\arrow[from=1-1, to=1-2]
		\arrow[from=2-4, to=2-5]
	\end{tikzcd}\]
	\begin{demostracion}
		Para verlo, demostramos que $f$ es inyectiva y sobreyectiva.
		
		Comenzamos viendo que $f$ es inyectiva. Sea $x\in \Ker(f)$, por la conmutatividad del diagrama, $h(\pi_1(x))=\pi_2(f(x)) = \pi_2(1) = 1$. Entonces, $\pi_1(x)\in \Ker(h) = \{1\}$ y $x\in \Ker(\pi_1)$, por tanto existe $\tilde x\in A_1$ tal que $i_1(\tilde x)=x$, de nuevo por conmutatividad, $i_2(g(\tilde x))=f(i_1(\tilde x))=f(x)=1$ y como $i_2$ y $g$ son inyectivas, $\tilde x = 1$ y $x = i_1(\tilde x)=1$.
		
		Ahora vemos la sobreyectividad. Sea $x\in B_2$, como $\pi_1$ y $h$ son sobreyectivas existe $ x_1\in B_1$ tal que $\pi_2(x)=h(\pi_1(x_1)) = \pi_2(f(x_1))$. Por tanto, $x f(x_1)^{-1} \in \Ker(\pi_2)= \Image(i_2)$ y existe $\tilde x\in A_2$ tal que $i_2(\tilde x)=x f(x_1)^{-1}$. Usando que $g$ es isomorfismo, $i_2(\tilde x) = f(i_1(g^{-1}(\tilde x)))$. Despejando $x = f(i_1(g^{-1}(\tilde x)) x_1)$, $f$ es sobreyectiva.
	\end{demostracion}
\end{proposicion}



\section{Complejos de cadenas}
	
%Sea $R$ un anillo. Un $R$-módulo graduado es una sucesión $C=\{C_n\}_{n\in \Z}$ de $R$-módulos. 
%Una función de grado $r\in\Z$ entre dos $R$-módulos graduados $C$ y $C'$ es una sucesión de homomorfismos de $R$-módulos $f=\{\homo {f_n} {C_n} {C_{n+r}}\}_{n\in\Z}$.

%Un complejo de cadenas es un par $(C,d)$ donde $C$ es un $R$-módulo graduado y $d$ es un 
% la cohomologia mide como de lejos está el complejo de ser exacto.
Sea $R$ un anillo con unidad. Un $R$-módulo graduado es una sucesión de $R$-módulos $C=\{C_n\}_{n\in\Z}$. Diremos que los elementos del $R$-módulo $C_n$ tienen grado $n$.
Definimos un mapa de grado $r$ entre dos $R$-módulos graduados $C$ y $C'$ como una sucesión de homomorfismos de $R$-módulos $f=\{\homo {f_n} {C_n} {C_{n+r}'}\}_{n\in\Z}$.

Un complejo de cadenas sobre $R$ es el par $(C,d)$ donde $C$ es un $R$-módulo graduado y $\homo d C C$ un mapa de grado $-1$ llamado operador de borde que verifica $d_{n}\circ d_{n+1}=0$ para todo $n\in\Z$, escrito sin subíndices como $d^2 = 0$. Equivalentemente, el operador de borde verifica $\Image(d_{n+1})\subseteq \Ker(d_{n})$.

\begin{equation}
	\cdots  \xrightarrow{} C_{n+1} \xrightarrow{d_{n+1}} C_n \xrightarrow{d_n} C_{n-1} \xrightarrow{} \cdots 
\end{equation}

Definimos los $n$-ciclos y $n$-bordes como $Z_n(C)=\Ker(d_{n})$ y $B_n(C)=\Image(d_{n+1})$ respectivamente. Asimismo, definimos el $n$-ésimo grupo de homología del complejo $C$ como $H_n(C) = \frac{Z_n(C)}{B_n(C)}$. El complejo de cadenas $C$ será exacto si y solo si $H_n(C) = 0$ para todo $n\in\Z$. Así, los grupos de homología miden cómo de lejos está el complejo de ser exacto.

En el caso en que el operador $d$ tenga grado $1$ en vez de $-1$, se dirá que $(C,d)$ es un complejo de cocadenas y $d$ un operador de coborde. Se escribirá todo con superíndices y con el prefijo \textit{co-} para distinguirlo de un complejo de cadenas. De manera análoga definimos los $n$-cociclos, $n$-cobordes y el $n$-ésimo grupo de cohomología como $Z^n(C)=\Ker(d^n)$, $B^n(C)=\Image(d^{n-1})$ y $H^n(C)=\frac{Z^n(C)}{B^n(C)}$.

Notese que no hay ninguna diferencia esencial entre complejos de cadenas y cocadenas y siempre podemos pasar de uno a otro cambiando los índices $C^n = C_{-n}$. En la práctica, para que no sean equivalentes, se suele requerir que los complejos de cadenas verifiquen $C_n = 0$ para todo $n>0$ y los complejos de cocadenas $C^n=0$ para $n<0$.

\section{Cohomología de grupos}

%introduccion explicar que queremos construir como lo haremos, si eso quitar esta definicion y ponerlo en un parrafo

\begin{definicion}
	Sea $A$ un $G$-módulo con acción $\varphi$ y sea $n\in \N$. Denotamos por $C^n_\varphi(G,A)$ al conjunto de funciones de $G^n$ a $A$. Este conjunto junto con la operación $(f+g)(g_1,\ldots,g_n) = f(g_1,\ldots,g_n) + g(g_1,\ldots,g_n)$ para $f,g\in C^n$ y la acción de $G$ dada por $(g\cdot f)(g_1,\ldots,g_n) = g\cdot f(g_1,\ldots,g_n)$ para $f\in C^n$ y $g\in G$ hacen a $C^n_\varphi(G,A)$ un $G$-módulo. % de hecho es un ZQ-modulo de anillos 
	%(f+g)(q_1,...,q_n) = f(q_1,...,q_n)+g(q_1,...,q_n)
	% (qf)(q_1,...,q_n) = q f(q_1,...,q_n)
	% ((q_1+q_2)f)(q_1,...,q_n) = q_1f(q_1,...,q_n) + q_2f(q_1,...,q_n)
	A los elementos de $C^n_\varphi(G,A)$ los llamaremos $n$-cocadenas. A menudo escribiremos $C^n$ cuando esté claro de qué $G$-módulo estamos hablando. % esto suena cringe

	%$C=\bigoplus_{n=0}^\infty C^n$ para $f\in C^r$ y $g\in C^s$ definimos el producto de $f$ y $g$ como $(fg)(q_1,\ldots,q_{r+s}) = f()$
\end{definicion}


\begin{definicion}
	Dado $n\in \N$, el operador coborde $\homo {\partial^n} {C^n} {C^{n+1}}$ se define como
	\begin{equation*}
		\partial^n f = \sum\limits_{i=0}^{n+1} (-1)^{i} d_i f
	\end{equation*}
	donde el operador $d_i$ viene dado por 
	\[
		(d_if)(g_1,\ldots,g_n) = 
		\begin{cases*} % confuso porque en la formula de arriba f \in C^n y aqui f\in C^{n-1}
			g_1\cdot f(g_2,\ldots,g_n) 				& si $ i=0$ \\
			f(g_1,\ldots,g_ig_{i+1},\ldots,g_n) 	& si $ 0 < i <n$ \\ %f(q_1,\ldots,q_{i-1},q_iq_{i+1},q_{i+2},\ldots,q_{n+1}) & si $ 0 < i <n+1$ \\
			f(g_1,\ldots,g_{n-1}) 					& si $ i=n$
		\end{cases*}
	\]
	Diremos que $\partial f$ es el coborde de $f$. % definir el operador \partial C -> C
\end{definicion}

%Es obvio que el operador coborde es homomorfismo de Q-modulos

% dar indicacion o ejemplo del calculo
\begin{lema}\label{lem:coborde}
	Sea $f\in C^{n-1}$, entonces para todo $i=0,\ldots,n+1$ y $j=i,\ldots,n$
	\begin{equation}
		d_id_jf=d_{j+1}d_if
	\end{equation}
	\begin{demostracion}
		Para facilitar los calculos, extendemos $(g_1,\ldots,g_{n+1})$ a $(g_0,\ldots,g_{n+3})$ con $g_0=g_{n+2}=g_{n+3}=1$.
		Definimos la operación de $d_i$ sobre la tupla como $d_i(g_0,\ldots,g_{n+3}) = (g_0,\ldots,g_ig_{i+1},\ldots,g_{n+3})$. La tupla obtenida tras aplicar $d_i$ y $d_j$, se aplicará a $f$ de la siguiente forma 
		\begin{equation*}
			f\cdot (\tilde g_0,\ldots,\tilde g_{n+1}) = \tilde g_0 \cdot f(\tilde g_1,\ldots,\tilde g_{n-1})
		\end{equation*}
		esto es equivalente a calcular $d_i(d_jf)(g_1,\ldots,g_{n+1})$.
		
%		\textit{Caso $j=i$}
%		\begin{align*}
%			&(d_i(d_i))(1,1,g_1,\ldots,g_{n+2},1,1) = \\
%			&d_i(\ldots,1,g_1,\ldots,g_ig_{i+1},\ldots,g_n,1,\ldots) = \\
%			&(\ldots,1,g_1,\ldots,g_ig_{i+1}g_{i+2},\ldots,g_n,1,\ldots)
%		\end{align*}
%		\begin{align*}
%			&(d_{i+1}(d_i))(1,1,g_1,\ldots,g_{n+2},1,1) = \\
%			&d_i(\ldots,1,g_1,\ldots,g_{i+1}g_{i+2},\ldots,g_n,1,\ldots) = \\
%			&(\ldots,1,g_1,\ldots,g_ig_{i+1}g_{i+2},\ldots,g_n,1,\ldots)
%		\end{align*}
%		Son iguales
%		
%		\textit{Caso $j=i+1$}
%		\begin{align*}
%			&(d_i(d_{i+1}))(\ldots,1,g_1,\ldots,g_n,1,\ldots) = \\
%			&d_{i+1}(\ldots,1,g_1,\ldots,g_ig_{i+1},\ldots,g_n,1,\ldots) = \\
%			&(\ldots,1,g_1,\ldots,g_ig_{i+1},g_{i+2}g_{i+3},\ldots,g_n,1,\ldots)
%		\end{align*}
%		\begin{align*}
%			&(d_{i+2}(d_i))(\ldots,1,g_1,\ldots,g_n,1,\ldots) = \\
%			&d_i(\ldots,1,g_1,\ldots,g_{i+2}g_{i+3},\ldots,g_n,1,\ldots) = \\
%			&(\ldots,1,g_1,\ldots,g_ig_{i+1},g_{i+2}g_{i+3},\ldots,g_n,1,\ldots)
%		\end{align*}
%		Son iguales
%		
%		\textit{Caso $j>i+1$}
%		\begin{align*}
%			&(d_{j+1}(d_i))(\ldots,1,g_1,\ldots,g_n,1,\ldots) = \\
%			&d_i(\ldots,1,g_1,\ldots,g_{j+1}g_{j+2},\ldots,g_n,1,\ldots) = \\
%			&(\ldots,1,g_1,\ldots,g_ig_{i+1},\ldots,g_{j+1}g_{j+2},\ldots,g_n,1,\ldots)
%		\end{align*}
%		
%		\begin{align*}
%			(\partial^2(\partial^1f))(g_1,g_2,g_3) 
%			& = g_1\cdot (\partial^1f)(g_2,g_3) - (\partial^1f)(g_1g_2,g_3) + (\partial^1f)(g_1,g_2g_3) - (\partial^1f)(g_1,g_2) \\
%			& = g_1g_2\cdot f(g_3) - g_1\cdot f(g_2g_3) + g_1\cdot f(g_2) \\
%			& - g_1g_2\cdot f(g_3) + f(g_1g_2g_3) - f(g_1g_2) \\
%			& + g_1\cdot f(g_2g_3) - f(g_1g_2g_3) + f(g_1) \\
%			& - g_1\cdot f(g_2) + f(g_1g_2) - f(g_1)
%		\end{align*}
	\end{demostracion}
\end{lema}

\begin{teorema}\label{prop:cochaincomplex}
	Sea $n\in \N$, entonces $\partial^{n+1}\partial^n = 0$, lo que hace a
%	\begin{equation*}
%		C^0 \xrightarrow{\partial^0} C^1 \xrightarrow{\partial^1} C^2\xrightarrow{\partial^2} \cdots
%	\end{equation*}
	$(C,\partial)$ un complejo de cocadenas.
	\begin{demostracion}
		
		Para probarlo separaremos la siguiente suma en dos triangulos,  $j<i$ y $j\geq i$, aplicaremos el Lema \ref{lem:coborde} al segundo y finalmente reescribiremos los índices del sumatorio. 
		Sea $f\in C^n$ %y $g_1,\ldots,g_{n+2}\in G$, entonces
		\begin{align*}
		\partial^{n+1}(\partial^{n} f) 
			&= \sum_{i=0}^{n+2} (-1)^i d_i(\partial^n f) = \sum_{i=0}^{n+2}\sum_{j=0}^{n+1} (-1)^{i+j} d_i(d_jf)\\
			&= \sum_{0\leq j < i \leq n+2} (-1)^{i+j} d_i(d_jf)  + \sum_{0\leq i \leq j \leq n+1} (-1)^{i+j} d_i(d_jf)\\
			&= \sum_{0\leq j < i \leq n+2} (-1)^{i+j} d_i(d_jf)  + \sum_{0\leq i \leq j \leq n+1} (-1)^{i+j} d_{j+1}(d_if) \\
			&= \sum_{0\leq j < i \leq n+2} (-1)^{i+j} d_i(d_jf)  + \sum_{0\leq j < i \leq n+2} (-1)^{i+j-1} d_{i}(d_jf) \\
			&= \sum_{0\leq j < i \leq n+2} (-1)^{i+j}(d_i(d_jf)-d_i(d_jf)) = 0
		%(\partial^{n+1}(\partial_{n} f))(g_1,\ldots, g_{n+2}) = \sum\limits_{i=0}^{n+2}\sum\limits_{j=0}^{n+1} (-1)^{i+j} (d_i(d_jf))(g_1,\ldots, g_{n+2})
		\end{align*}
	\end{demostracion}
\end{teorema}

% simplificar esto haciendo referencia a la seccion de complejos de cocadenas y escribir solo lo de la accion 
\begin{definicion}
	La proposición anterior nos permite definir los grupos de $n$-cobordes y $n$-cociclos como 
	\begin{align*}
		B^n_\varphi(G,A) &= \Image(\partial^{n-1}) \\
		Z^n_\varphi(G,A) &= \Ker(\partial^{n})
	\end{align*}
	
	Con ellos podemos definir entonces el $n$-ésimo grupo de cohomología como 
	\begin{equation}
		H^n_\varphi(G,A) = \frac{Z^n_\varphi(G,A)}{B^n_\varphi(G,A)} = \frac{\Image(\partial^{n-1})}{\Ker(\partial^{n})}
	\end{equation}
	
	Dado $f\in Z^n(G,A)$, denotaremos su clase de equivalencia por $[f]\in H^n(G,A)$. Diremos que dos cociclos que pertenecen a la misma clase son cohomólogos. 
\end{definicion}

\begin{teorema}\label{thm:trivialH}
	% dicho de otra forma, H^n(Q,A) es un Z_m-modulo donde m=|Q|
	Sea $A$ un $G$-módulo con $\mcd(\ord{G}, \ord{A}) = 1$. Entonces $H^n(G,A)$ es trivial para todo $n\in\Z^+$.
	\begin{demostracion}
		Sea $c\in Z^n(Q,A)$
%		\begin{multline}\label{eq:ncocycle}
%			q_1\cdot c(q_2,\hdots,q_{n+1}) + \left[\sum\limits_{i=1}^{n} (-1)^i c(q_1,\ldots,q_iq_{i+1},\hdots,q_{n+1}) \right] + \\ + (-1)^{n+1}c(q_1,\hdots,q_n) = 0
%		\end{multline}		
		\begin{equation}\label{eq:ncocycle}
			\sum\limits_{i=0}^{n+1} (-1)^i(d_ic)(g_1,\ldots,g_{n+1}) = 0
		\end{equation}
		
		Definiendo $\tilde c(g_1,\hdots,g_{n-1}) = {\displaystyle \sum_{g_n\in G} c(g_1,\hdots,g_{n})}$ y haciendo la suma en \eqref{eq:ncocycle} para cada $g_{n+1} \in G$
%		\begin{multline*}
%			q_1\cdot \tilde c(q_2,\hdots,q_n) + \left[\sum\limits_{i=1}^{n-1}(-1)^i \tilde c(q_1,\hdots,q_iq_{i+1},\hdots,q_{n}) \right] +\\+ (-1)^n\tilde c(q_1,\hdots,q_{n-1})+ \ord{Q}(-1)^{n+1} c(q_1,\hdots,q_n) = 0
%		\end{multline*}
		
%		\begin{multline*}
%			\sum\limits_{q_{n+1}\in Q}\sum\limits_{i=0}^{n+1} (-1)^i(d_ic)(q_1,\ldots,q_{n+1}) = \\ =			
%			\sum\limits_{q_{n+1}\in Q}d_0c(q_1,\ldots,q_{n+1}) + 
%			\sum\limits_{q_{n+1}\in Q}\sum\limits_{i=i}^{n-1} (-1)^i(d_ic)(q_1,\ldots,q_{n+1}) + \\ +
%			\sum\limits_{q_{n+1}\in Q}d_n c(q_1,\ldots,q_{n+1}) + 
%			\sum\limits_{q_{n+1}\in Q}d_{n+1}c(q_1,\ldots,q_{n+1}) = \\ =			
%			q_1 \cdot \tilde c(q_1,\ldots,q_n) + 
%			\sum\limits_{i=i}^{n-1} (-1)^i(d_i\tilde c)(q_1,\ldots,q_n) + \\ +
%			(-1)^n\tilde c(q_1,\ldots,q_{n-1}) + 
%			(-1)^{n+1}\ord{Q}\tilde c(q_1,\ldots,q_n) = 0
%		\end{multline*}

		\begin{multline*}
			\sum\limits_{g_{n+1}\in G}\sum\limits_{i=0}^{n+1} (-1)^i(d_ic)(g_1,\ldots,g_{n+1}) = \\ =			
			%g_1 \cdot \tilde c(g_2,\ldots,g_n) + 
			\sum\limits_{i=0}^{n} (-1)^i(d_i\tilde c)(g_1,\ldots,g_n) +
			(-1)^{n+1}\ord{G} c(g_1,\ldots,g_n) = 0
		\end{multline*}
		
		Como $\mcd(\ord{G}, \ord{A}) = 1$, por la identidad de Bezout existe $k\in \N$ tal que $k \ord{G} \equiv (-1)^{n} \mod \ord{A}$. Multiplicando por $k$
		\begin{equation*}
			c(g_1,\hdots,g_n) = \sum\limits_{i=0}^n (-1)^i(d_i k\tilde c)(g_1,\ldots,g_n)
		\end{equation*}
		
		El cociclo $c$ es el coborde de la función $k \tilde c$ y por tanto $H^n(G,A)$ es trivial.
		
		
	\end{demostracion}
\end{teorema}

\section{Cociclos normalizados y $H^2$}

 \textit{Resultados que se usan en el capitulo de extensiones}

%\begin{definicion}% en general un n-cociclo f es normalizado cuando f(q_1,...,q_n) = 0 si cualquiera de los q_i = 0
%	Un $2$-cociclo $c\in Z^2(Q,A)$ se dice que es normalizado cuando $c(1,1)= 0$
%\end{definicion}
%
%% esto tb es cierto para un cociclo no normalizado con accion trivial
%\begin{proposicion}
%	Sea $c\in Z^2(Q,A)$ un $2$-cociclo normalizado. Entonces para todo $q\in Q$
%	\begin{equation*}
%		c(1,q)=0=c(q,1)
%	\end{equation*}
%	\begin{demostracion}
%		Evaluando la condición de un $2$-cociclo en $(1,1,q)\in Q^3$
%		\begin{equation*}
%			1\cdot c(1,q) - c(1,q) + c(1,q) - c(1,1) = c(1,q) =0
%		\end{equation*} 
%		De igual manera, evaluando en $(q,1,1)\in Q^3$
%		\begin{equation*}
%			q\cdot c(1,1) - c(q,1) + c(q,1) - c(q,1) = -c(q,1) = 0
%		\end{equation*}
%	\end{demostracion}
%\end{proposicion}

\begin{definicion}% en general un n-cociclo f es normalizado cuando f(q_1,...,q_n) = 0 si cualquiera de los q_i = 0
	Un $2$-cociclo $c\in Z^2(G,A)$ se dice que es normalizado cuando para todo $g\in G$
	\begin{equation*}
		c(1,g)= 0 = c(g,1)
	\end{equation*}
	% definimos Z^2_N y B^2_N
\end{definicion}

\begin{proposicion}
	$c\in Z^2(G,A)$ es un $2$-cociclo normalizado si y solo si $c(1,1)=0$.
	\begin{demostracion}
		Sea $c(1,1) = 0$. Evaluando $\partial^2 c$ en $(1,1,g)\in G^3$
		\begin{equation*}
			1\cdot c(1,g) - c(1,g) + c(1,g) - c(1,1) = c(1,g) =0
		\end{equation*} 
		De igual manera, evaluando en $(g,1,1)\in G^3$
		\begin{equation*}
			g\cdot c(1,1) - c(g,1) + c(g,1) - c(g,1) = -c(g,1) = 0
		\end{equation*}
	\end{demostracion}
\end{proposicion}

\begin{lema}\label{prop:normcoc}
	Todo $2$-cociclo $c\in Z^2(G,A)$ es cohomólogo a un $2$-cociclo normalizado. %Esto es, $\frac{Z^{*n}_\varphi(Q,A)}{B^n_\varphi(Q,A)} = \frac{Z^n_\varphi(Q,A)}{B^n_\varphi(Q,A)}$
	\begin{demostracion}
		Tomamos una función $\homo \phi G A$ tal que $\phi(1) = -c(1,1)$ y construímos el $2$-coborde b de $\phi$
		\begin{equation*}
			b(g_1,g_2) = (\partial^1 \phi)(g_1,g_2) =  q_1\cdot \phi(g_2) - \phi(g_1g_2) + \phi(g_1)
		\end{equation*}
		$b$ verifica que $b(1,1) = \phi(1) = -c(1,1)$ y por tanto el cociclo $\tilde c = c+b$ es un $2$-cociclo normalizado.
	\end{demostracion}
\end{lema}

\begin{teorema}\label{thm:h2hn2}
	Tenemos la siguiente biyección
	\begin{equation}
		H^2_N(G,A) = \frac{Z^2_N(G,A)}{B^2_N(G,A)} \cong \frac{Z^2(G,A)}{B^2(G,A)} = H^2(G,A)
	\end{equation}
	\begin{demostracion}
		Por el Lema \ref{prop:normcoc}, tenemos el siguiente epimorfismo
		\begin{align*}
			\Phi \ \colon \ Z^2_N(G,A) &\twoheadrightarrow H^2(G,A) \\
			c \ &\mapsto [c]
		\end{align*}
		
		El kernel de $\Phi$ es precisamente la intersección de $Z^2_N(G,A)$ y $B^2(G,A)$, esto es, $B^2_N(G,A)$. Por el Primer Teorema de Isomorfía, se sigue el resultado.
	\end{demostracion}
\end{teorema}


% !TeX root=../tfg2.tex

% TODO
% G-modulos

\chapter{Extensiones de grupos}

\begin{definicion}
	Una sucesión de grupos $G_i$ y homomorfismos $f_i$
	\begin{equation*}
		G_0\xrightarrow{f_1} G_1 \xrightarrow{f_2} G_2\xrightarrow{f_3} \cdots \xrightarrow{f_n} G_n \xrightarrow{f_{n+1}} \cdots
	\end{equation*}
	se dice que es exacta si $\Ker(f_{i+1})=\Image(f_i)$. Diremos que es una sucesión exacta corta cuando es de la forma
	\begin{equation}\label{eq:sec}
		1\xrightarrow{} G_1 \xrightarrow{f_2} G_2\xrightarrow{f_3} G_3 \xrightarrow{} 1
	\end{equation}
\end{definicion}

\begin{observacion}
	Una sucesión exacta corta como \eqref{eq:sec} es equivalente a decir que $f_2$ es inyectiva, $f_3$ sobreyectiva y $G_2/f_2(G_1)\cong G_3$
	\begin{demostracion}
		$\Ker(f_2)=\Image(f_1)=\{1\} \implies f_2$ es inyectiva.
		
		$\Image(f_3)=\Ker(f_4)=G_3 \implies f_3$ es sobreyectiva.
		
		$\Ker(f_3)=\Image(f_2) \cong G_1 \implies {G_2}/{f_2(G_1)}\cong G_3$ por el Primer Teorema de Isomorfía.
	\end{demostracion}
\end{observacion}

\begin{definicion}
	Una extensión de un grupo $Q$ por un grupo $N$ es una sucesión exacta corta
	\begin{equation}\label{eq:ext}
		\extension i \pi N E Q 
	\end{equation}
	Decimos que otra extensión $\extension {i'} {\pi'} N {E'} Q $ es equivalente si existe un homomorfismo $\homo f E {E'}$ tal que el siguiente diagrama conmuta:

%	\begin{figure}[h!] % quitar figure?
%		\centering	
		% https://q.uiver.app/?q=WzAsMTAsWzEsMCwiTiJdLFsyLDAsIkUiXSxbMywwLCJRIl0sWzQsMCwiMSJdLFswLDAsIjEiXSxbMCwxLCIxIl0sWzEsMSwiTiJdLFszLDEsIlEiXSxbNCwxLCIxIl0sWzIsMSwiRSciXSxbNCwwXSxbMCwxLCJpIl0sWzEsMiwiXFxwaSJdLFsyLDNdLFs1LDZdLFs2LDksImknIl0sWzksNywiXFxwaSciXSxbNyw4XSxbMiw3LCIiLDEseyJvZmZzZXQiOi0xLCJzdHlsZSI6eyJoZWFkIjp7Im5hbWUiOiJub25lIn19fV0sWzcsMiwiIiwxLHsib2Zmc2V0IjotMSwic3R5bGUiOnsiaGVhZCI6eyJuYW1lIjoibm9uZSJ9fX1dLFsxLDksImYiXSxbMCw2LCIiLDEseyJvZmZzZXQiOi0xLCJzdHlsZSI6eyJoZWFkIjp7Im5hbWUiOiJub25lIn19fV0sWzYsMCwiIiwxLHsib2Zmc2V0IjotMSwic3R5bGUiOnsiaGVhZCI6eyJuYW1lIjoibm9uZSJ9fX1dXQ==
\begin{equation}\begin{tikzcd}
	1 & N & E & Q & 1 \\
	1 & N & {E'} & Q & 1
	\arrow[from=1-1, to=1-2]
	\arrow["i", from=1-2, to=1-3]
	\arrow["\pi", from=1-3, to=1-4]
	\arrow[from=1-4, to=1-5]
	\arrow[from=2-1, to=2-2]
	\arrow["{i'}", from=2-2, to=2-3]
	\arrow["{\pi'}", from=2-3, to=2-4]
	\arrow[from=2-4, to=2-5]
	\arrow[shift left=1, no head, from=1-4, to=2-4]
	\arrow[shift left=1, no head, from=2-4, to=1-4]
	\arrow["f", from=1-3, to=2-3]
	\arrow[shift left=1, no head, from=1-2, to=2-2]
	\arrow[shift left=1, no head, from=2-2, to=1-2]
\end{tikzcd}
\end{equation}
	%\end{figure}
	% observar que necesariamente f es iso
\end{definicion}

% limpiar
\begin{observacion}\label{prop:eqiso}Dos extensiones equivalentes son isomorfas.
	\begin{demostracion}
		Sea $x_2\in E$ tal que $f(x_2)=1$, por la conmutatividad de $f$, $\pi(x_2)=\pi'(f(x_2)) = \pi'(1) = 1$. $x_2\in \Ker(\pi) = Im(i)$, por tanto existe $x_1\in N$ tal que $i(x_1)=x_2$, utilizando que $f\circ i=i'$, $i'(x_1)=f(i(x_1))=1$ y como $i$ es inyectiva, $x_2$ es único y $f$ es inyectiva.
		
		Sea $x_2'\in E'$, como $\pi$ es sobreyectiva existe $x_2\in E$ tal que $\pi'(x_2')=\pi(x_2) = \pi'(f(x_2))$. Por tanto, $x_2' f(x_2)^{-1} \in \Ker(\pi')=Im(i')$ y existe $x_1\in N$ tal que $f(i(x_1))=i'(x_1)=x_2' f(x_2)^{-1}$, $x_2' = f(i(x_1)x_2)$ y $f$ es sobreyectiva.
	\end{demostracion}
\end{observacion}

% define isomorfismo de extensiones, reeescribir demostracion
\begin{observacion}
	Ser una equivalencia de extensiones es más débil que ser un isomorfismo, como se ve en el siguiente ejemplo.
	% https://q.uiver.app/?q=WzAsMTAsWzAsMCwiMSJdLFsxLDAsIlxcbWF0aGJie1p9XzMiXSxbMiwwLCJcXG1hdGhiYntafV85Il0sWzMsMCwiXFxtYXRoYmJ7Wn1fMyJdLFs0LDAsIjEiXSxbMCwxLCIxIl0sWzEsMSwiXFxtYXRoYmJ7Wn1fMyJdLFsyLDEsIlxcbWF0aGJie1p9XzkiXSxbMywxLCJcXG1hdGhiYntafV8zIl0sWzQsMSwiMSJdLFswLDFdLFsxLDIsIlxcdGltZXMgMyJdLFsyLDMsIlxcdGltZXMgMSJdLFszLDRdLFs1LDZdLFs2LDcsIlxcdGltZXMgMyJdLFs3LDgsIlxcdGltZXMgMiJdLFs4LDldLFsxLDYsIiIsMSx7Im9mZnNldCI6LTEsInN0eWxlIjp7ImhlYWQiOnsibmFtZSI6Im5vbmUifX19XSxbNiwxLCIiLDEseyJvZmZzZXQiOi0xLCJzdHlsZSI6eyJoZWFkIjp7Im5hbWUiOiJub25lIn19fV0sWzIsNywiZiIsMl0sWzMsOCwiIiwxLHsib2Zmc2V0IjotMSwic3R5bGUiOnsiaGVhZCI6eyJuYW1lIjoibm9uZSJ9fX1dLFs4LDMsIiIsMSx7Im9mZnNldCI6LTEsInN0eWxlIjp7ImhlYWQiOnsibmFtZSI6Im5vbmUifX19XV0=
\[\begin{tikzcd}
	1 & {\mathbb{Z}_3} & {\mathbb{Z}_9} & {\mathbb{Z}_3} & 1 \\
	1 & {\mathbb{Z}_3} & {\mathbb{Z}_9} & {\mathbb{Z}_3} & 1
	\arrow[from=1-1, to=1-2]
	\arrow["{\times 3}", from=1-2, to=1-3]
	\arrow["{\times 1}", from=1-3, to=1-4]
	\arrow[from=1-4, to=1-5]
	\arrow[from=2-1, to=2-2]
	\arrow["{\times 3}", from=2-2, to=2-3]
	\arrow["{\times 2}", from=2-3, to=2-4]
	\arrow[from=2-4, to=2-5]
	\arrow[shift left=1, no head, from=1-2, to=2-2]
	\arrow[shift left=1, no head, from=2-2, to=1-2]
	\arrow["f"', from=1-3, to=2-3]
	\arrow[shift left=1, no head, from=1-4, to=2-4]
	\arrow[shift left=1, no head, from=2-4, to=1-4]
\end{tikzcd}\]
	\begin{demostracion}
		Un automorfismo $f$ de $\mathbb{Z}_9$ viene dado por $f(x) = kx$ con $x\in \mathbb{Z}_9$ y $k\in \mathbb{Z}_9^{\times} = \{1,2,4,5,7,8\}$		 
		
		Para que el diagrama conmute a la derecha, $(\times2\circ f)(x) = 2kx = x \mod 3$, $k \equiv 2 \mod 3 \implies k = 2,5,8$
		
		Por otro lado, para que conmute a la izquierda, $(f\circ \times3)(x) = 3kx = 3x \mod 9$, por lo que $k = 1,4,8$.
		
		Por tanto, no existe un isomorfismo $f$ que haga al diagrama conmutativo y las extensiones no son equivalentes.
	\end{demostracion}
\end{observacion}

% definir E_j, \pi_j e i_j?
\begin{proposicion}
	La equivalencia de extensiones es una relación de equivalencia.
	\begin{demostracion}
		\begin{enumerate}
			\item Reflexiva: $E$ es equivalente a sí misma tomando $f=1_E$
			\item Simétrica: Si $\homo f {E_1} {E_2}$ es una equivalencia, por la Observación \eqref{prop:eqiso}, $\homo {f^{-1}} {E_2} {E_1}$ es una equivalencia. % obvio? f \circ i_1 = i_2 \implies 
						\item Transitiva: Si $\homo f {E_1} {E_2}$ y $\homo g {E_2} {E_3}$ son equivalencias, $g\circ f\circ i_1 = g\circ i_2 = i_3$ y $\pi_1 \circ g \circ f = \pi_2 \circ f = \pi_3$ entonces $\homo {g\circ f} {E_1} {E_3}$ es una equivalencia.
		\end{enumerate}
	\end{demostracion}
\end{proposicion}

\begin{observacion}\label{extact}
	Una extensión \eqref{eq:ext} determina, por conjugación por elementos de $E$, un homomorfismo $\homo \alpha E {\Aut(N)}$ definido por % 
	\begin{equation*}
		\alpha(g)(n) = n^g = g^{-1}ng
	\end{equation*}
	
	Entonces, $\alpha(N) = \Inn(N)$ y $\alpha$ induce un homomorfismo $\homo {\tilde\alpha} {E/N} {\Out(N)}$ % explicar que esta bien definida
	\begin{equation*}
		\tilde\alpha(gN) = \overline{\alpha(g)} % esto es el abstract kernel de la extension
	\end{equation*}
	
	El homomorfismo $\tilde\alpha$ se conoce como el kernel abstracto de la extensión. % no se usa
	
	Fijando una sección $s$ de $\pi$, para todo $q\in Q$, la conjugación por $s(q)$ determina un automorfismo $\varphi(s(q))$ de $N$ definido por $\varphi(s(q))(n) = \alpha(s(q))(n)$. Notese que la función $\homo \varphi Q {\Aut(N)}$ no es necesariamente un homomorfismo de grupos, pero sí lo es salvo automorfismos internos. En particular, si la sección $s$ es un homomorfismo o el grupo de automorfismos internos de $N$ es trivial, como se estudia en las secciones \ref{sec:split} y \ref{sec:ab}, entonces $\varphi$ sí es un homomorfismo y podremos hablar de la acción de la extensión. %%?????????????????
\end{observacion}

\begin{observacion}% extensiones equivalentes dan lugar al mismo kernel abstracto
	Dos extensiones $E$ y $E'$ equivalentes vienen dadas por una misma acción de $Q$ en $N$. Por ello, para estudiar las extensiones salvo equivalencia podemos fijar una acción $\homo \varphi Q {\Aut(N)}$ y estudiar las extensiones que dan lugar a esa acción.
	\begin{demostracion}
	 	$f\left(i(n)^{s(q)}\right) = f(i(n))^{f(s(q))} = i'(n)^{s'(q)}$
	 \end{demostracion}
\end{observacion}

\begin{definicion}
	Denotamos por $\Ext_\varphi(Q,N)$ a las clases de extensiones equivalentes que dan lugar a la acción $\varphi$ de $Q$ en $N$.
\end{definicion}
% !TeX root=../tfg2.tex

\section{Extensiones separables}\label{sec:split}


\begin{definicion}	
	Decimos que una extensión $\extension i \pi N E Q$ es separable cuando existe una sección $\homo s Q E$ de $\pi$ que es un homomorfismo.
\end{definicion}

%G tiene un subgrupo isomorfo a Q, interseca trivialmente con N por ser una sucesion exacta y por tanto es un producto semidirecto...
\begin{teorema}\label{splitext}
	Una extensión $\extension i \pi N E Q$ separable es equivalente a $\extension {} {} N {Q\ltimes_{\varphi} N} Q$, donde $\homo \varphi Q \Aut(N)$ es una acción de grupos de $Q$ en $N$. % explicar que es una accion y no una funcion
	\begin{demostracion}
		Si $E$ es separable, existe una sección $\homo s Q E$ que es homomorfismo. $s$ es inyectiva porque es una inversa por la derecha de $\pi$ y, por tanto, $Q\cong s(Q) \leq E$.
		$\pi (i(N)) = \{1\}$ y $s(q)\in i(N)$ sí y solo sí $q = 1$, por lo que $s(Q)\cap i(N) = {1_E}$ y $E$ es isomorfo a un producto semidirecto externo de $Q$ por $N$.
		
		El isomorfismo $\homo f {Q\ltimes N} E$ viene dado por $(q,n)\in Q\ltimes N$, $f(q,n) = s(q)i(n)$. Para ver que es un isomorfismo basta comprobar que la operación de grupo en $E$ es compatible con la del producto semidirecto.
		
		Sean $n_1,n_2\in N$ y $q_1,q_2\in Q$
		\begin{equation*}
			s(q_1)i(n_1)s(q_2)i(n_2) = s(q_1)s(q_2)i(n_1)^{q_2}i(n_2) = s(q_1q_2)i(n_1^{q_2}n_2) 
		\end{equation*}
		
		Para ver que las extensiones son equivalentes, se toman la inclusión $i'$ y proyección $\pi'$ canónicas de $Q\ltimes N$ y se verifica trivialmente que $i=f \circ i'$ y $\pi = \pi' \circ f$.
		% <=
		%En la otra dirección, definiendo para $n\in N, \ q\in Q$ $i(n) = (1_Q,n)$ y $\pi((n,q)) = qN$ % TODO
	\end{demostracion}
\end{teorema}


% Splittings (Escisiones)
\subsection{Clasificación de las escisiones}

Hemos visto que las extensiones separables son únicas salvo equivalencia. A continuación, tomaremos la extensión separable canónica

\begin{equation}\label{extsplit}
	\extension i \pi A {Q \ltimes A} Q
\end{equation}
y daremos una clasificación de todas las escisiones posibles de la extensión cuando A es abeliano.
% caso no abeliano H^1 blablabla

\begin{definicion}
	Diremos que dos escisiones $s_1$ y $s_2$ son $A$-conjugadas si existe un $a\in A$ tal que $s_1(q)=s_2(q)^{i(a)}$ para todo $q\in Q$
\end{definicion}

% escisiones vienen dadas por 1-cociclos
\begin{proposicion}
	Las escisiones de \eqref{extsplit} son homomorfismos de la forma $s(q) = (q,c(q))$ donde $\homo c Q A$ es un $1$-cociclo.
	\begin{proof}
		Una sección $s$ de $\pi$ tiene la forma $s(q) = (q,c(q))$ donde $c$ es una función $\homo c Q A$. Imponiendo que la sección sea un homomorfismo
		\begin{align*}
			s(q_1)s(q_2) &= (q_1q_2,c(q_1)\cdot q_2 + c(q_2)) \\
			s(q_1q_2) &= (q_1q_2,c(q_1q_2))
		\end{align*}
		la función $c$ tiene que verificar la ecuación de un $1$-cociclo para que $s$ sea una escisión.
		\begin{equation}
			c(q_1q_2) = c(q_1)\cdot q_2 + c(q_2)
		\end{equation}
	\end{proof}
\end{proposicion}



\begin{proposicion}
	Sean $s_1$ y $s_2$ dos escisiones y $c_1$ y $c_2$ los $1$-cociclos asociados. Entonces $s_1$ y $s_2$ son $A$-conjugadas si $c_1$ y $c_2$ se diferencian en un $1$-coborde.
	\begin{demostracion}
		Si existe un $a \in A$ tal que $s_1(q) = s_2(q)^{i(a)}$ para todo $q\in Q$, 
		\begin{align*}
		(q,c_1(q)) &= (1,-a)(q,c_2(q))(1,a) \\
					&= (q,-a \cdot q +  c_2(q) + a)
		\end{align*}
		
		La diferencia entre $c_1$ y $c_2$ verifica la ecuación de un $1$-coborde
		
		\begin{equation}
			c_2(q) - c_1(q) = a\cdot q - a
		\end{equation}
	\end{demostracion}
\end{proposicion}

%\begin{observacion}
%	Las definiciones de $1$-cociclos y $1$-cobordes dadas se corresponden con las definiciones usuales en el caso en que $N$ es abeliano.
%	\begin{align}
%		c(q_1q_2) = c(q_1)^{q_2}c(q_2) = c(q_1)\cdot q_2 + c(q_2) \\
%		c_1(q) = n^{-q}c_2(q)n \iff c_2(q)-c_1(q) = n\cdot q - n
%	\end{align}
%\end{observacion}

% explicar que se corresponde con el grupo de cohomologia en el caso abeliano
\begin{teorema}\label{h1}
	Sea $E$ una extensión separable de $Q$ por $A$. Entonces, las clases de escisiones $A$-conjugadas están en correspondencia uno a uno con los elementos de $H^1(Q,A)$.
\end{teorema}

% se comprueba trivialmente que es una relacion de equivalencia
% !TeX root=../tfg2.tex

\section{Extensiones con kernel abeliano}\label{sec:ab}

A continuación estudiaremos el caso en que $N$ es un grupo abeliano que a partir de ahora denotaremos por $A$. 

Por la Observación \ref{extact}, como $\Inn(A)$ es trivial, $Out(N)=Aut(N)$ y la acción de $Q$ en $A$ es un homomorfismo de grupos, lo que hace a $A$ un $Q$-módulo. A partir de ahora, fijaremos una acción de $Q$ en $A$ y estudiaremos todas las extensiones que dan lugar a dicha acción.

\begin{equation}\label{exta}
	1\to A\xrightarrow{i} E\xrightarrow{\pi} Q\to 1
\end{equation}

\begin{equation}
	\varphi \colon Q \to Aut(A)
\end{equation}

% suena raro lo del axioma de eleccion aprende a escribir
Para estudiar esta extensión, consideramos una sección $s$ de $\pi$, que en el caso de $Q$ infinito se puede asegurar que existe asumiendo el axioma de elección. Como $Q\cong E/i(A)$, dados $g,h\in Q$, $\pi\left(s(g)s(h)s(gh)^{-1}\right) = 1_{Q}$ por ser $\pi$ homomorfismo. Por tanto, $s(gh)$ y $s(g)s(h)$ distan en un elemento de $i(A)$ y podemos definir una función $\homo c {Q\times Q} {A}$ que mide cuánto dista $s$ de ser un homomorfismo

\begin{equation}\label{eq:factorsystem}
	s(g)s(h) = i\left(c(g,h)\right)s(gh)
\end{equation}

Podemos recuperar la extensión \eqref{exta} a partir de la acción $\varphi$ que hemos fijado y de la función $c$. 
Como $E=\bigcup\limits_{q\in Q} i(A)s(q) = i(A)s(Q)$ es una unión disjunta, podemos expresar unívocamente cada elemento de $E$ como un producto de elementos de $i(A)$ y $s(Q)$. Es decir, tenemos una biyección $A\times Q \to E$. A partir del producto en $E$, podemos definir una operación de grupo en $A\times Q$, que denotaremos por $E_{c}$. Dados $(a_1,q_1),(a_2,q_2)\in A\times Q$ tenemos:
\begin{align}\begin{split}
i(a_1)s(q_1)i(a_2)s(q_2) &= i(a_1)i(q_1\cdot a_2)s(q_1)s(q_2) \\ &= i(a_1+ q_1\cdot a_2 + c(q_1,q_2))s(q_1q_2)
\end{split}\end{align}
\noindent Por tanto, la operación en $E_c$ viene dada por:

\begin{equation}\label{extop}
(a_1,q_1)(a_2,q_2) = (a_1 + q_1\cdot a_2 + c(q_1,q_2),q_1q_2)
\end{equation}

Notese que este producto no depende directamente de la sección $s$ escogida. Por ello, supondremos que la sección $s$ es normalizada 

\begin{equation}
	s(1) = 1
\end{equation}
de donde obtenemos que $c$ verifica la siguiente condición de normalización
\begin{equation}\label{cocnorm}
	c(1,q) = 0 = c(q,1)
\end{equation}
De esta forma, el isomorfismo $\homo f {E_c} E$ viene dado por $f(a,q) = i(a)s(q)$. La inclusión de $A$ a $E_c$ y la proyección a $Q$ son las canónicas, haciendo a la extensión $E_c$ equivalente a \eqref{exta}.

%Queda demostrada la siguiente proposición.
%\begin{proposicion}\label{prop:equivcoc}
%	La extensión $\extension {i'} {\pi'} A {E_c} Q$ es equivalente a \eqref{exta}.
%\end{proposicion}

%La siguiente proposición prueba que una función $\homo c {Q\times Q} A$ define una operación de grupo en $E_c$ con la acción $\varphi$ cuando $c$ es un 2-cociclo.

\begin{proposicion}\label{prop:res1}
Sea $\varphi$ una acción de $Q$ en $A$ y $\homo c {Q\times Q} A$ una función que verifica la condición de normalización \eqref{cocnorm}. Entonces, la operación \eqref{extop} define una extensión de $Q$ por $A$ cuando $c$ es un $2$-cociclo normalizado.
	\begin{demostracion}
	Para ver que la funcion define una operación de grupo comprobamos la asociatividad y la existencia de identidad e inversos.
	%	(a_1+q_1\cdot a_2 + c(q_1,q_2),q_1q_2)(a_3,q_3) = 
%	(a_1+q_1\cdot a_2 + c(q_1,q_2) + q_1q_2\cdot a_3 + c(q_1q_2,q_3),q_1q_2q_3)
	
%	(a_1,q_1)(a_2+q_2\cdot a_3 + c(q_2,q_3),q_2q_3) = 
%	(a_1 + q_1\cdot a_2 + q_1q_2\cdot a_3 + q_1\cdot c(q_2,q_3) + c(q_1,q_2q_3),q_1q_2q_3)
	
%	c(q_1,q_2) + c(q_1q_2,q_3) = q_1\cdot c(q_2,q_3) + c(q_1,q_2q_3)

	\textit{(i) Asociatividad.}
	Imponiendo que para todo $(a_i,q_i)\in A\times Q$ con $i=1,2,3$
	\begin{equation*}
		[(a_1,q_1)(a_2,q_2)](a_3,q_3) = (a_1,q_1)[(a_2,q_2)(a_3,q_3)]
	\end{equation*} 
	llegamos a que $c$ verifica la ecuación de un $2$-cociclo normalizado	
	\begin{equation}\label{eq2coc}	
		 q_1\cdot c(q_2,q_3) - c(q_1q_2,q_3) + c(q_1,q_2q_3) - c(q_1,q_2) = 0
	\end{equation}
	
	\textit{(ii) Identidad.} Comprobamos que $(0,1)$ es la identidad de $E_c$. 
	Sea $(a,q)\in A\times Q$,
	\begin{align*}
		(0,1)(a,q) &= (0 + 1\cdot a + c(1,q), q) = (a,q) \\
		(a,q)(0,1) &= (a+q\cdot 0 + c(q,1),q) = (a,q)
	\end{align*}
	
	\textit{(iii) Inverso.} Se comprueba utilizando \eqref{eq2coc} que el inverso de $(a,q)\in A\times Q$ es $(-q^{-1}\cdot a -c(q^{-1},q),q^{-1})$.
%	\begin{align*}
%		(a,q)(-q^{-1}\cdot a -c(q^{-1},q),q^{-1}) 
%		&= (a+q\cdot (-q^{-1}\cdot a -c(q^{-1},q)) + c(q,q^{-1}),1)   \\
%		&= (a - a - q\cdot c(q^{-1},q)+c(q,q^{-1}),1) \\
%		&= (c(q,1)-c(1,q),1) = (0,1) \\
%		(-q^{-1}\cdot a -c(q^{-1},q),q^{-1})(a,q)
%	\end{align*}
	
	Finalmente, comprobamos que la inclusión $i$ y proyección $\pi$ canónicas de $A\times Q$ son homomorfismos y hacen a la sucesión exacta.
	\begin{equation*}
		i(a_1)i(a_2)=(a_1 + 1 \cdot a_2+c(1,1),1) = (a_1+a_2,1)=i(a_1+a_2)
	\end{equation*}
	\begin{equation*}
		\pi((a_1,q_1)(a_2,q_2))=\pi(-,q_1q_2) = q_1q_2 = \pi(a_1,q_1)\pi(a_2,q_2)
	\end{equation*}
	\begin{equation*}
		\pi(i(a)) = \pi(a,1) = 1
	\end{equation*}
	\end{demostracion}
\end{proposicion}

\begin{proposicion}\label{prop:res2}
	Sea $E$ una extensión de $Q$ por $A$, $s_1$ y $s_2$ dos secciones normalizadas de $Q$ a $E$ y $c_1,c_2$ los cociclos asociados a $s_1$ y $s_2$. Entonces, $c_1$ y $c_2$ se diferencian en un $2$-coborde normalizado y la extensión $E$ determina la clase $[c_1]\in H_\varphi^2(Q,A)$. 
	
	\begin{demostracion}
		La diferencia de $s_1$ y $s_2$ define una función $\homo e Q {A}$ por $s_2(q) = i(e(q))s_1(q)$. Cambiando la sección $s_2$ por $s_1$ en \eqref{eq:factorsystem}
		\begin{align*}
			i(c_2(g,h))s_2(gh) 
			&= s_2(g)s_2(h) \\ 
			&= i(e(g))s_1(g)i(e(h))s_1(h)\\ 
			&= i(e(g) + g \cdot e(h)) s_1(g)s_1(h) \\ 
			&= i(e(g) + g \cdot e(h) + c_1(g,h))s_1(gh) \\
			&= i(e(g) + g \cdot e(h) - e(gh) + c_1(g,h))s_2(gh)
		\end{align*}
		obtenemos que la diferencia de $c_2$ y $c_1$ es el coborde de $e$
		\begin{equation} 
			(c_2-c_1)(g,h) = g \cdot e(h) - e(gh) + e(g)
		\end{equation}
		
		Además, por ser $c_1$ y $c_2$ cociclos normalizados
		\begin{equation*}
			(c_2-c_1)(1,1) = e(1) = 0
		\end{equation*}
		$c_2-c_1$ es un coborde normalizado.
	\end{demostracion}
\end{proposicion}

%En \ref{prop:equivcoc} hemos probado que la extens
Hemos visto en \ref{prop:res1} que un $2$-cociclo normalizado da lugar a una extensión de $Q$ por $A$ y en \ref{prop:res2} que dos extensiones son equivalentes cuando los $2$-cociclos normalizados son cohomólogos. La elección de cociclos normalizados es valida ya que como se ha visto en la Proposición \ref{prop:normcoc}, todo $2$-cociclo es cohomólogo a uno normalizado. En el Apéndice $\ref{apen:norm}$ se da la operación de grupo de $E_c$ para un cociclo no normalizado.

Queda demostrado el siguiente teorema. % Como se dice cuando algo no pierde generalidad

\begin{teorema}\label{h2}
	Sea $A$ un $Q$-módulo dado por una acción $\homo \varphi Q {Aut(A)}$. Entonces, las extensiones salvo equivalencia de $Q$ por $A$ están en correspondencia uno a uno con los elementos del segundo grupo de cohomología.
	\begin{equation*}
		\Ext_{\varphi}(Q,A)\cong H^2_{\varphi}(Q,A)
	\end{equation*}
\end{teorema}

\begin{observacion}\label{obs:split}
	El producto semidirecto se corresponde con el elemento neutro de $H^2(Q,A)$.
	\begin{demostracion}
		Por el Teorema \ref{splitext}, si una sección es un homomorfismo, el $2$-cociclo asociado a ésta es trivial.
	\end{demostracion}
\end{observacion}

% hablar de la estructura de grupo de H2 (suma de cociclos es cociclo): de esto hablo en la definicion de cociclo
% mencionar que es no vacio
\begin{proposicion}\label{extsum}
	Sean $[E_1],[E_2]\in Ext(Q,A)$ dos extensiones y $[c_1],[c_2]\in H^2(Q,A)$ sus cociclos asociados, podemos definir la suma $[E_1] + [E_2]$ como la clase de extensiones equivalentes asociada a $[c1+c2]\in H^2(Q,A)$. Es decir, $Ext_\varphi(Q,A)$ tiene una estructura de grupo abeliano heredada de $H^2(Q,A)$.
\end{proposicion}

En el Apéndice \ref{baersum} se da otra forma de construir la suma anterior sin el uso de cociclos.

\section{Teorema de Schur-Zassenhaus. Caso abeliano}



\begin{teorema}\label{thmschurab} Sea $G$ un grupo finito y sea $N \norm G$ abeliano, $\ord{N}=n$ y $\ord{G:N}=m$ con $\mcd(n,m)=1$. Entonces $G$ contiene subgrupos de orden $m$ y dos cualesquiera son conjugados.
	\begin{demostracion}
%		Tomamos una sección $s$ y su $2$-cociclo asociado $c$. 
%		\begin{equation}
%			c(q_1,q_2)\cdot q_3  - c(q_1,q_2q_3)+ c(q_1q_2,q_3) -c(q_2,q_3) =0
%		\end{equation}
%		Definimos la función ${\displaystyle \tilde c(x) = \sum_{t\in Q} c(t,x)}$ y hacemos la suma en $q_1\in Q$
%		\begin{equation}\label{coccob}
%			m c(q_2,q_3) = \tilde c(q_2)\cdot q_3 + \tilde c(q_3) - \tilde c(q_2q_3)
%		\end{equation}
%		
%		Como $\mcd(m,n)=1$, existe un $k\in \Z$ tal que $km\equiv 1 \mod n$. Multiplicando por $k$ llegamos a 
%		\begin{equation}
%			c(q_2,q_3) = k \tilde c(q_2)\cdot q_3 + k \tilde c(q_3) - k \tilde c(q_2q_3)
%		\end{equation}
%		Por tanto, $c$ verifica la ecuación de un coborde para la función $k \tilde c$ y $H^{2}(Q,N)$ es trivial. Por la Observación \ref{obs:split}, $G$ es un producto semidirecto de $Q$ por $N$.
%		
%		Para ver que dos subgrupos cualesquiera son conjugados, cogemos $s$ una escisión de $G$ y $a$ su $1$-cociclo asociado. 
%		\begin{equation}
%			a(q_1q_2)=a(q_1)\cdot q_2 + a(q_2)
%		\end{equation}
%		Haciendo la suma en $q_1\in Q$ y multiplicando por $k$
%		\begin{equation}
%			a(q_2) = k\tilde a - k\tilde a \cdot q_2
%		\end{equation}
%		vemos que $a$ verifica la ecuación de un $1$-coborde para la función $k\tilde a$ y $H^1(Q,N)$ es trivial. Por el Teorema $\ref{h1}$, todas las escisiones son $N$-conjugadas y por tanto son conjugadas.

		Por el Teorema \ref{thm:trivialH}, $H^2(Q,N)$ es trivial y por la Observación \ref{obs:split}, $G$ es un producto semidirecto de $Q$ por $N$. Esto prueba la parte de existencia.
		
		Para la conjugación, cualquier subgrupo $\tilde Q$ de $G$ de orden $m$, al ser $\mcd(n,m)=1$, tiene intersección trivial con $N$ y la proyección sobre $G/N$ es inyectiva. Por tanto, $\tilde Q$ es una escisión de $G$. Por el mismo teorema, $H^1(Q,N)$ es trivial y por el Teorema \ref{h1} todas las escisiones de la extensión $G$ son $N$-conjugadas y por tanto conjugadas. 
	\end{demostracion}
\end{teorema}
% !TeX root=../tfg2.tex

\section{Extensiones centrales y abelianas}

En esta sección introduciremos las extensiones centrales y daremos una caracterización de cúando el grupo intermedio de una extensión es abeliano.

\begin{definicion}
	Decimos que una extension $\extension i \pi A E Q$ es una extensión central cuando $i(A)\leq Z(E)$. 
	
	Diremos que es abeliana cuando el grupo central $E$ de la extensión es abeliano. No debe confundirse con una extensión con núcleo abeliano.
\end{definicion}

\begin{proposicion}
	Una extensión $\extension i \pi A E Q$ es central si y solo si la acción es trivial.
	\begin{demostracion}
		Sea una sección $\homo s Q E$, la acción de $Q$ en $A$ viene dada por conjugación $q\cdot a = \,^{s(q)}i(a)$ que es trivial si y solo si $\,^{s(q)}i(a) = i(a)$. Como la acción de $q$ no depende de la sección $s$, $i(a)$ está en el centro y la extensión es central.
	\end{demostracion}
\end{proposicion}

A partir de ahora consideraremos que $Q$ es abeliano y la acción de $Q$ en $A$ es trivial.

\begin{definicion}
	Sea $c\in Z^2(Q,A)$ un $2$-cociclo. Decimos que $c$ es un $2$-cociclo simétrico cuando para todo $q_1,q_2\in Q$
	\begin{equation*}
		c(q_1,q_2)=c(q_2,q_1)
	\end{equation*}
	
	Claramente la suma de dos cociclos simétricos también es simétrica. Denotaremos a los subgrupos de $2$-cociclos y $2$-cobordes simétricos como $Z^2(Q,A)_s$ y $B^2(Q,A)_s$ respectivamente.
\end{definicion}


\begin{teorema}\label{thm:abext}
	Una extensión $\extension i \pi A E Q$ es abeliana si y solo si es una extensión central, $Q$ es abeliano y todo cociclo asociado a la extensión es simétrico.
	\begin{demostracion}
		Supongamos que $E$ es abeliano. Entonces la extensión es central y cualquier subgrupo y cociente de $E$ es abeliano, en particular $Q$. Sean $q_1,q_2\in Q$, entonces
		\begin{equation*}
			c(q_1,q_2)=s(q_1)s(q_2)s(q_1q_2)^{-1} = s(q_2)s(q_1)s(q_2q_1)^{-1} = c(q_2,q_1) % enverdad falta i
		\end{equation*}
		
		Supongamos ahora que $c$ es un cociclo simétrico y $A$ y $Q$ son abelianos. Entonces la operación en $E_c$ descrita en \ref{extop} viene dada por 
		\begin{align*}
			(a_1,q_1)(a_2,q_2) 
			& = (a_1+a_2 + c(q_1,q_2),q_1q_2) \\
			& = (a_2+a_1 + c(q_2,q_1),q_2q_1) \\
			& = (a_2,q_2)(a_1,q_1)
		\end{align*}
		y $E$ es abeliano.
	\end{demostracion}
\end{teorema}

\begin{proposicion}
		Todo $2$-coborde es simétrico. Por tanto, $B^2(Q,A)_s=B^2(Q,A)$.
	\begin{demostracion}
		Sea $\phi\in C^1(Q,A)$ una $1$-cocadena. Entonces
		\begin{equation*}
			(\partial^1\phi)(q_1,q_2) = \phi(q_2) - \phi(q_1q_2) + \phi(q_1) = (\partial^1\phi)(q_2,q_1).
		\end{equation*}
	\end{demostracion}
\end{proposicion}

\begin{corolario}
	Sea $c\in Z^2(Q,A)_s$ un $2$-cociclo simétrico. Entonces, $c'\in [c]$ es también un $2$-cociclo simétrico.
	\begin{demostracion}
		$c' = c+b$ para algún $b\in B^2(Q,A)=B^2(Q,A)_s\leq Z^2(Q,A)_s$. $Z^2(Q,A)_s$ es un subgrupo y por tanto $c'$ es simétrico.
	\end{demostracion}
\end{corolario}

Por tanto, las clases de cociclos simétricos están bien definidas y tiene sentido hablar de $H^2(Q,A)_s$, el subgrupo de clases de cociclos simétricos. Por lo visto en los teoremas \ref{h2} y \ref{thm:abext}, las extensiones abelianas están clasificadas  salvo equivalencia por $H^2(Q,A)_s$.

\begin{teorema}
	Sea $Q$ abeliano y $A$ un $Q$-módulo trivial. Entonces, las extensiones abelianas salvo equivalencia están en biyección con los elementos de $H^2(Q,A)_s$.
	\begin{equation*}
		\ExtFunctor_{Ab}(Q,A) \cong H^2(Q,A)_s
	\end{equation*}
\end{teorema}
\chapter{Teoremas de Hall}

\begin{definicion} 
	\pi subgrupos
	O_\pi subgrupos
\end{definicion}
% !TeX root=../tfg2.tex

\chapter{Transfer}

Sea $G$ un grupo y $H$ un subgrupo de $G$ de índice finito $n$. Dado un homomorfismo $\homo \theta H A$ sobre un grupo abeliano, el transfer permite definir un homomorfismo de $G$ sobre $A$. %El estudio de homomorfismos sobre grupos abelianos es de gran interés 

Para ello tomamos $\transversal t n$ un conjunto de representantes de las coclases a derecha de $H$ en $G$. La multiplicación a derecha de las coclases $Ht_i$ por elementos de $G$ nos define una acción
\begin{align}\label{minitransfer}
	G/H\times G &\to {G/H} \nonumber\\
	(Ht_i,g) &\mapsto Ht_ig = Ht_{(i)g}
\end{align}
donde $i\mapsto (i)g$ es una permutación de $S_n$. Se tiene entonces que $t_igt_{(i)g}^{-1}\in H$ para todo $g\in G$ y esto nos da una forma natural de definir una función sobre $G$ haciendo el producto siguiente que llamaremos pre-transfer de $G$ a $H$.
\begin{equation}
	{\pretransfer G H}(g)= \prod_{i=1}^n t_igt_{(i)g}
\end{equation}
El pre-transfer no es en general independiente de la transversal ni un homomorfismo de grupos. Estos dos problemas se pueden solucionar componiendo el pre-transfer con un homomorfismo sobre un grupo abeliano. %De hecho, haciendo el producto en distinto orden no está bien definido.

\section{Homomorfismo del transfer}

\begin{definicion}
	Sea $G$ un grupo, $H$ un subgrupo de $G$ de índice finito $n$, $\{t_1,\ldots, t_n\}$ una transversal a derecha de $H$ a $G$ y $\homo \theta H A$ un homomorfismo sobre un grupo abeliano $A$. El transfer de $\theta$ se define como la composición de $\theta$ con el pre-transfer $\pretransfer G H$
	\begin{align}
		{\transfer G H} \colon 	G &\to A \nonumber \\
							g &\mapsto \prod_{i=1}^n \theta\left(t_igt_{(i)g}^{-1}\right)
	\end{align}
	
	Dado que $A$ es abeliano, el orden de los factores no importa y $\transfer G H$ está bien definido.  % sacar fuera, poner arriba? nose
\end{definicion}

%En principio, la transferencia definida parece depender

% independencia de la transversal
\begin{proposicion}
	El transfer $\homo {\transfer G H} G A$ es un homomorfismo que no depende de la elección de la transversal.
	\begin{demostracion}
		Sean  $\transversal t n$ y  $\transversal s n$ dos transversales tales que $s_i = h_it_i$ con $h_i\in H$. Entonces, para todo $g\in G$
		\begin{align*}
			\prod_{i=1}^n \theta\left(s_igs_{(i)g}^{-1}\right) 
			&= \prod_{i=1}^n \theta\left(h_it_igt_{(i)g}^{-1}h_{(i)g}^{-1}\right) \\
			&= \prod_{i=1}^n \theta\left(t_igt_{(i)g}^{-1}\right)\theta\left(h_ih_{(i)g}^{-1}\right)\\
			&=\prod_{i=1}^n \theta\left(t_igt_{(i)g}^{-1}\right) 
		\end{align*}
		% observar que las h se cancelan porque (i)g es una permutacion de los indices

		Para ver que es un homomorfismo, cogemos $x,y\in G$
		\begin{align*}
			\theta^*(xy) &= \prod_{i=1}^n \theta\left(t_ixyt_{(i)xy}^{-1}\right) \\
			&=  \prod_{i=1}^n \theta\left((t_ixt_{(i)x}^{-1})(t_{(i)x}yt_{(i)xy}^{-1})\right) \\
			&= \prod_{i=1}^n \theta\left(t_ixt_{(i)x}^{-1}\right)\theta\left(t_{(i)x}yt_{(i)xy}^{-1}\right)\\
			&= \theta^*(x)\theta^*(y)
		\end{align*}
	\end{demostracion}
\end{proposicion}

\begin{observacion}
	Sea $G$ un grupo y $\homo \tau G A$ un homomorfismo sobre un grupo abeliano $A$. Entonces, $G' \leq \Ker(\tau)$ y por tanto $\tau$ factoriza de forma única a través de la abelianización de $G$. Es decir, existe un único homomorfismo $\homo {\tilde\tau} {\Ab G} A $ que hace al siguiente diagrama conmutativo.
	% https://q.uiver.app/?q=WzAsMyxbMCwwLCJHIl0sWzIsMCwiQSJdLFswLDIsIkdee1xcdGV4dHthYn19Il0sWzAsMSwiXFx0aGV0YSJdLFswLDIsIlxccGkiLDJdLFsyLDEsIlxcZXhpc3RzIVxcdGlsZGVcXHRoZXRhIiwyXV0=
\[\begin{tikzcd}
	G && A \\
	\\
	{\Ab G}
	\arrow["\tau", from=1-1, to=1-3]
	\arrow["\AbFunctor"', from=1-1, to=3-1]
	\arrow["{\exists!\tilde\tau}"', from=3-1, to=1-3]
\end{tikzcd}\]
\end{observacion}

\section{Cálculo del transfer}

En ocasiones, dado un $x\in G$ se puede calcular el transfer de manera eficiente con una buena elección de los representantes de las coclases. El elemento $x$ actúa sobre las coclases $\transversal {Ht} n$ por multiplicación a derecha permutándolas. Las órbitas bajo esta acción tienen la siguiente forma

\begin{equation}
	\left\{Hs_i,\ldots,Hs_i^{l_i-1}\right\}
\end{equation}
con $l_i$ el menor natural tal que $Hs_ix^{l_i} = Hs_i$.

Se tiene entonces que los elementos $s_ix^j$ con $i=1,\ldots,k$, $j=0,\ldots,l_i-1$ y $k$ el número de órbitas forman una transversal de $H$ a $G$. Además, por el teorema de la Órbita-Estabilizador, se tiene que $l_i \bigm| \ord{G}$.

% por favor quitar este trabalenguas
\begin{lema}
	Si $\transversal s k$ son representantes de las coclases de las órbitas de la acción de $x$, entonces
	\begin{equation}
		 \tau(x) = \prod_{i=1}^k \theta(\,^{s_i}x^{l_i})
	\end{equation}
	\begin{demostracion}
		Miramos la contribución de la órbita de $s_i$ al transfer de $x$.
		\begin{align*}
			\prod_{j=0}^{l_i-1}\theta(s_ix^j x (s_ix^j)^{-1}) = \prod_{j=0}^{l_i-1}\theta(s_ixs_i^{-1}) = \theta(s_ixs_i^{-1})^{l_i} = \theta(\,^{s_i}x^{l_i})
		\end{align*}
		
		Multiplicando las contribuciones de todas las órbitas se llega al resultado.
	\end{demostracion}
\end{lema}

\section{Transfer a un subgrupo}

Nos centramos a continuación en el caso en que $\theta$ es la abelianización del subgrupo $H$, es decir, $\homo \theta H {\Ab H}$. En este caso, denotaremos al transfer por $\homo {\transfer G H} G {\Ab H}$ y diremos que $\transfer G H$ es el transfer de $G$ a $H$. % ta bn?


\begin{ejemplo}
	Si $G$ es abeliano y $H\leq G$ con $\ord{G:H} = n < \infty$, el transfer de $G$ a $H$ es $g \mapsto g^n$.
	\begin{demostracion}
		Sea $\transversal t n$ una transversal de $H$. Entonces para todo $g\in G$
		\begin{equation*}
			\prod_{i=1}^n t_ig(t_{(i)g})^{-1} = g^n \prod_{i=1}^n t_i(t_{(i)g})^{-1} = g^n
		\end{equation*}
	\end{demostracion}
\end{ejemplo}

\begin{proposicion}
	Sea $G$ un grupo y $H\leq K \leq G$ tal que $\ord{G:H}$ es finito. Entonces, el transfer de $G$ a $H$ es $\transfer G H = \ptransfer K H \circ \transfer G K = \ptransfer K H \circ \ptransfer G K \circ \AbFunctor$.
	
	% POR FAVOR LAS CURSIVAS ESTAN RARAS
	
	% https://q.uiver.app/?q=WzAsNixbMCwwLCJHIl0sWzIsMCwiSyJdLFs0LDAsIkgiXSxbMCwyLCJHXntcXHRleHR7YWJ9fSJdLFsyLDIsIktee1xcdGV4dHthYn19Il0sWzQsMiwiSF57XFx0ZXh0e2FifX0iXSxbMCwxLCJQX3tHL0t9Il0sWzEsMiwiUF97Sy9IfSJdLFswLDMsIkFiIl0sWzEsNCwiQWIiXSxbMiw1LCJBYiJdLFszLDQsIlxcdGlsZGVcXHRhdV97Ry9LfSJdLFs0LDUsIlxcdGlsZGVcXHRhdV97Sy9IfSJdLFswLDQsIlxcdGF1X3tHL0t9Il0sWzEsNSwiXFx0YXVfe0svSH0iXV0=
\[\begin{tikzcd}
	G && K && H \\
	\\
	{\Ab G} && {\Ab K} && {\Ab H}
	\arrow["{\pretransfer G K}", from=1-1, to=1-3]
	\arrow["{\pretransfer K H}", from=1-3, to=1-5]
	\arrow["\AbFunctor", from=1-1, to=3-1]
	\arrow["\AbFunctor", from=1-3, to=3-3]
	\arrow["\AbFunctor", from=1-5, to=3-5]
	\arrow["{\ptransfer G K}", from=3-1, to=3-3]
	\arrow["{\ptransfer K H}", from=3-3, to=3-5]
	\arrow["{\transfer G K}", from=1-1, to=3-3]
	\arrow["{\transfer K H}", from=1-3, to=3-5]
\end{tikzcd}\]
	\begin{demostracion}
		Sean $\transversal g n$ y $\transversal k m$ transversales a derecha de $K$ a $G$ y de $H$ a $K$ respectivamente. Entonces 
		\begin{equation*}
			G = \bigsqcup_{i=1}^{n} K g_i = \bigsqcup_{i=1}^n\bigsqcup_{j=1}^m Hk_jg_i
		\end{equation*}
		y $\{k_jg_i \ | \ i=1,\ldots,n, \ j=1,\ldots m\}$ es una transversal de $H$ a $G$.
		
		Miramos como actúa un elemento $g$ en la coclase $Hk_jg_i$. 
		\begin{equation}\label{eq:transact}
			Hk_jg_ig = Hk_j g_ig(g_{(i)g}^{-1}g_{(i)g}) = Hk_{j} (g_igg_{(i)g}^{-1})g_{(i)g} = Hk_{(j)g_igg_{(i)g}^{-1}}g_{(i)g} % problema: el g_ig no se puede aplicar a Hk_j salvo que esté en K
		\end{equation}
		
		Evaluando $\ptransfer K H \circ \transfer G K$ en $g$ y utilizando \eqref{eq:transact}		
		\begin{align*}
		{\ptransfer K H} \circ {\transfer G K (g)} &= {\ptransfer K H}\left(\prod_{i=1}^n K' g_i g g_{(i)g}^{-1}\right) \\
			&= \prod_{j=1}^m\prod_{i=1}^n H'k_j g_i g g_{(i)g}^{-1} k_{(j)g_i g g_{(i)g}^{-1}}^{-1}  \\
			&= \prod_{j=1}^m\prod_{i=1}^n H'k_j g_i g  (k_{(j)g_i g g_{(i)g}^{-1}} g_{(i)g})^{-1} \\
			%&= \prod_{j=1}^m\prod_{i=1}^n H'k_j g_i g  (k_{(j)g_{(i)g}})^{-1} \\
			%&= \prod_{j=1}^m\prod_{i=1}^n H'k_j g_i g  (k_{j}g_{(i)g})^{-1} \\
			%&= \prod_{j=1}^m\prod_{i=1}^n H'k_j g_i g  (k_{j}g_{i})_g^{-1} \\
			& = {\transfer G H}(g)
		\end{align*}
	\end{demostracion}
\end{proposicion}

\begin{proposicion}
	Sea $G$ un grupo y $H\leq G$ de índice finito $n$. Sea $\transfer G H$ el transfer de $G$ a $H$. Entonces $(i\circ {\ptransfer G H})(gG') = g^nG'$ donde $i$ es la inclusión de $\Ab H$ en $\Ab G$ definida por $i(hH') = hG'$.

% https://q.uiver.app/?q=WzAsMyxbMCwwLCJHIl0sWzAsMiwiXFxBYiBHIl0sWzIsMiwiXFxBYiBIIl0sWzEsMiwiXFxwdHJhbnNmZXIgRyBIIiwyLHsib2Zmc2V0IjoxfV0sWzIsMSwiaSIsMix7Im9mZnNldCI6MX1dLFswLDIsIlxcdHJhbnNmZXIgRyBIIl0sWzAsMSwiXFxBYkZ1bmN0b3IiLDJdXQ==
\[\begin{tikzcd}
	G \\
	\\
	{\Ab G} && {\Ab H}
	\arrow["{\ptransfer G H}"', shift right=1, from=3-1, to=3-3]
	\arrow["i"', shift right=1, from=3-3, to=3-1]
	\arrow["{\transfer G H}", from=1-1, to=3-3]
	\arrow["\AbFunctor"', from=1-1, to=3-1]
\end{tikzcd}\]
	
	\begin{demostracion}
		Sea $\transversal t n$ una transversal de $H$ y $\transfer G H$ el transfer de $G$ a $H$. %Dado $h\in H$, $t_iht_{(i)h}^{-1} \in H$ %(g) = \prod_{i=1}^n  t_igt_{(i)g}^{-1}H'
		
		\begin{align*}
		(i\circ {\ptransfer G H})(gG') 
		&= i(\prod_{i=1}^n  t_igt_{(i)g}^{-1}H') \\
		&= \prod_{i=1}^n  i(t_igt_{(i)g}^{-1}H') \\
		&= \prod_{i=1}^n  t_igt_{(i)g}^{-1}G' \\
		&= g^n\prod_{i=1}^n  t_it_{(i)g}^{-1}G'=g^nG'
		\end{align*}
	\end{demostracion} 
\end{proposicion}

%\begin{proposicion}
%	Sea $G$ un grupo y $H$ un subgrupo abeliano de $G$ de índice finito $n$. Entonces, el transfer de $G$ a $H$ es $\theta(g)=g^n$
%	\begin{demostracion}
%		% https://q.uiver.app/?q=WzAsMyxbMCwwLCJHIl0sWzIsMCwiSCJdLFswLDIsIlxcQWJ7R30iXSxbMCwxLCJcXHRoZXRhIl0sWzAsMiwiXFxwaSIsMl0sWzIsMSwiXFx0aWxkZVxcdGhldGEiLDAseyJvZmZzZXQiOi0xfV0sWzEsMiwiaSIsMCx7Im9mZnNldCI6LTF9XV0=
%		\[\begin{tikzcd}
%			G && H \\
%			\\
%			{\Ab{G}}
%			\arrow["\theta", from=1-1, to=1-3]
%			\arrow["\pi"', from=1-1, to=3-1]
%			\arrow["\tilde\theta", shift left=1, from=3-1, to=1-3]
%			\arrow["i", shift left=1, from=1-3, to=3-1]
%		\end{tikzcd}\]
%		Usando que $\theta = \tilde\theta \circ \pi$, $(i\circ\theta)(g) = $
%	\end{demostracion}
%\end{proposicion}

\subsection{Transfer al centro}

\begin{proposicion}
	Sea $G$ un grupo y $H$ un subgrupo central de $G$ de índice $\ord{G:H}=n$. Entonces, el transfer de $G$ a $H$ es $\transfer G H (g) = g^{n}$
	\begin{demostracion}
		$\,^{s_i}g^{l_i}\in H\leq Z(G)$ y conjugando por $s_i$ se llega a $g^{l_i}\in Z(G)$. Finalmente 
		\begin{equation}
			\prod_{i=1}^k \,^{s_i}g^{l_i} = \prod_{i=1}^k g^{l_i} = g^n
		\end{equation}
	\end{demostracion}
\end{proposicion}

%Notese que en general, la función de elevar cada elemento a una potencia no es un homomorfismo salvo que $G$ sea abeliano. % blablabla soltar rollo sobre lo remarcable que es que elevar al indice del centro sí lo sea

\begin{proposicion}\label{prop:quotfg} % poner encima del teorema de schur
	Sea $G$ un grupo finitamente generado y $H$ un subgrupo de $G$ de índice finito $n$. Entonces $H$ es finitamente generado.
	\begin{demostracion}
		Sea $X$ un sistema generador finito de $G$ y $\{1=t_1,\ldots,t_n\}$ una transversal de $H$ a $G$. Definimos la función $\tau_{i}(g) = t_igt_{(i)g}^{-1}\in H$ como en \eqref{minitransfer}, de esta forma $t_i g = \tau_{i}(g)t_{(i)g}$.
		Sea $h\in H$, entonces $h=x_1\cdots x_k$ donde cada $x_i\in X\cup X^{-1}$. Demostramos por inducción en $k$ que $h$ es finitamente generado en $H$ por $Y=\{\tau_i(x) \ | \ x\in X, \ i=1,\ldots,n\}$. Llamamos $\tilde X = X \cup \transversal t n$.
		
		Para el caso base, $k=1$, tenemos $h = x_1\in H$ y $\tau_1(x_1) = t_1x_1t_{(1)x_1} = x_1$ observando que $t_{(1)x_1} = t_1$ ya que $x_1\in H$, por lo que $h$ está generado por $\tau_1(x_1)$.
		
		Supongamos ahora que todo elemento de $H$ generado por menos de $k$ elementos de $\tilde X\cup\tilde X^{-1}$ es finitamente generado en $H$ por elementos de $Y$. 
		\begin{align*}
			h = x_1\cdots x_k = t_1 x_1\cdots x_k &= \tau_{1}(x_1)t_{(1)x_1} x_2\cdots x_k \\
			&= \tau_{1}(x_1) \tau_{(1)x_1}(x_2)t_{(1)x_1x_2} x_3\cdots x_k
		\end{align*}
		
		Por hipótesis de inducción, $t_{(1)x_1x_2} x_3\cdots x_k$ es finitamente generado por elementos de $Y$ y, por tanto, $h$ también.
	
%		\begin{equation*}
%			h = t_1x_1\cdots x_k = \tau_{1}(x_1)t_{(1)x_1}x_2\cdots x_k = \prod_{i=1}^k\left[ \tau_{(1)\prod_{j=1}^{i}x_j}\Big(\prod_{j=1}^{i}x_j\Big)\right]t_{(1)h}
%		\end{equation*}
%		
%		Como $t_1 = 1$ y $h\in H$, $t_{(1)h} = t_1 = 1$ y $H$ está generado por $\tau_i(x)$ para $i=\{1,\ldots,n\}$ y $x\in X\cup X^{-1}$.
	\end{demostracion}
\end{proposicion}

\begin{corolario}[Schur]
	Sea $G$ un grupo y $Z(G)$ de índice finito $n$, entonces $G'$ es finito y $G'^n = \{1\}$.
	\begin{demostracion}
		Sea $C=Z(G)$ y $\transversal {Cg} n$ las coclases a derecha de $C$ en $G$. Sean $c_ig_i,c_jg_j\in G$, entonces $[c_ig_i,c_j,g_j] = [g_i,g_j]$ y $G'$ está finitamente generado. Por el Segundo Teorema de Isomorfía $\frac{G'}{G'\cap C} = \frac{G'C}{C}$ es finito y por la Proposición \ref{prop:quotfg}, $G'\cap C$ está finitamente generado.
		El transfer de $G$ a $C$ viene dado por $x\mapsto x^n$ y como $G'$ está contenido en el kernel, $G'^n = \{1\}$, en particular, $(G'\cap C)^n = \{1\}$. Por el teorema de clasificación de grupos abelianos finitamente generados, $G'\cap C$ es finito y como $\ord{G':G'\cap C}$ es finito se deduce que $G'$ es finito.
	\end{demostracion}
\end{corolario}


%\subsection{Transfer a un subgrupo de Sylow}

\apendices
% !TeX root=../tfg2.tex
\chapter{Ejercicios resueltos}

\begin{ejercicio}
	Estudiar las extensiones salvo equivalencia de $\Z_p$ por $\Z_p$. % y ver que $H^2(\Z_p,\Z_p) \cong \Z_p$.
	\begin{solucion}
		Los grupos de orden $p^2$ son abelianos y por tanto la extensión es central y tiene acción trivial.
		Observamos que si $[c]\in H^2(\Z_p,\Z_p)$ $p[c] = [0]$, por lo que $H^2(\Z_p,\Z_p)$ es un $p$-grupo abeliano de exponente $p$, es decir, isomorfo a $\Z_p^k$ para algún $k\in \N$.
		
		Contando el número de extensiones veremos que $0<k<2$ y por tanto, el número de extensiones equivalentes debe ser $p$.
		
		\textit{(Extensiones isomorfas a $\Z_p\times \Z_p$).}
		\begin{equation*}
			\extension i \pi {\Z_p} {\Z_p\times \Z_p} {\Z_p}
		\end{equation*}
		
		La proyección $\homo \pi {\Z_p\times \Z_p} {\Z_p}$ viene dada por las imágenes de los generadores $e_1=(1,0)$ y $e_2=(0,1)$. Como $\pi$ es sobreyectiva, $\pi(e_1)$ o $\pi(e_2)$ es distinto de la identidad, supongamos que $e_1\notin \Ker(\pi)$. Entonces $\langle e_1 \rangle$ es una escisión de $\pi$ y la extensión es trivial, por lo que se corresponde con el elemento neutro de $H^2$ y es única salvo equivalencia.
		
		\textit{(Extensiones isomorfas a $\Z_{p^2}$).}
		\begin{equation*}
			\extension {i_n} {\pi_n} {\Z_p} {\Z_{p^2}} {\Z_p}
		\end{equation*}
		
		$\Z_{p^2}$ es cíclico y $\Z_p$ debe incluirse en el único subgrupo de $\Z_{p^2}$ de orden $p$. 
		
		Hay $p-1$ formas de incluir $\Z_p$ en $\langle p \rangle \leq \Z_{p^2}$ que se corresponden con los distintos automorfismos de $\langle p \rangle$. Las inclusiones $i_n$ vienen dadas por $i_n(x) = pnx \mod p^2$ donde $n=1,\ldots,p-1$.
		
		Las proyecciones $\pi_n$ vienen dadas por la imagen del $1\in \Z_{p^2}$, que puede mandarse a cualquier elemento no trivial de $\Z_p$ y por tanto hay $p-1$ proyecciones distintas definidas por $\pi_n(1) = n \mod p$ para $n=1,\ldots, p-1$.
		
		En total, hay $(p-1)^2$ formas de componer las $i_n$ y $\pi_n$. Sumando la extensión trivial dan un total de $p^2-2p+2$, que es menor que $p^2$ y mayor que $1$ para todo primo $p$. 
		
		\textit{(Representantes de las extensiones y cociclos asociados).} 
		La primera extensión es única salvo equivalencia y podemos tomar la inclusión en la primera coordenada $i(x) = (x,0)$ y la proyección en la segunda $\pi(x,y) = y$. El cociclo asociado $c_0$ es trivial ya que la extension escinde.
		
		Para la segunda extensión podemos fijar la inclusión $i_1$ para simplificar cálculos. Sean $n,m\in \{1,\ldots,p-1\}$ distintos y tomemos las proyecciones $\pi_m$ y $\pi_n$. Supongamos que existe un homomorfismo $f$ que haga al siguiente diagrama conmutativo
		\[\begin{tikzcd}
			1 & {\Z_p} & {\Z_{p^2}} & {\Z_p} & 1 \\
			1 & {\Z_p} & {\Z_{p^2}} & {\Z_p} & 1
			\arrow[from=1-1, to=1-2]
			\arrow["{i_1}", from=1-2, to=1-3]
			\arrow["{\pi_m}", from=1-3, to=1-4]
			\arrow[from=1-4, to=1-5]
			\arrow[from=2-1, to=2-2]
			\arrow["{i_1}", from=2-2, to=2-3]
			\arrow["{\pi_n}", from=2-3, to=2-4]
			\arrow[from=2-4, to=2-5]
			\arrow[shift left=1, no head, from=1-2, to=2-2]
			\arrow[shift left=1, no head, from=2-2, to=1-2]
			\arrow["f"', from=1-3, to=2-3]
			\arrow[shift left=1, no head, from=1-4, to=2-4]
			\arrow[shift left=1, no head, from=2-4, to=1-4]
		\end{tikzcd}\]

		Por un lado, $f(i_1(1)) = i_1(1) = p$ y por tanto $f(p)=p$. Por otro lado, $\pi_n(f(1)) = \pi_m(1) = m$ y $f(1)\in \pi_n^{-1}(m) = n^{-1}m + \langle p\rangle$. Esto es absurdo ya que $f(p) = pf(1) = pn^{-1}m \neq p$ pero $n\neq m$. Esto prueba que las extensiones $(\Z_{p^2},i_1,\pi_n)$ son inequivalentes para todo $n=1,\ldots,p-1$.
		
		Una sección de $\pi_n$ es $s_n$ definida por $s_n(x) = n^{-1}x \mod p = \overline{n^{-1}x}\in \{0,\ldots,p-1\}$. En efecto, $\pi_n(s_n(x)) = \pi_n(\overline{n^{-1}x}) = nn^{-1}x = x$ para $x=0,\ldots,p-1$.
		El cociclo asociado a esta sección es 
		\begin{equation*}
			c_n(x,y) = i_1^{-1}(s_n(x)+s_n(y)-s_n(x+y)) = \frac{\overline{n^{-1}x}+\overline{n^{-1}y}-\overline{n^{-1}(x+y)}}{p} %= i_1^{-1}\left(\overline{n^{-1}x}+\overline{n^{-1}y}-\overline{n^{-1}(x+y)}\right) 
		\end{equation*}
		
		Como $H^2$ es cíclico, podemos tomar $c_1$, cuya expresión es sencilla, y generar el resto de los cociclos con él
		
		\[
		    c_1(x,y) = \frac{\overline{x} + \overline{y} - \overline{x+y}}{p} =  \begin{cases}
		        0 & \text{si } x+y < p\\
		        1 & \text{si } x+y \geq p
		        \end{cases}
		\]
		% comprobar si se corresponden con los pi_n o estan permutados por un n^-1 o algo asi
		\[
		    c_n(x,y) = nc_1(x,y) =  \begin{cases}
		        0 & \text{si } x+y < p\\
		        n & \text{si } x+y \geq p
		        \end{cases}
		\]
	\end{solucion}
\end{ejercicio}

%% !TeX root=../tfg2.tex
\chapter{Normalización de un cociclo}\label{apen:norm}

Tomando una sección $s$ de la extensión $\extension i \pi A E Q$ y usando la Proposición \ref{prop:normcoc} si llamamos a $x = c(1,1)$, podemos normalizar el cociclo sumandole el $2$-coborde $b(q_1,q_2) = -q_1\cdot x$.

\begin{equation}
	\tilde c(q_1,q_2) = c(q_1,q_2) - q_1\cdot c(1,1) = s(q_1)s(q_2)s(q_1q_2)^{-1} - q_1\cdot s(1)
\end{equation}

\begin{align*}
	i(a_1)s(q_1)i(a_2)s(q_2) &= i(a_1)i(a_2)^{q_1}s(q_1)s(q_2) \\
	&= i(a_1 + q_1\cdot a_2 + c(q_1,q_2) - q_1\cdot x)
\end{align*}

% operacion

Operación definida en $A\times Q$ a partir de la operación de $E$
\begin{align*}
	(a_1,q_1)(a_2,q_2) = (a_1+q_1\cdot a_2 + c(q_1,q_2) - q_1\cdot x,q_1q_2)
\end{align*}
	
% identidad
Identidad
\begin{align*}
	&(0,1)(a,q) = (0 + 1\cdot a + c(1,q) - 1\cdot x,q) = (a,q) \\
	&(a,q)(0,1) = (a+q\cdot 0 + c(q,1) - q\cdot x,q) = (a,q)
\end{align*}
	
% inverso
Inverso
\begin{align*}
	(a,q)(x-q^{-1}\cdot c(q,q^{-1}) - q^{-1}\cdot a,q^{-1}) &= (a + q\cdot x - c(q,q^{-1}) -a + c(q,q^{-1}) - q\cdot x, 1) = (0,1) \\
	(x-q^{-1}\cdot c(q,q^{-1}) - q^{-1}\cdot a,q^{-1})(a,q) &= (x-q^{-1}\cdot c(q,q^{-1}) - q^{-1}\cdot a + q^{-1}\cdot a + c(q^{-1},q) - q^{-1}\cdot x, 1) =
\end{align*}

\begin{align*}
	&x -q^{-1}\cdot c(q,q^{-1}) + c(q^{-1},q) - q^{-1}\cdot x = \\
	&x - c(1,q^{-1}) + c(q^{-1},1) - c(q^{-1},q) + c(q^{-1},q) - q^{-1}\cdot x = \\
	&x - c(1,q^{-1}) + c(q^{-1},1) - q^{-1}\cdot x = \tilde c(q^{-1},1)-\tilde c(1,q^{-1}) = 0
\end{align*}
	
% asociatividad
Asociatividad
\begin{align*}
	&(a_1,q_1)(a_2,q_2)(a_3,q_3) =\\
	 &(a_1 + q_1\cdot a_2 + c(q_1,q_2) - q_1\cdot x,q_1q_2)(a_3,q_3) =\\
	  &(a_1 + q_1\cdot a_2 + c(q_1,q_2) - q_1\cdot x + q_1q_2\cdot a_3 + c(q_1q_2,q_3)-q_1q_2\cdot x,q_1q_2q_3)
\end{align*}

\begin{align*}
	&(a_1,q_1)(a_2,q_2)(a_3,q_3) = \\ 
	&(a_1,q_1)(a_2+q_2\cdot a_3 + c(q_2,q_3) - q_2\cdot x,q_2q_3) =\\
	&(a_1+q_1\cdot a_2 + q_1q_2\cdot a_3 + q_1\cdot c(q_2,q_3) - q_1q_2\cdot x + c(q_1,q_2q_3) - q_1\cdot x,q_1q_2q_3)
\end{align*}
% !TeX root=../tfg2.tex
\chapter{Suma de Baer}\label{baersum}


\begin{lema}\label{prop:pullback}
	Dada una extensión $\extension i \pi A E Q$ y un homomorfismo $\homo \alpha {Q'} Q$, existe una extensión $\extension {} {} A {E'} {Q'}$ única salvo equivalencia que hace al siguiente diagrama conmutativo.
	% https://q.uiver.app/?q=WzAsMTAsWzEsMCwiQSJdLFsyLDAsIkUiXSxbMywwLCJRIl0sWzQsMCwiMSJdLFswLDAsIjEiXSxbMCwxLCIxIl0sWzEsMSwiQSJdLFszLDEsIlEnIl0sWzQsMSwiMSJdLFsyLDEsIkUnIl0sWzQsMF0sWzAsMSwiaSJdLFsxLDIsIlxccGkiXSxbMiwzXSxbNSw2XSxbNiw5LCJpJyJdLFs5LDcsIlxccGknIl0sWzcsOF0sWzAsNiwiIiwxLHsib2Zmc2V0IjotMSwic3R5bGUiOnsiaGVhZCI6eyJuYW1lIjoibm9uZSJ9fX1dLFs2LDAsIiIsMSx7Im9mZnNldCI6LTEsInN0eWxlIjp7ImhlYWQiOnsibmFtZSI6Im5vbmUifX19XSxbNywyLCJcXGFscGhhIl0sWzksMSwiZiJdXQ==
	\[\begin{tikzcd}
		1 & A & E & Q & 1 \\
		1 & A & {E'} & {Q'} & 1
		\arrow[from=1-1, to=1-2]
		\arrow["i", from=1-2, to=1-3]
		\arrow["\pi", from=1-3, to=1-4]
		\arrow[from=1-4, to=1-5]
		\arrow[from=2-1, to=2-2]
		\arrow["{i'}", from=2-2, to=2-3]
		\arrow["{\pi'}", from=2-3, to=2-4]
		\arrow[from=2-4, to=2-5]
		\arrow[shift left=1, no head, from=1-2, to=2-2]
		\arrow[shift left=1, no head, from=2-2, to=1-2]
		\arrow["\alpha", from=2-4, to=1-4]
		\arrow["f", from=2-3, to=1-3]
	\end{tikzcd}\]
	\begin{demostracion}
		Tomamos $E'=E\times_Q Q' = \{(e,q)\in E\times Q \ | \ \pi(e) = \alpha(q)\}$ el pullback de $\pi$ y $\alpha$ con la inclusión $i'$ y proyección $\pi'$ definidas por
		\begin{gather*}
			i'(a) = (i(a),1) \\
			\pi'(e,q) = q
		\end{gather*}
		
		Entonces $f(e,q) = e$ hace al diagrama conmutativo, ya que 
		\begin{gather*}
			(f\circ i') (a) =  i(a)\\
			(\pi\circ f)(e,q) = \pi(e) = \alpha(q) = (\alpha\circ \pi')(e,q)
		\end{gather*}
			
		Finalmente comprobamos que es exacta
		\begin{equation*}
			\pi'(i'(a)) = \pi'(a,1)=1
		\end{equation*}
		
		Para probar que es única salvo equivalencia
	\end{demostracion}
\end{lema}

\begin{lema}\label{prop:pushout}
	Dada una extensión $\extension i \pi A E Q$ y un homomorfismo $\homo \beta {A} {A'}$, existe una extensión $\extension {} {} {A'} {E'} Q$ única salvo equivalencia que hace al siguiente diagrama conmutativo.
	% https://q.uiver.app/?q=WzAsMTAsWzEsMCwiQSJdLFsyLDAsIkUiXSxbMywwLCJRIl0sWzQsMCwiMSJdLFswLDAsIjEiXSxbMCwxLCIxIl0sWzEsMSwiQSciXSxbMywxLCJRIl0sWzQsMSwiMSJdLFsyLDEsIkUnIl0sWzQsMF0sWzAsMSwiaSJdLFsxLDIsIlxccGkiXSxbMiwzXSxbNSw2XSxbNiw5LCJpJyJdLFs5LDcsIlxccGknIl0sWzcsOF0sWzEsOSwiZiJdLFs3LDIsIiIsMSx7Im9mZnNldCI6LTEsInN0eWxlIjp7ImhlYWQiOnsibmFtZSI6Im5vbmUifX19XSxbMiw3LCIiLDEseyJvZmZzZXQiOi0xLCJzdHlsZSI6eyJoZWFkIjp7Im5hbWUiOiJub25lIn19fV0sWzAsNiwiXFxiZXRhIl1d
	\[\begin{tikzcd}
		1 & A & E & Q & 1 \\
		1 & {A'} & {E'} & Q & 1
		\arrow[from=1-1, to=1-2]
		\arrow["i", from=1-2, to=1-3]
		\arrow["\pi", from=1-3, to=1-4]
		\arrow[from=1-4, to=1-5]
		\arrow[from=2-1, to=2-2]
		\arrow["{i'}", from=2-2, to=2-3]
		\arrow["{\pi'}", from=2-3, to=2-4]
		\arrow[from=2-4, to=2-5]
		\arrow["f", from=1-3, to=2-3]
		\arrow[shift left=1, no head, from=2-4, to=1-4]
		\arrow[shift left=1, no head, from=1-4, to=2-4]
		\arrow["\beta", from=1-2, to=2-2]
	\end{tikzcd}\]
	\begin{demostracion}
		Tomamos $E'=E\sqcup_A A' = \frac{E\times A'}{\{(i(a),-\beta(a))\in E\times A' \ | \ a\in A\}}$ el pushout de $i$ y $\beta$ con la inclusión $i'$ y proyección $\pi'$ definidas por
		\begin{gather*}
			i'(a) = \overline{(1,a)} \\
			\pi'(\overline{(e,a)}) = \pi(e)
		\end{gather*}
		
		Entonces $f(e) = \overline{(e,0)}$ hace al diagrama conmutativo, ya que 
		\begin{gather*}
			(i'\circ \beta)(a) = \overline{(1,\beta(a))} = \overline{(i(a),0)} = (f\circ i)(a)\\
			(\pi'\circ f)(e) = \pi(e)
		\end{gather*}
		
		Finalmente se prueba que la sucesión es exacta
		\begin{equation*}
			(\pi¡\circ i')(a) = \pi'(\overline{(1,a)}) = \pi(1) = 1
		 \end{equation*}
		 
		Para probar que es única salvo equivalencia
	\end{demostracion}
\end{lema}

\begin{teorema}
	Dadas dos extensiones $\extension{i_j} {\pi_j} A {E_j} Q$, para $j=1,2$, podemos expresar la suma de extensiones dada por la estructura aditiva de $H^2(Q,A)$ descrita en la Proposición \ref{extsum} con el siguiente diagrama conmutativo. La última fila se conoce como la suma de Baer de las extensiones $E_1$ y $E_2$.
	
	% https://q.uiver.app/?q=WzAsMTUsWzEsNCwiQSJdLFszLDQsIlxcZnJhY3tFXzFcXHRpbWVzX1EgRV8yfXtLZXIoKyl9Il0sWzUsNCwiUSJdLFszLDAsIkVfMVxcdGltZXMgRV8yIl0sWzEsMiwiQVxcdGltZXMgQSJdLFs1LDIsIlEiXSxbMSwwLCJBXFx0aW1lcyBBIl0sWzUsMCwiUVxcdGltZXMgUSJdLFszLDIsIkVfMVxcdGltZXNfUSBFXzIiXSxbMCw0LCIxIl0sWzAsMiwiMSJdLFswLDAsIjEiXSxbNiw0LCIxIl0sWzYsMiwiMSJdLFs2LDAsIjEiXSxbNiwzLCJpXzFcXHRpbWVzIGlfMiIsMCx7InN0eWxlIjp7InRhaWwiOnsibmFtZSI6Imhvb2siLCJzaWRlIjoidG9wIn19fV0sWzMsNywiXFxwaV8xXFx0aW1lcyBcXHBpXzIiLDAseyJvZmZzZXQiOi0xLCJzdHlsZSI6eyJoZWFkIjp7Im5hbWUiOiJlcGkifX19XSxbNCw4LCJcXHRpbGRlXFxpbWF0aCIsMCx7InN0eWxlIjp7InRhaWwiOnsibmFtZSI6Imhvb2siLCJzaWRlIjoidG9wIn19fV0sWzgsNSwiXFx0aWxkZVxccGkiLDAseyJvZmZzZXQiOi0xLCJzdHlsZSI6eyJoZWFkIjp7Im5hbWUiOiJlcGkifX19XSxbNCwwLCIrIiwyLHsic3R5bGUiOnsiaGVhZCI6eyJuYW1lIjoiZXBpIn19fV0sWzAsMSwiaV8zIiwwLHsic3R5bGUiOnsidGFpbCI6eyJuYW1lIjoiaG9vayIsInNpZGUiOiJ0b3AifX19XSxbMSwyLCJcXHBpXzMiLDAseyJvZmZzZXQiOi0xLCJzdHlsZSI6eyJoZWFkIjp7Im5hbWUiOiJlcGkifX19XSxbNywzLCJzXzFcXHRpbWVzIHNfMiIsMCx7Im9mZnNldCI6LTEsInN0eWxlIjp7InRhaWwiOnsibmFtZSI6Imhvb2siLCJzaWRlIjoidG9wIn19fV0sWzIsMSwic18zIiwwLHsib2Zmc2V0IjotMSwic3R5bGUiOnsidGFpbCI6eyJuYW1lIjoiaG9vayIsInNpZGUiOiJ0b3AifX19XSxbNSw3LCJcXERlbHRhIiwwLHsic3R5bGUiOnsidGFpbCI6eyJuYW1lIjoiaG9vayIsInNpZGUiOiJ0b3AifX19XSxbNSw4LCJcXHRpbGRlIHMiLDAseyJvZmZzZXQiOi0xLCJzdHlsZSI6eyJ0YWlsIjp7Im5hbWUiOiJob29rIiwic2lkZSI6InRvcCJ9fX1dLFs4LDMsImkiLDAseyJzdHlsZSI6eyJ0YWlsIjp7Im5hbWUiOiJob29rIiwic2lkZSI6InRvcCJ9fX1dLFs2LDQsIiIsMSx7Im9mZnNldCI6LTEsInN0eWxlIjp7ImhlYWQiOnsibmFtZSI6Im5vbmUifX19XSxbNCw2LCIiLDEseyJvZmZzZXQiOi0xLCJzdHlsZSI6eyJoZWFkIjp7Im5hbWUiOiJub25lIn19fV0sWzIsNSwiIiwyLHsib2Zmc2V0IjotMSwic3R5bGUiOnsiaGVhZCI6eyJuYW1lIjoibm9uZSJ9fX1dLFs1LDIsIiIsMix7Im9mZnNldCI6LTEsInN0eWxlIjp7ImhlYWQiOnsibmFtZSI6Im5vbmUifX19XSxbOCwxLCJcXHBpX3tLZXIoKyl9IiwyLHsic3R5bGUiOnsiaGVhZCI6eyJuYW1lIjoiZXBpIn19fV0sWzExLDZdLFsxMCw0XSxbOSwwXSxbNywxNF0sWzUsMTNdLFsyLDEyXV0=
		\begin{tikzcd}
			1 & {A\times A} && {E_1\times E_2} && {Q\times Q} & 1 \\
			\\
			1 & {A\times A} && {E_1\times_Q E_2} && Q & 1 \\
			\\
			1 & A && {\frac{E_1\times_Q E_2}{\Ker(+)}} && Q & 1
			\arrow["{i_1\times i_2}", hook, from=1-2, to=1-4]
			\arrow["{\pi_1\times \pi_2}", shift left=1, two heads, from=1-4, to=1-6]
			\arrow["\tilde\imath", hook, from=3-2, to=3-4]
			\arrow["\tilde\pi", shift left=1, two heads, from=3-4, to=3-6]
			\arrow["{+}"', two heads, from=3-2, to=5-2]
			\arrow["{i_3}", hook, from=5-2, to=5-4]
			\arrow["{\pi_3}", shift left=1, two heads, from=5-4, to=5-6]
			\arrow["{s_1\times s_2}", shift left=1, hook, from=1-6, to=1-4]
			\arrow["{s_3}", shift left=1, hook, from=5-6, to=5-4]
			\arrow["\Delta", hook, from=3-6, to=1-6]
			\arrow["{\tilde s}", shift left=1, hook, from=3-6, to=3-4]
			\arrow["i", hook, from=3-4, to=1-4]
			\arrow[shift left=1, no head, from=1-2, to=3-2]
			\arrow[shift left=1, no head, from=3-2, to=1-2]
			\arrow[shift left=1, no head, from=5-6, to=3-6]
			\arrow[shift left=1, no head, from=3-6, to=5-6]
			\arrow["{\pi_{\Ker(+)}}"', two heads, from=3-4, to=5-4]
			\arrow[from=1-1, to=1-2]
			\arrow[from=3-1, to=3-2]
			\arrow[from=5-1, to=5-2]
			\arrow[from=1-6, to=1-7]
			\arrow[from=3-6, to=3-7]
			\arrow[from=5-6, to=5-7]
		\end{tikzcd}
		
		donde $\Delta$ es la inclusión diagonal, $+$ es la proyección a través de la suma de las componentes y $E_1\times_Q E_2$ es el pullback de $E_1$ y $E_2$.
		\begin{gather*}
			+(a_1,a_2) = a_1+a_2 \\
			\Delta(q) = (q,q) \\
			E_1\times_Q E_2 = \{(e_1,e_2)\in E_1\times E_2\ |\ \pi_1(e_1)=\pi_2(e_2)\}
		\end{gather*}
		
		\begin{demostracion}
			A partir de las extensiones $E_1$ y $E_2$ se construye la extensión del producto directo tomando la inclusión y proyección coordenada a coordenada. El objetivo será utilizar los cociclos $c_1$ y $c_2$ asociados a las extensiones $E_1$ y $E_2$ respectivamente para construir una sucesión exacta $\extension {i_3} {\pi_3} A {E_3} Q$ cuyo cociclo asociado sea $c_3=c_1+c_2$.
		\begin{equation}
			1\xrightarrow{} A\times A \xrightarrow{i_1\times i_2} E_1\times E_2\xrightarrow{\pi_1\times\pi_2} Q\times Q \xrightarrow{} 1 % comprobar que es exacta
		\end{equation}
		La sección $s_1\times s_2$ de $\pi_1\times \pi_2$ tiene como cociclo asociado
		\begin{align}
			(c_1\times c_2)\colon (Q\times Q)\times (Q\times Q) &\to A\times A \nonumber\\
			                              ((q_{11},q_{12}),(q_{21},q_{22})) &\mapsto (c_1(q_{11},q_{12}),c_2(q_{21},q_{22})) % comprobar el orden de la operacion
		\end{align}
		Proyectando $A\times A$ sobre $A$ haciendo la suma de componentes movemos $(c_1\times c_2)((q_{11},q_{12}),(q_{21},q_{22}))$ a $c_1(q_{11},q_{12}) + c_2(q_{21},q_{22})$. Ahora identificando $q_{11}$ con $q_{21}$ y $q_{12}$ con $q_{22}$ mediante la inclusión diagonal $\homo \Delta Q {Q\times Q}$, el cociclo asociado a la ultima fila es $c_3=+\circ (c_1\times c_2)\circ \Delta=c_1+c_2$
		
		%en verdad seria \Delta\times\Delta
		\begin{equation}
			Q\times Q \xrightarrow{\Delta} (Q\times Q)\times (Q\times Q) \xrightarrow{c_1\times c_2} A\times A\xrightarrow{+} A
		\end{equation}
		
		A continuación, buscamos el grupo que hace que la extensión de $Q$ por $A$ conmute con el diagrama.
		
		Por el Lema \ref{prop:pullback}, tenemos el siguiente diagrama conmutativo
		% https://q.uiver.app/?q=WzAsMTAsWzEsMCwiQVxcdGltZXMgQSJdLFsyLDAsIkVfMVxcdGltZXMgRV8yIl0sWzMsMCwiUVxcdGltZXMgUSJdLFs0LDAsIjEiXSxbMCwwLCIxIl0sWzAsMSwiMSJdLFsxLDEsIkFcXHRpbWVzIEEiXSxbMywxLCJRIl0sWzQsMSwiMSJdLFsyLDEsIihFXzFcXHRpbWVzIEVfMilcXHRpbWVzX3tRXFx0aW1lcyBRfVEiXSxbNCwwXSxbMCwxLCJpXzFcXHRpbWVzIGlfMiJdLFsxLDIsIlxccGlfMVxcdGltZXMgXFxwaV8yIl0sWzIsM10sWzUsNl0sWzYsOSwiXFx0aWxkZSBpIl0sWzcsOF0sWzAsNiwiIiwxLHsib2Zmc2V0IjotMSwic3R5bGUiOnsiaGVhZCI6eyJuYW1lIjoibm9uZSJ9fX1dLFs2LDAsIiIsMSx7Im9mZnNldCI6LTEsInN0eWxlIjp7ImhlYWQiOnsibmFtZSI6Im5vbmUifX19XSxbNywyLCJcXERlbHRhIl0sWzksMSwiaSJdLFs5LDcsIlxcdGlsZGVcXHBpIl1d
		\[\begin{tikzcd}
			1 & {A\times A} & {E_1\times E_2} & {Q\times Q} & 1 \\
			1 & {A\times A} & {(E_1\times E_2)\times_{Q\times Q}Q} & {Q} & 1
			\arrow[from=1-1, to=1-2]
			\arrow["{i_1\times i_2}", from=1-2, to=1-3]
			\arrow["{\pi_1\times \pi_2}", from=1-3, to=1-4]
			\arrow[from=1-4, to=1-5]
			\arrow[from=2-1, to=2-2]
			\arrow["{\tilde \imath}", from=2-2, to=2-3]
			\arrow[from=2-4, to=2-5]
			\arrow[shift left=1, no head, from=1-2, to=2-2]
			\arrow[shift left=1, no head, from=2-2, to=1-2]
			\arrow["\Delta", from=2-4, to=1-4]
			\arrow["i", from=2-3, to=1-3]
			\arrow["\tilde\pi", from=2-3, to=2-4]
		\end{tikzcd}\]
		
		Observese que el pullback de $\pi_1\times \pi_2$ y $\Delta$ es igual que el pullback de $\pi_1$y $\pi_2$.
		\begin{align*}
			(E_1\times E_2)\times_{Q\times Q}Q 
			&= \{(e_1,e_2,q)\in E_1\times E_2\times Q \ | \ (\pi_1\times \pi_2)(e_1,e_2) = (q,q)\} \\
			&= \{(e_1,e_2)\in E_1\times E_2 \ | \ \pi_1(e_1)=\pi_2(e_2)\} \\
			&= E_1\times_Q E_2
		\end{align*}
		
		Por el Lema \ref{prop:pushout}, tenemos el siguiente diagrama conmutativo
		
		\end{demostracion}
\end{teorema}



\begin{thebibliography}{9999}
\addcontentsline{toc}{chapter}{Bibliograf\'{\i}a}


\bibitem{Byrd}
P.~F.~Byrd, New relations between Fibonacci and Bernoulli numbers, Fibonacci Quat. \textbf{13} (1975), 59-69.

\bibitem{GAP}
The GAP Group, GAP Groups, Algorithms and Programming, Version 4.4.12, 2008 (http://www.gap-system.org).

\bibitem{Hungerford} T.~W.~Hungerford, \textsl{Algebra}, Graduate Texts in Mathematics, Springer-Verlag, New York, 1974.

\bibitem{Spivak}
M.~Spivak, \textsl{Calculus}, $3^{\text{a}}$ ed.,
Revert\'e, Barcelona, 2012.


\end{thebibliography} 

\end{document}