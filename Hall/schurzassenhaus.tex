% !TeX root=../tfg2.tex

\begin{teorema}[Schur-Zassenhaus] Sea $G$ un grupo finito y $N\norm G$ un subgrupo de Hall, es decir, $\ord{N}=n$ y $\ord{G:N}=m$ con $\mcd(n,m)=1$. Entonces $G$ contiene subgrupos de orden $m$ y son conjugados. % explicar que G es producto semidirecto de N por Q
	\begin{demostracion}
		El caso $N$ abeliano ya ha sido demostrado en el Teorema \ref{thmschurab}. Basta demostrar la existencia y la conjugación en el caso general.
		
		\textit{(i) Existencia.} Lo demostramos por inducción fuerte sobre el orden de $G$. 
		Podemos tomar como caso base $C_p$, que como es abeliano se cumple el resultado. 
		Para el caso general, tomemos un primo $p \big| n$ y $P\in \Syl p N$. Sean $L=\Norm G P$ y $C=\Center(P)$.
		Como $C$ es característico en $P$ y $P\norm L$, tenemos que $C\norm L$.
		Por el argumento de Frattini, $G=LN$.
		Observamos que $\ord{L:N\cap L} = \ord{LN:N} = m$ y que $N\cap L \norm L$ por el Segundo Teorema de Isomorfía.
		
		El subgrupo $C$ es no trivial ya que es el centro de un $p$-subgrupo. Aplicando inducción sobre el grupo $L/C$, existe $H/C\leq L/C$ de orden $m$ y volviendo a $G$ por el Teorema de Correspondencia, existe un subgrupo $H$ de índice $m$ en $C$. Aplicando el caso abeliano a $C$ y $H$ se concluye que existe un subgrupo $Q\leq H$ de orden $m$.
%		\begin{figure}[h!]
%			\begin{tikzpicture}[node distance=2cm, scale=1, transform shape]
%				\title{Diagrama}
%				
%				\node(G) {$G$};
%				\node(N)		[below right = 1cm and 3cm of G] {$N$};
%				\node(L)		[below = 1.5cm of G] {$L=N_{G}(P)$};
%				\node(NcL)	[below = 1.5cm of N] {$N\cap L$};
%				\node(P)		[below = 0.5cm of NcL] {$P$};
%				\node(C)		[below = 1.5cm of NcL] {$C=Z(P)$};
%				\node(H)		[below = 1.5cm of L] {$H$};
%				\node(Q)		[below = 1.5cm of H] {$Q$};
%				\node(C1)	[below = 1.5cm of C] {$1$};
%				
%				
%				\draw(G)-- node[above]{$\norm$} node[below]{$m$}(N);
%				\draw(G)--(L);
%				\draw(N)--(NcL);
%				\draw(L)-- node[above]{$\norm$} node[below]{$m$}(NcL);
%				\draw(L)--(H);
%				%\draw(NcL)--(C);
%				\draw(NcL)--(P);
%				\draw(P)--(C);
%				\draw(H)-- node[above]{$\norm$} node[below]{$m$}(C);
%				\draw(H)--(Q);
%				\draw(Q)-- node[above]{$\norm$} node[below]{$m$}(C1);
%				\draw(C)--(C1);
%	
%			\end{tikzpicture}
%		\end{figure}
		
		\textit{(ii) Conjugación (Caso N resoluble)}. Sean $Q_1$ y $Q_2$ subgrupos de $G$ de orden $m$. Como $N$ es resoluble $N'\neq N$, $N'\car N\norm G$ y por tanto, $N'\norm G$. Aplicando el caso abeliano a $N/N'$ y $G/N'$, $Q_1N'/N'$ y $Q_2N'/N'$ son conjugados. Por tanto, existe $g_1\in G$ tal que $Q_1^{g_1} \leq Q_2 N'$. % aclarar esto
		$Q_1^{g_1}$ y $Q_2$ son subgrupos de orden $m$ en $Q_2N'$ y aplicando inducción sobre la longitud derivada de $N$ se llega a $Q_1^{g_d}\leq Q_2N^{(d)} = Q_2$.
		
		\textit{(iii) Conjugación (Caso G/N resoluble)}.
	\end{demostracion}
\end{teorema}