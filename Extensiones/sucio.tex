% !TeX root=../tfg2.tex

\chapter{Extensiones sucio}




% definir Q modulos
% cociclos cobordes
% cambiar la accion a derecha
\begin{definicion}
	Sea $A$ un $Q$-módulo dado por una acción $\homo \varphi Q A$ y $n\in\N$. 
	Un $n$-cociclo es una función $\homo f {Q^n} A$ tal que $\forall q_1,\hdots,q_{n+1}\in Q$ se verifica
	\begin{equation*}
		q_1\cdot f(q_2,\hdots,q_{n+1}) + \left[\sum\limits_{i=1}^{n}(-1)^i f(q_1,\hdots,q_iq_{i+1},\hdots,q_{n+1}) \right] + (-1)^{n+1}f(q_1,\hdots,q_n) = 0
	\end{equation*}
	
	Un $n$-coborde es una función $\homo f {Q^n} A$ tal que existe una función $\homo \phi {Q^{n-1}} A$ tal que
	
	\begin{equation*}
		f(q_1,\hdots, q_n) = q_1\cdot \phi(q_2,\hdots,q_{n+1}) + \left[\sum\limits_{i=1}^{n}(-1)^i \phi(q_1,\hdots,q_iq_{i+1},\hdots,q_{n+1}) \right] + (-1)^{n+1}\phi(q_1,\hdots,q_n)
	\end{equation*} 
	
	Los cociclos y cobordes heredan de $A$ una estructura de grupo abeliano. A estos grupos los denotamos $Z^n_\varphi(Q,A)$ y $B^n_\varphi(Q,A)$ respectivamente.
\end{definicion}

\begin{proposicion}
	\begin{demostracion}
		A partir de las extensiones $E_1$ y $E_2$ se construye la extensión del producto directo tomando la inclusión y proyección coordenada a coordenada. El objetivo será utilizar los cociclos $c_1$ y $c_2$ para construir una sucesión exacta $1\xrightarrow{} A\xrightarrow{i_3} E_3\xrightarrow{\pi_3} Q \xrightarrow{} 1$ cuyo cociclo asociado sea $c_3=c_1+c_2$.
		\begin{equation}
			1\xrightarrow{} A\times A \xrightarrow{i_1\times i_2} E_1\times E_2\xrightarrow{\pi_1\times\pi_2} Q\times Q \xrightarrow{} 1 % comprobar que es exacta
		\end{equation}
		La sección $s_1\times s_2$ de $\pi_1\times \pi_2$ tiene como cociclo asociado
		\begin{align}
			(c_1\times c_2)\colon (Q\times Q)\times (Q\times Q) &\to A\times A \\
			                              ((q_{11},q_{12}),(q_{21},q_{22})) &\mapsto (c_1(q_{11},q_{12}),c_2(q_{21},q_{22})) % comprobar el orden de la operacion
		\end{align}
		Proyectando $A\times A$ sobre $A$ y haciendo la suma de componentes en $A$ movemos $(c_1\times c_2)((q_{11},q_{12}),(q_{21},q_{22}))$ a $c_1(q_{11},q_{12}) + c_2(q_{21},q_{22})$. Basta identificar $q_{11}$ con $q_{21}$ y $q_{12}$ con $q_{22}$ mediante $\homo \Delta Q {Q\times Q}$ definido por $\Delta(q)=(q,q)$. Notese que el morfismo diagonal está definido para las secciones $s_1$ y $s_2$, por tanto, para los cociclos estará definido de $Q\times Q$ en $(Q\times Q)\times (Q\times Q)$.
		
		\begin{equation}
			Q\times Q \xrightarrow{\Delta} (Q\times Q)\times (Q\times Q) \xrightarrow{c_1\times c_2} A\times A\xrightarrow{+} A
		\end{equation}
		El cociclo que buscamos es $c_3=+\circ (c_1\times c_2)\circ \Delta=c_1+c_2$
		
		Completando el diagrama de $\Delta$ y $\pi_1 \times \pi_2$ con $\homo i {\tilde E} {E_1\times E_2}$ y $\homo {\tilde\pi} E Q$, para $x\in \tilde E$ $(\pi_1(i(x)),\pi_2(i(x)))=(\tilde\pi(x),\tilde\pi(x))$ lo que implica que $\pi_1(i(x))=\pi_2(i(x))$ y por tanto $i(x)\in E_1\times_Q E_2$. Lo natural es tomar $\tilde E = E_1\times_Q E_2$, $i$ la inclusión y $\tilde\pi(e_1,e_2)=\pi_1(e_1)=\pi_2(e_2)$.
		
		% https://q.uiver.app/?q=WzAsNCxbMCwwLCJFXzFcXHRpbWVzIEVfMiJdLFsyLDAsIlFcXHRpbWVzIFEiXSxbMiwxLCJRIl0sWzAsMSwiXFx0aWxkZSBFIl0sWzMsMCwiaSJdLFsxLDAsInNfMVxcdGltZXMgc18yIl0sWzIsMSwiXFxEZWx0YSJdLFsyLDMsIlxcdGlsZGUgcyJdXQ==
		\[\begin{tikzcd}
			{E_1\times E_2} && {Q\times Q} \\
			{\tilde E} && Q
			\arrow["i", from=2-1, to=1-1]
			\arrow["{\pi_1\times \pi_2}", from=1-1, to=1-3]
			\arrow["\Delta", from=2-3, to=1-3]
			\arrow["{\tilde \pi}", from=2-1, to=2-3]
		\end{tikzcd}\]
		
		$\Ker(\tilde \pi) = \{(e_1,e_2)\in E_1\times_Q E_2 \colon \tilde\pi(e_1,e_2)=0\}=\{(e_1,e_2)\in E_1\times_Q E_2 \colon \pi_1(e_1)=\pi_2(e_2)=0\} = A\times A$
	
		Por tanto, la sucesión $1\xrightarrow{} A\times A \xrightarrow{\tilde\imath} E_1\times_Q E_2\xrightarrow{\tilde\pi} Q \xrightarrow{} 1$ es exacta.
		
		$i_3(a_1+a_2)=\pi(a_1,a_2)=\pi(a_1+a_2,0)=\pi(0,a_1+a_2)$, para que esté bien definida, tenemos que cocientar $E_1\times_Q E_2$ por $Ker(+)=\{(a_1,a_2)\in A\times A \colon a_1+a_2=0\}=\{(a,-a) \colon a\in A\}$
		
		
		% Fin
		El cociclo correspondiente a la extensión es por tanto $c_1+c_2$ y por la proposición \ref{extsum} la suma está bien definida en clases de extensiones equivalentes.
	\end{demostracion}
\end{proposicion}
